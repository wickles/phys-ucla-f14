% ***********************************************************
% ******************* PHYSICS HEADER ************************
% ***********************************************************
% Version 2
\documentclass[12pt]{article} 





\usepackage{datetime} % allows easy formatting of dates, e.g. \formatdate{dd}{mm}{yyyy}

\usepackage{amsmath} % AMS Math Package
\usepackage{amsthm} % Theorem Formatting
\usepackage{amssymb}	% Math symbols such as \mathbb
\usepackage{graphicx} % Allows for eps images
\usepackage{multicol} % Allows for multiple columns
\usepackage[dvips,letterpaper,margin=1in,bottom=1in]{geometry}
 % Sets margins and page size
\pagestyle{empty} % Removes page numbers
\makeatletter % Need for anything that contains an @ command 
%\renewcommand{\maketitle} % Redefine maketitle to conserve space
%{ \begingroup \vskip 10pt \begin{center} \large {\bf \@title}
%	\vskip 10pt \large \@author \hskip 20pt \@date \end{center}
%  \vskip 10pt \endgroup \setcounter{footnote}{0} }
\makeatother % End of region containing @ commands
\renewcommand{\labelenumi}{(\alph{enumi})} % Use letters for enumerate
% \DeclareMathOperator{\Sample}{Sample}
\let\vaccent=\v % rename builtin command \v{} to \vaccent{}
\renewcommand{\v}[1]{\ensuremath{\mathbf{#1}}} % for vectors
\newcommand{\gv}[1]{\ensuremath{\mbox{\boldmath$ #1 $}}} 
% for vectors of Greek letters
\newcommand{\vx}{\ensuremath{\v{x}}} 
% for vectors of Greek letters
\newcommand{\vy}{\ensuremath{\v{y}}} 
% for vectors of Greek letters
\newcommand{\xdot}{\ensuremath{\dot{x}}} 
% for vectors of Greek letters

\newcommand{\ydot}{\ensuremath{\dot{y}}} 
% for vectors of Greek letters
\usepackage{commath} % for some nice standardized syntax stuff. 
	% \dif, \Dif, \od, \pd, \md, \(abs | envert), \(norm | enVert), \(set | cbr), \sbr, \eval, \int(o | c)(o | c), etc
\newcommand{\bbar}[1]{\bar{\bar{#1}}} % for barring things twice -- use \cbar or \zbar instead of original \bbar

\newcommand{\uv}[1]{\ensuremath{\mathbf{\hat{#1}}}} % for unit vector
\newcommand{\abs}[1]{\left| #1 \right|} % for absolute value
\newcommand{\avg}[1]{\left< #1 \right>} % for average
\let\underdot=\d % rename builtin command \d{} to \underdot{}
\renewcommand{\d}[2]{\frac{d #1}{d #2}} % for derivatives
\newcommand{\dd}[2]{\frac{d^2 #1}{d #2^2}} % for double derivatives
\newcommand{\pd}[2]{\frac{\partial #1}{\partial #2}} 
% for partial derivatives
\newcommand{\fd}[2]{\frac{\delta #1}{\delta #2}} 
% for functional derivatives

\newcommand{\pdd}[2]{\frac{\partial^2 #1}{\partial #2^2}} 
% for double partial derivatives
\newcommand{\pdc}[3]{\left( \frac{\partial #1}{\partial #2}
 \right)_{#3}} % for thermodynamic partial derivatives
\newcommand{\ket}[1]{\left| #1 \right>} % for Dirac bras
\newcommand{\bra}[1]{\left< #1 \right|} % for Dirac kets
\newcommand{\braket}[2]{\left< #1 \vphantom{#2} \right|
 \left. #2 \vphantom{#1} \right>} % for Dirac brackets
\newcommand{\matrixel}[3]{\left< #1 \vphantom{#2#3} \right|
 #2 \left| #3 \vphantom{#1#2} \right>} % for Dirac matrix elements
\newcommand{\grad}[1]{\gv{\nabla} #1} % for gradient
\let\divsymb=\div % rename builtin command \div to \divsymb
\renewcommand{\div}[1]{\gv{\nabla} \cdot #1} % for divergence
\newcommand{\curl}[1]{\gv{\nabla} \times #1} % for curl
\let\baraccent=\= % rename builtin command \= to \baraccent
\renewcommand{\=}[1]{\stackrel{#1}{=}} % for putting numbers above =
\newtheorem{prop}{Proposition}
\newtheorem{thm}{Theorem}[section]
\newtheorem{lem}[thm]{Lemma}
\theoremstyle{definition}
\newtheorem{dfn}{Definition}
\theoremstyle{remark}
\newtheorem*{rmk}{Remark}
\newcommand{\bigO}{\mathcal{O}} % big O notation
\let \bigo = \bigO % deprecated version. keeping for now because need to update instances in older files










\makeatletter
% À droite
\renewcommand\subsection{\@startsection {subsection}{1}{\z@}%
                                   {-3.5ex \@plus -1ex \@minus -.2ex}%
                                   {2.3ex \@plus.2ex}%
                                   {\raggedright\normalfont\Large\bfseries}}
\makeatother


\makeatletter
\def\section{\@ifstar\unnumberedsection\numberedsection}
\def\numberedsection{\@ifnextchar[%]
  \numberedsectionwithtwoarguments\numberedsectionwithoneargument}
\def\unnumberedsection{\@ifnextchar[%]
  \unnumberedsectionwithtwoarguments\unnumberedsectionwithoneargument}
\def\numberedsectionwithoneargument#1{\numberedsectionwithtwoarguments[#1]{#1}}
\def\unnumberedsectionwithoneargument#1{\unnumberedsectionwithtwoarguments[#1]{#1}}
\def\numberedsectionwithtwoarguments[#1]#2{%
  \ifhmode\par\fi
  \removelastskip
  \vskip 5ex\goodbreak
  \refstepcounter{section}%
  \hbox to \hsize{\vbox{%
      \noindent
      \leavevmode
      \begingroup
      \Large\bfseries\raggedleft
      \thesection.\ 
      #2\par
      \endgroup
      \vskip -2ex
      \noindent\hrulefill
      \vskip -2.2ex\nobreak
      \noindent\hrulefill
      }}\nobreak
  \vskip 2ex\nobreak
  \addcontentsline{toc}{section}{%
    \protect\numberline{\thesection}%
    #1}%
  }
\def\unnumberedsectionwithtwoarguments[#1]#2{%
  \ifhmode\par\fi
  \removelastskip
  \vskip 5ex\goodbreak
%  \refstepcounter{section}%
  \hbox to \hsize{\vbox{%
      \noindent
      \leavevmode
      \begingroup
      \Large\bfseries\raggedleft
%      \thesection.\ 
      #2\par
      \endgroup
      \vskip -2ex
      \noindent\hrulefill
      \vskip -2.2ex\nobreak
      \noindent\hrulefill
      }}\nobreak
  \vskip 2ex\nobreak
  \addcontentsline{toc}{section}{%
%    \protect\numberline{\thesection}%
    #1}%
  }
\makeatother
\pagestyle{empty}




% ***********************************************************
% ********************** END HEADER *************************
% ***********************************************************
