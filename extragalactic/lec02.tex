% declare document class and geometry
%\documentclass[12pt]{article} % use larger type; default would be 10pt
% ***********************************************************
% ******************* PHYSICS HEADER ************************
% ***********************************************************
% Version 2
\documentclass[12pt]{article} 





\usepackage{datetime} % allows easy formatting of dates, e.g. \formatdate{dd}{mm}{yyyy}

\usepackage{amsmath} % AMS Math Package
\usepackage{amsthm} % Theorem Formatting
\usepackage{amssymb}	% Math symbols such as \mathbb
\usepackage{graphicx} % Allows for eps images
\usepackage{multicol} % Allows for multiple columns
\usepackage[dvips,letterpaper,margin=1in,bottom=1in]{geometry}
 % Sets margins and page size
\pagestyle{empty} % Removes page numbers
\makeatletter % Need for anything that contains an @ command 
%\renewcommand{\maketitle} % Redefine maketitle to conserve space
%{ \begingroup \vskip 10pt \begin{center} \large {\bf \@title}
%	\vskip 10pt \large \@author \hskip 20pt \@date \end{center}
%  \vskip 10pt \endgroup \setcounter{footnote}{0} }
\makeatother % End of region containing @ commands
\renewcommand{\labelenumi}{(\alph{enumi})} % Use letters for enumerate
% \DeclareMathOperator{\Sample}{Sample}
\let\vaccent=\v % rename builtin command \v{} to \vaccent{}
\renewcommand{\v}[1]{\ensuremath{\mathbf{#1}}} % for vectors
\newcommand{\gv}[1]{\ensuremath{\mbox{\boldmath$ #1 $}}} 
% for vectors of Greek letters
\newcommand{\vx}{\ensuremath{\v{x}}} 
% for vectors of Greek letters
\newcommand{\vy}{\ensuremath{\v{y}}} 
% for vectors of Greek letters
\newcommand{\xdot}{\ensuremath{\dot{x}}} 
% for vectors of Greek letters

\newcommand{\ydot}{\ensuremath{\dot{y}}} 
% for vectors of Greek letters
\usepackage{commath} % for some nice standardized syntax stuff. 
	% \dif, \Dif, \od, \pd, \md, \(abs | envert), \(norm | enVert), \(set | cbr), \sbr, \eval, \int(o | c)(o | c), etc
\newcommand{\bbar}[1]{\bar{\bar{#1}}} % for barring things twice -- use \cbar or \zbar instead of original \bbar

\newcommand{\uv}[1]{\ensuremath{\mathbf{\hat{#1}}}} % for unit vector
\newcommand{\abs}[1]{\left| #1 \right|} % for absolute value
\newcommand{\avg}[1]{\left< #1 \right>} % for average
\let\underdot=\d % rename builtin command \d{} to \underdot{}
\renewcommand{\d}[2]{\frac{d #1}{d #2}} % for derivatives
\newcommand{\dd}[2]{\frac{d^2 #1}{d #2^2}} % for double derivatives
\newcommand{\pd}[2]{\frac{\partial #1}{\partial #2}} 
% for partial derivatives
\newcommand{\fd}[2]{\frac{\delta #1}{\delta #2}} 
% for functional derivatives

\newcommand{\pdd}[2]{\frac{\partial^2 #1}{\partial #2^2}} 
% for double partial derivatives
\newcommand{\pdc}[3]{\left( \frac{\partial #1}{\partial #2}
 \right)_{#3}} % for thermodynamic partial derivatives
\newcommand{\ket}[1]{\left| #1 \right>} % for Dirac bras
\newcommand{\bra}[1]{\left< #1 \right|} % for Dirac kets
\newcommand{\braket}[2]{\left< #1 \vphantom{#2} \right|
 \left. #2 \vphantom{#1} \right>} % for Dirac brackets
\newcommand{\matrixel}[3]{\left< #1 \vphantom{#2#3} \right|
 #2 \left| #3 \vphantom{#1#2} \right>} % for Dirac matrix elements
\newcommand{\grad}[1]{\gv{\nabla} #1} % for gradient
\let\divsymb=\div % rename builtin command \div to \divsymb
\renewcommand{\div}[1]{\gv{\nabla} \cdot #1} % for divergence
\newcommand{\curl}[1]{\gv{\nabla} \times #1} % for curl
\let\baraccent=\= % rename builtin command \= to \baraccent
\renewcommand{\=}[1]{\stackrel{#1}{=}} % for putting numbers above =
\newtheorem{prop}{Proposition}
\newtheorem{thm}{Theorem}[section]
\newtheorem{lem}[thm]{Lemma}
\theoremstyle{definition}
\newtheorem{dfn}{Definition}
\theoremstyle{remark}
\newtheorem*{rmk}{Remark}
\newcommand{\bigO}{\mathcal{O}} % big O notation
\let \bigo = \bigO % deprecated version. keeping for now because need to update instances in older files










\makeatletter
% À droite
\renewcommand\subsection{\@startsection {subsection}{1}{\z@}%
                                   {-3.5ex \@plus -1ex \@minus -.2ex}%
                                   {2.3ex \@plus.2ex}%
                                   {\raggedright\normalfont\Large\bfseries}}
\makeatother


\makeatletter
\def\section{\@ifstar\unnumberedsection\numberedsection}
\def\numberedsection{\@ifnextchar[%]
  \numberedsectionwithtwoarguments\numberedsectionwithoneargument}
\def\unnumberedsection{\@ifnextchar[%]
  \unnumberedsectionwithtwoarguments\unnumberedsectionwithoneargument}
\def\numberedsectionwithoneargument#1{\numberedsectionwithtwoarguments[#1]{#1}}
\def\unnumberedsectionwithoneargument#1{\unnumberedsectionwithtwoarguments[#1]{#1}}
\def\numberedsectionwithtwoarguments[#1]#2{%
  \ifhmode\par\fi
  \removelastskip
  \vskip 5ex\goodbreak
  \refstepcounter{section}%
  \hbox to \hsize{\vbox{%
      \noindent
      \leavevmode
      \begingroup
      \Large\bfseries\raggedleft
      \thesection.\ 
      #2\par
      \endgroup
      \vskip -2ex
      \noindent\hrulefill
      \vskip -2.2ex\nobreak
      \noindent\hrulefill
      }}\nobreak
  \vskip 2ex\nobreak
  \addcontentsline{toc}{section}{%
    \protect\numberline{\thesection}%
    #1}%
  }
\def\unnumberedsectionwithtwoarguments[#1]#2{%
  \ifhmode\par\fi
  \removelastskip
  \vskip 5ex\goodbreak
%  \refstepcounter{section}%
  \hbox to \hsize{\vbox{%
      \noindent
      \leavevmode
      \begingroup
      \Large\bfseries\raggedleft
%      \thesection.\ 
      #2\par
      \endgroup
      \vskip -2ex
      \noindent\hrulefill
      \vskip -2.2ex\nobreak
      \noindent\hrulefill
      }}\nobreak
  \vskip 2ex\nobreak
  \addcontentsline{toc}{section}{%
%    \protect\numberline{\thesection}%
    #1}%
  }
\makeatother
\pagestyle{empty}




% ***********************************************************
% ********************** END HEADER *************************
% ***********************************************************

\usepackage[margin=1in]{geometry} % handle page geometry





\title{Phys 220A -- Extragalactic Astronomy -- Lec02}
\author{UCLA, Winter 2014}
\date{\formatdate{7}{1}{2014}} % Activate to display a given date or no date (if empty),
         % otherwise the current date is printed 

\begin{document}
\setlength{\unitlength}{1mm}
\maketitle

\section{Correction from last time}
Elmegreen Chapter 8. Looking in quadrants
towards the solar center, we have the radial velocity increasing until
we get to the subsolar point and then decreasing until it goes
negative outside of the solar circle. 
\begin{figure}
\centering
\includegraphics[width=.5\textwidth]{figures/radial.pdf}
\end{figure}

Another note: our estimations of the Oort constants will never be
perfect because we live in non-axisymmetric galaxy, have spiral arms
and a bar. 


\section{Rotation laws}
\subsection{Kepler}
Let's look at the orbit around a point mass M
\begin{equation}
v(R) = \left(\frac{GM}{R}\right)^{1/2}
\end{equation}
\begin{equation}
\omega = \frac{v}{r} = \sqrt{GM} R^{-3/2}
\end{equation}
\begin{equation}
\d{\omega}{R} = -\frac{3}{2} \sqrt{GM} R^{-5/2}
\end{equation}
\begin{equation}
A = - \frac{R}{2} \d{\omega}{R} = \frac{3}{4} \sqrt{GM} R^{-3/2}
\end{equation}
\begin{equation}
A = \frac{3}{4} \omega
\end{equation}
\begin{equation}
B = -\frac{1}{4}\omega
\end{equation}
This is not what we see around the Milky way, where A and B are roughly comparable. 

\subsection{Solid Body}
The dynamical timescale 
\begin{equation}
t_{dyn} \approx \frac{1}{\sqrt{G\rho}}
\end{equation}
\begin{equation}
\omega = const
\end{equation}
\begin{equation}
A = 0
\end{equation} 
\begin{equation}
B = -\omega
\end{equation}


\subsection{Flat Rotation}
\begin{equation}
V = constant
\end{equation}
\begin{equation}
\rho \propto r^{-2}
\end{equation}
\begin{equation}
A = \frac{R}{2} \frac{V_c}{R^2} = \frac{V_c}{2R} = \frac{\omega}{2}
\end{equation}
\begin{equation}
B = -\frac{\omega}{2}
\end{equation}

\section{Epicycles}
Describes orbits that are almost circles. The idea is that we have a guiding center and completes simple harmonic motion around it. We call the radial oscillation frequency $\kappa$. See Binney and Tremaine
\begin{equation}
\kappa = -4B(A-B)
\end{equation}
in our local neighborhood this is
\begin{equation}
\kappa = 37 km/s/Kpc
\end{equation}
Let's look at some real world cases. For Kepler
\begin{equation}
\kappa = \sqrt{4\frac{\omega}{4} \omega} = \omega
\end{equation}
Which is a way of saying that this is an ellipse. What about solid body rotation?
\begin{equation}
\kappa = \sqrt{4\omega^2} = 2\omega
\end{equation}
This is an ellipse that is centered around the center of the galaxy

For a flat rotation curve we have
\begin{equation}
\kappa = \sqrt{2} \omega
\end{equation}
So it will not close on itself. 


\section{Differential Rotation}
We get differential rotation because the angular frequency changes as a function of radius. We have a special case where
\begin{equation}
\omega - \omega_p = n\kappa
\end{equation}
Where n is an integer. If n is 1 then we have an ellipse like in the Keplerian orbit. In other words they come back to the same part as they start in an integer number of orbits. If we have something satisfying this conditions then we could have a kinematic density wave


\section{Toomre Parameter}
Disk stability parameter
\begin{equation}
Q =\frac{\sigma_R \kappa}{3.36 G\Sigma} > 1
\end{equation}
For stability
\end{document}
