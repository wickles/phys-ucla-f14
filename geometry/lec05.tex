% declare document class and geometry
\documentclass[12pt]{article} % use larger type; default would be 10pt
\usepackage[margin=1in]{geometry} % handle page geometry

% import packages and commands
% standard packages
\usepackage{graphicx} % support the \includegraphics command and options
\usepackage{amsmath} % for nice math commands and environments

% font packages
\usepackage{amssymb} % for \mathbb, \mathfrak fonts
\usepackage{mathrsfs} % for \mathscr font
\DeclareMathAlphabet{\mathpzc}{OT1}{pzc}{m}{it} % defines \mathpzc for Zapf Chancery (standard postscript) font

% other packages
\usepackage{datetime} % allows easy formatting of dates, e.g. \formatdate{dd}{mm}{yyyy}
\usepackage{caption} % makes figure captions better, more configurable
\usepackage[squaren]{SIunits} % for nice units formatting e.g. \unit{50}{\kilo\gram}
\usepackage{cancel} % for crossing out terms with \cancel
\usepackage{verbatim} % for verbatim and comment environments
\usepackage{tensor} % for \indices e.g. M\indices{^a_b^{cd}_e}, and \tensor e.g. \tensor[^a_b^c_d]{M}{^a_b^c_d}
\usepackage{feynmp-auto} % for Feynman diagrams. 
\usepackage{pgfplots} % for plotting in tikzpicture environment

% new commands
\newcommand{\fslash}[1]{#1\!\!\!/} % feynman slash
\newcommand{\opname}[1]{\operatorname{#1}} % custom operator names
\newcommand{\pd}{\partial} % partial differential shortcut
\newcommand{\ket}[1]{\left| #1 \right>} % for Dirac kets
\newcommand{\bra}[1]{\left< #1 \right|} % for Dirac bras
\newcommand{\braket}[2]{\left< #1 \vphantom{#2} \right| 
	\left. #2 \vphantom{#1} \right>} % for Dirac brackets

\let\vaccent=\v % rename builtin command \v{} to \vaccent{}
%\renewcommand{\v}[1]{\ensuremath{\mathbf{#1}}} % for vectors
\renewcommand{\v}[1]{\ensuremath{\boldsymbol{\mathbf{#1}}}} % for vectors
%\newcommand{\gv}[1]{\ensurmath{\mbox{\boldmath$ #1 $}}} % for vectors of Greek letters
\newcommand{\uv}[1]{\ensuremath{\boldsymbol{\mathbf{\widehat{#1}}}}} % for unit vectors
\newcommand{\abs}[1]{\left| #1 \right|} % for absolute value ||x||
\newcommand{\norm}[1]{\left\Vert #1 \right\Vert} % for norm ||v||
\newcommand{\avg}[1]{\left< #1 \right>} % for average <x>
\newcommand{\inner}[2]{\left< #1, #2 \right>} % for inner product <x,y>
\newcommand{\set}[1]{ \left\{ #1 \right\} } % for sets {a,b,c,...}

% shortcuts
\newcommand{\reals}{\mathbb{R}} % real numbers
\newcommand{\complexes}{\mathbb{C}} % complex numbers
\newcommand{\nats}{\mathbb{N}} % natural numbers
\newcommand{\irrats}{\mathbb{Q}} % irrationals
\newcommand{\quats}{\mathbb{H}} % quaternions (a la Hamilton)
\newcommand{\euclids}{\mathbb{E}} % Euclidean space
\newcommand{\bigo}{\mathcal{O}} % big O notation





%%%%%%%%%%%%%%%%%%%
% some templates for various things
\begin{comment}

% template for figures
\begin{figure}
\centering
\includegraphics{myfile.png}
\caption{This is a caption}
\label{fig:myfigure}
\end{figure}

% template for Feynman diagrams using feynmf/feynmp
\begin{fmfgraph*}(40,25)
\fmfleft{em,ep}
\fmf{fermion}{em,Zee,ep}
\fmf{photon,label=$Z$}{Zee,Zff}
\fmf{fermion}{fb,Zff,f}
\fmfright{fb,f}
\fmfdot{Zee,Zff}
\end{fmfgraph*}

% template for drawing plots with pgfplot
\pgfplotsset{compat=1.3,compat/path replacement=1.5.1}
\begin{tikzpicture}
\begin{axis}[
extra x ticks={-2,2},
extra y ticks={-2,2},
extra tick style={grid=major}]
\addplot {x};
\draw (axis cs:0,0) circle[radius=2];
\end{axis}
\end{tikzpicture}

\end{comment}
%%%%%%%%%%%%%%%%%%%



\title{Math 217 -- Geometry and Physics -- Lec05}
\author{UCLA, Fall 2014}
\date{\formatdate{13}{10}{2014}} % Activate to display a given date or no date (if empty),
         % otherwise the current date is printed 

\begin{document}
\maketitle


\section{More stuff}

\subsection{Remark on Thom isomorphisms from Algebraic Topology}

Given a vector bundle $E^{n+r} \overset{\pi}{\longrightarrow} M^n = \cup_\alpha U_\alpha$ where we map between $E \leftrightarrow \set{U_\alpha, g_{\alpha\beta}}$. Here $g_{\alpha\beta}$ is the map between the images of the functions
\begin{align}
\psi_\alpha &; \pi^{-1}(U_\alpha) \widetilde{\rightarrow} U_\alpha \times \reals^r, \quad \text{and} \\
\psi_\beta &; \pi^{-1}(U_\beta) \widetilde{\rightarrow} U_\beta \times \reals^r.
\end{align}
Then there is a map
\begin{equation}
\begin{matrix}
H^* (M) & \tilde{\longrightarrow} & H_{CU}^{*+r} (E) \\
[\omega] & \mapsto & \Phi(E) \wedge [\pi^* \omega]
\end{matrix}
\end{equation}
where $\Phi(E)$ is the Thom class of $E$. 

\textbf{Fact}: For any section $s : M \rightarrow E$ we have $s^* \Phi(E) \in H^r (M)$ is the Euler class $e(E)$ of $E$. 

This holds for any cohomology theory.

\subsection{Thom isomorphism gives a way to define the push-forward}

Consider a smooth map $f : X \rightarrow Y$. Then given the immersion $i : X \hookrightarrow \reals^N (\hookrightarrow S^N)$ we have the immersion $X \overset{(f,i)}{\hookrightarrow} Y \times S^N$. Then we have $r = \opname{rank} N = \opname{codim} X$ and we can define the push-forward in any (generalized) cohomology theory
\begin{equation}
H^*(X) \overset{\text{Thom}}{\longrightarrow} H_{CU}^{*+r}(N_X / (Y \times S^N)) \overset{\text{inclusion}}{\longrightarrow} H^{*+r}(Y \times S^N)
\end{equation}
and the first and last parts map to $H^*(Y)$ by $f_*$ and $\pi_*$ (integration along $S^N$) respectively. 

[talked more Riemann-Roch, Chern-Weil, Atiyah-Hirzebruch RR, Maurer-Cartan form]







\end{document}
