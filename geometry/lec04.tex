% declare document class and geometry
\documentclass[12pt]{article} % use larger type; default would be 10pt
\usepackage[margin=1in]{geometry} % handle page geometry

% import packages and commands
% standard packages
\usepackage{graphicx} % support the \includegraphics command and options
\usepackage{amsmath} % for nice math commands and environments

% font packages
\usepackage{amssymb} % for \mathbb, \mathfrak fonts
\usepackage{mathrsfs} % for \mathscr font
\DeclareMathAlphabet{\mathpzc}{OT1}{pzc}{m}{it} % defines \mathpzc for Zapf Chancery (standard postscript) font

% other packages
\usepackage{datetime} % allows easy formatting of dates, e.g. \formatdate{dd}{mm}{yyyy}
\usepackage{caption} % makes figure captions better, more configurable
\usepackage[squaren]{SIunits} % for nice units formatting e.g. \unit{50}{\kilo\gram}
\usepackage{cancel} % for crossing out terms with \cancel
\usepackage{verbatim} % for verbatim and comment environments
\usepackage{tensor} % for \indices e.g. M\indices{^a_b^{cd}_e}, and \tensor e.g. \tensor[^a_b^c_d]{M}{^a_b^c_d}
\usepackage{feynmp-auto} % for Feynman diagrams. 
\usepackage{pgfplots} % for plotting in tikzpicture environment

% new commands
\newcommand{\fslash}[1]{#1\!\!\!/} % feynman slash
\newcommand{\opname}[1]{\operatorname{#1}} % custom operator names
\newcommand{\pd}{\partial} % partial differential shortcut
\newcommand{\ket}[1]{\left| #1 \right>} % for Dirac kets
\newcommand{\bra}[1]{\left< #1 \right|} % for Dirac bras
\newcommand{\braket}[2]{\left< #1 \vphantom{#2} \right| 
	\left. #2 \vphantom{#1} \right>} % for Dirac brackets

\let\vaccent=\v % rename builtin command \v{} to \vaccent{}
%\renewcommand{\v}[1]{\ensuremath{\mathbf{#1}}} % for vectors
\renewcommand{\v}[1]{\ensuremath{\boldsymbol{\mathbf{#1}}}} % for vectors
%\newcommand{\gv}[1]{\ensurmath{\mbox{\boldmath$ #1 $}}} % for vectors of Greek letters
\newcommand{\uv}[1]{\ensuremath{\boldsymbol{\mathbf{\widehat{#1}}}}} % for unit vectors
\newcommand{\abs}[1]{\left| #1 \right|} % for absolute value ||x||
\newcommand{\norm}[1]{\left\Vert #1 \right\Vert} % for norm ||v||
\newcommand{\avg}[1]{\left< #1 \right>} % for average <x>
\newcommand{\inner}[2]{\left< #1, #2 \right>} % for inner product <x,y>
\newcommand{\set}[1]{ \left\{ #1 \right\} } % for sets {a,b,c,...}

% shortcuts
\newcommand{\reals}{\mathbb{R}} % real numbers
\newcommand{\complexes}{\mathbb{C}} % complex numbers
\newcommand{\nats}{\mathbb{N}} % natural numbers
\newcommand{\irrats}{\mathbb{Q}} % irrationals
\newcommand{\quats}{\mathbb{H}} % quaternions (a la Hamilton)
\newcommand{\euclids}{\mathbb{E}} % Euclidean space
\newcommand{\bigo}{\mathcal{O}} % big O notation





%%%%%%%%%%%%%%%%%%%
% some templates for various things
\begin{comment}

% template for figures
\begin{figure}
\centering
\includegraphics{myfile.png}
\caption{This is a caption}
\label{fig:myfigure}
\end{figure}

% template for Feynman diagrams using feynmf/feynmp
\begin{fmfgraph*}(40,25)
\fmfleft{em,ep}
\fmf{fermion}{em,Zee,ep}
\fmf{photon,label=$Z$}{Zee,Zff}
\fmf{fermion}{fb,Zff,f}
\fmfright{fb,f}
\fmfdot{Zee,Zff}
\end{fmfgraph*}

% template for drawing plots with pgfplot
\pgfplotsset{compat=1.3,compat/path replacement=1.5.1}
\begin{tikzpicture}
\begin{axis}[
extra x ticks={-2,2},
extra y ticks={-2,2},
extra tick style={grid=major}]
\addplot {x};
\draw (axis cs:0,0) circle[radius=2];
\end{axis}
\end{tikzpicture}

\end{comment}
%%%%%%%%%%%%%%%%%%%



\title{Math 217 -- Geometry and Physics -- Lec04}
\author{UCLA, Fall 2014}
\date{\formatdate{10}{10}{2014}} % Activate to display a given date or no date (if empty),
         % otherwise the current date is printed 

\begin{document}
\maketitle


\section{Stuff}

Recall again the Gauss-Bonnet-Chern theorem: for $M^{2n}$ a compact closed oriented manifold, we have
\begin{equation}
\int_M \varepsilon(TM) = \chi(M) = \sum_{i=0}^{2n} (-1)^i b_i.
\end{equation}

Then we have
\begin{enumerate}
\item $H^k(M) \cong H^{2n-k}(M)$
\item Thom isomorphism: Let $\Delta \subseteq M$ be a submanifold and 
\begin{equation}
N_\Delta \cong (TM|_\Delta) / T\Delta
\end{equation}
the normal bundle of $\Delta$ in $M$. Let $T(N_\Delta)$ be the Thom class of $N_\Delta$. Then the Poincare dual $\eta_\Delta$ of $\Delta$ is equal to $T(N_\Delta) \in H_c^*(N_\Delta)$ (Bott-Tu). That is, given the projection $i_{\Delta*} : N_\Delta \hookrightarrow M$ we have $i_{\Delta*} T(N_\Delta) = \eta_\Delta$. 
\end{enumerate}

Given a basis $\set{ \omega_i }$ of $H^*(M) = \oplus_{i=0}^{2n} H^i (M)$, define a dual basis $\set{ \tau_i }$ of $H^*(M)^*$ through the Poincare duality $H^k \cong (H^{2n-k})^*$ such that $\int_M \omega_i \wedge \tau_j = \delta_{ij}$. 

\textbf{Kuenneth Theorem} states that 
\begin{equation}
\begin{matrix}
H^*(M \times M) & \cong & H^*(M) & \otimes_\reals & H^*(M) \\
\set{ \pi^* \omega_i \wedge \rho^* \tau_j } & & \set{ \omega_i } & & \set{ \tau_j }
\end{matrix}
\end{equation}

Denote $i : \Delta \hookrightarrow M \times M$, $\Delta$ the diagonal $\set{ x,x } \in M \times M$ isomorphic to $M$. 

Let $\eta_\Delta \in H^*(M \times M)$ be the Poincare dual of $\Delta$. Then
\begin{equation}
\eta_\Delta = \sum_{ij} c_{ij} \pi^* \omega_i \wedge \rho^* \tau_j
\end{equation}

\textbf{Lemma}: 
\begin{equation}
\eta_\Delta = \sum_i (-1)^{\opname{deg} \omega_i} \pi^* \omega_i \wedge \rho^* \tau_i
\end{equation}

\textbf{Proof}: We have $\pi \circ i = \opname{id}, \rho \circ i = \opname{id} \implies i^* \rho^*, i^* \pi^*$ are identity on cohomolgy. Then 
\begin{align}
\int_\Delta i^* (\pi^* \tau_k \wedge \rho^* \omega_l) &= \int_M i^* (\pi^* \tau_k \wedge \rho^* \omega_l) \\
	&= \int_M (i^* \pi^*) \tau_k \wedge (i^* \rho^*) \omega_l \\
	&= \int_M \tau_k \wedge \omega_l \\
	&= \delta_{kl} (-1)^{\opname{deg} \tau_k \opname{deg} \omega_l}.
\end{align}
Furthermore
\begin{align}
\int_\Delta \pi^* \tau_k \wedge \rho^* \omega_l &= \int_{M \times M} (\pi^* \tau_k \wedge \rho^* \omega_l) \wedge \eta_\Delta \\
	&= \sum_{ij} c_{ij} \int_{M \times M} (\pi^* \tau_k \wedge \rho^* \omega_l) \wedge (\pi^* \omega_i \wedge \rho^* \tau_j) \\
	&= \sum_{ij} c_{ij} (-1)^{(\opname{deg} \tau_k + \opname{deg} \omega_l) \opname{deg} \omega_i} \int_{M \times M} \pi^* (\omega_i \wedge \tau_k) \wedge \rho^* (\omega_l \wedge \tau_j) \\
	&= \sum_{ij} c_{ij} (-1)^{(\opname{deg} \tau_k + \opname{deg} \omega_l) \opname{deg} \omega_i} (\int_M \omega_i \wedge \tau_k) (\int_M \omega_l \wedge \tau_j) \\
	&= \sum_{ij} c_{ij} (-1)^{(\opname{deg} \tau_k + \opname{deg} \omega_l) \opname{deg} \omega_i} \delta_{ik} \delta{lj} \\
	&= c_{kl} (-1)^{(\opname{deg} \tau_k + \opname{deg} \omega_l) \opname{deg} \omega_k}
\end{align}
where
\begin{equation}
c_{kl} = 
\begin{cases}
0 & k \neq l, \\
(-1)^{\opname{deg} \omega_k} & k = l.
\end{cases}
\end{equation}

\textbf{Lemma}: Let $T_\Delta$ be the tangent bundle of $\Delta \hookrightarrow M \times M$. Then $T_\Delta \cong N_\Delta$. 

\textbf{Proof}: 
\begin{equation}
\begin{matrix}
0 & \rightarrow & T_\Delta & \rightarrow & T_{M \times M} |_\Delta & \rightarrow & N_\Delta & \rightarrow 0 \\
& & \cong & & \cong & & \cong & & \\
0 & \rightarrow & T_M & \rightarrow & TM \oplus TM & \rightarrow & T_M & \rightarrow 0 
\end{matrix}
\end{equation}
We have
\begin{equation}
\int_\Delta i_\Delta^* \eta_\Delta = \int_\Delta i_\Delta^* \Phi(N_\Delta) = \int_\Delta \varepsilon(N_\Delta) = \int_M \varepsilon(TM)
\end{equation}
where $s : M \rightarrow E$ is a section and $s^* \Phi(E) = \varepsilon(E)$. We have 
\begin{align}
\int_\Delta i_\Delta^* \eta_\Delta &= \sum_i (-1)^{\opname{deg} \omega_i} \int_\Delta i_\Delta^* (\pi^* \omega_i \wedge \rho^* \tau_i) \\
	&= \sum_i (-1)^{\opname{deg} \omega_i} \int_M (i^* \pi^* ) \omega_i \wedge (i^* \rho^* \tau_i) \\
	&= \sum_i (-1)^{\opname{deg} \omega_i} \int_M \underbrace{\omega_i \wedge \tau_i}_{= 1} \\
	&= \sum_{i=0}^{2n} (-1)^i \opname{dim} H^i(M) = \chi(M).
\end{align}


\section{More on Lie Algebras}

Let $g$ be a Lie algebra of a Lie group $G$. We have $g = T_e G$ where $e \in G$ is the identity, and this is also seen as the space of left invariant vector fields. We have the commutator taking $g \times g \rightarrow g$ defined by $X,Y \mapsto [X,Y]$. We have the identities
\begin{enumerate}
\item $[X,Y] = -[Y, X]$, and
\item $[[X,Y],Z] + [[Y,Z], X] + [[Z,X],Y] = 0$, the Jacobi identity.
\end{enumerate}

For example the Lie group $SO_n$ has the Lie algebra $\mathfrak{so}_n$. We have $\mathfrak{so}_n \cong \Lambda^2 \reals^n$. So $\set{ e_1, \dots, e_n }$ is an orthonormal basis w.r.t. the inner product $\inner{\cdot}{\cdot}$. And we have
\begin{equation}
v \wedge w (x) = \inner{v}{x} w - \inner{w}{x} v \quad \forall x \in \reals^n
\end{equation}

There is a one to one correspondence between one parameter subgroups of $G$ and orbits of elements of $g$ passing through $e$.
\begin{equation}
\psi : \reals \rightarrow G	\quad	\leftrightarrow	\quad	\psi'(0) = X \in G
\end{equation}
Geodesics of $G$ with Killing metric on $G$.







\end{document}
