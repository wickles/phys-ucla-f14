% declare document class and geometry
\documentclass[12pt]{article} % use larger type; default would be 10pt
\usepackage[margin=1in]{geometry} % handle page geometry

% standard packages
\usepackage{graphicx} % support the \includegraphics command and options
\usepackage{amsmath} % for nice math commands and environments
\usepackage{mathtools} % extends amsmath with bug fixes and useful commands e.g. \shortintertext
\usepackage{amsthm} % for theorem and proof environments

% font packages
\usepackage{amssymb} % for \mathbb, \mathfrak fonts
\usepackage{mathrsfs} % for \mathscr font
\DeclareMathAlphabet{\mathpzc}{OT1}{pzc}{m}{it} % defines \mathpzc for Zapf Chancery (standard postscript) font

% other packages
\usepackage{datetime} % allows easy formatting of dates, e.g. \formatdate{dd}{mm}{yyyy}
\usepackage{caption} % makes figure captions better, more configurable
\usepackage{enumitem} % allows for custom labels on enumerated lists, e.g. \begin{enumerate}[label=\textbf{(\alph*)}]
\usepackage[squaren]{SIunits} % for nice units formatting e.g. \unit{50}{\kilo\gram}
\usepackage{cancel} % for crossing out terms with \cancel
\usepackage{verbatim} % for verbatim and comment environments
\usepackage{tensor} % for \indices e.g. M\indices{^a_b^{cd}_e}, and \tensor e.g. \tensor[^a_b^c_d]{M}{^a_b^c_d}
\usepackage{feynmp-auto} % for Feynman diagrams. 
\usepackage{pgfplots} % for plotting in tikzpicture environment
\usepackage{commath} % for some nice standardized syntax stuff. \dif, \Dif \od, \pd, \md, \(abs | envert), \(norm | enVert), \(set | cbr), \sbr, \eval, \int(o | c)(o | c), etc
\usepackage{slashed} % provides a command \slashed[1] for Feynman slash notation
%\newcommand{\fslash}[1]{#1\!\!\!/} % feynman slash
%\newcommand{\fsl}[1]{\ensuremath{\mathrlap{\!\not{\phantom{#1}}}#1}}% \fsl{<symbol>}
	% alternative feynman slash

% new commands
\newcommand{\beg}{\begin} % a few letters less for beginning environments
\newenvironment{eqn}{\begin{equation}}{\end{equation}} % a lot fewer letter for equation environment

% rotate stuff
\usepackage{rotating}
	% provides environments for rotating arbitrary objects, e.g. sideways, turn[ang], rotate[ang]
	% also provides macro \turnbox{ang}{stuff}
%\newcommand{\sideways}[1]{\begin{sideways} #1 \end{sideways}} % turn things 90 degrees CCW
%\newcommand{\turn}[2][]{\begin{turn}{#2} #1 \end{turn}} % \turn[ang]{stuff} turns things arbitrary +/- angle

% notational commands
\newcommand{\opname}[1]{\operatorname{#1}} % custom operator names
%\newcommand{\pd}{\partial} % partial differential shortcut
\newcommand{\ket}[1]{\left| #1 \right>} % for Dirac kets
%\newcommand{\ket}[1]{| #1 \rangle}
\newcommand{\bra}[1]{\left< #1 \right|} % for Dirac bras
%\newcommand{\bra}[1]{\langle #1 |}
\newcommand{\braket}[2]{\left< #1 \vphantom{#2} \right| \left. #2 \vphantom{#1} \right>} 
	% for Dirac bra-kets \braket{bra}{ket}
%\newcommand{\braket}[2]{\langle #1 | #2 \rangle} 
\newcommand{\matrixel}[3]{\left< #1 \vphantom{#2#3} \right| #2 \left| #3 \vphantom{#1#2} \right>} 
	% for Dirac matrix elements \matrixel{bra}{op}{ket}
%\newcommand{\matrixel}[3]{\langle #1 | #2 | #3 \rangle} 

%\newcommand{\pd}[2]{\frac{\partial #1}{\partial #2}} % for partial derivatives
%\newcommand{\fd}[2]{\frac{\delta #1}{\delta #2}} % for functional derivatives
\let \vaccent = \v % rename builtin command \v{} to \vaccent{}
%\renewcommand{\v}[1]{\ensuremath{\mathbf{#1}}} % for vectors
\renewcommand{\v}[1]{\ensuremath{\boldsymbol{\mathbf{#1}}}} % for vectors
%\newcommand{\gv}[1]{\ensurmath{\mbox{\boldmath$ #1 $}}} % for vectors of Greek letters
\newcommand{\uv}[1]{\ensuremath{\boldsymbol{\mathbf{\widehat{#1}}}}} % for unit vectors
%\newcommand{\abs}[1]{\left| #1 \right|} % for absolute value ||x||
%\newcommand{\mag}{\abs} % magnitude, just another name for \abs
%\newcommand{\norm}[1]{\left\Vert #1 \right\Vert} % for norm ||v||
\newcommand{\vd}[1]{\v{\dot{#1}}} % for dotted vectors
\newcommand{\vdd}[1]{\v{\ddot{#1}}} % for ddotted vectors
\newcommand{\vddd}[1]{\v{\dddot{#1}}} % for dddotted vectors
\newcommand{\vdddd}[1]{\v{\ddddot{#1}}} % for ddddotted vectors
\newcommand{\avg}[1]{\left< #1 \right>} % for average <x>
\newcommand{\inner}[2]{\left< #1, #2 \right>} % for inner product <x,y>
%\newcommand{\set}[1]{ \left\{ #1 \right\} } % for sets {a,b,c,...}
\newcommand{\tr}{\opname{tr}} % for trace
\newcommand{\Tr}{\opname{Tr}} % for Trace
\newcommand{\rank}{\opname{rank}} % for rank
\let \fancyre = \Re
\let \fancyim = \Im
\newcommand{\Res}{\opname{Res}\limits} % for residue function -- change to put limits on bottom
\renewcommand{\Re}{\opname{Re}}
\renewcommand{\Im}{\opname{Im}}
\renewcommand{\bbar}[1]{\bar{\bar{#1}}} 
	% for barring things twice -- use \cbar or \zbar instead of original \bbar
\newcommand{\bbbar}[1]{\bar{\bbar{#1}}}
\newcommand{\bbbbar}[1]{\bar{\bbbar{#1}}}

\newcommand{\inv}{^{-1}}

% temporary fixes -- commath's versions are bad for powers, like $\dif^3 x$
\renewcommand{\dif}{\mathrm{d}} % \opname{d} better maybe?
\renewcommand{\Dif}{\mathrm{D}}

% notational shortcuts
\newcommand{\bigO}{\mathcal{O}} % big O notation
\let \bigo = \bigO % keep for now, need to update instances in older files
\newcommand{\Lag}{\mathcal{L}} % fancy Lagrangian
\newcommand{\Ham}{\mathcal{H}} % fancy Hamiltonian
\newcommand{\reals}{\mathbb{R}} % real numbers
\newcommand{\complexes}{\mathbb{C}} % complex numbers
\newcommand{\ints}{\mathbb{Z}} % integers
\newcommand{\nats}{\mathbb{N}} % natural numbers
\newcommand{\irrats}{\mathbb{Q}} % irrationals
\newcommand{\quats}{\mathbb{H}} % quaternions (a la Hamilton)
\newcommand{\euclids}{\mathbb{E}} % Euclidean space
\newcommand{\R}{\reals}
\newcommand{\C}{\complexes}
\newcommand{\Z}{\ints}
\newcommand{\Q}{\irrats}
\newcommand{\N}{\nats}
\newcommand{\E}{\euclids}
\newcommand{\RP}{\mathbb{RP}} % real projective space
\newcommand{\CP}{\mathbb{CP}} % complex projective space

% matrix shortcuts!
\newcommand{\pmat}[1]{\begin{pmatrix} #1 \end{pmatrix}}
\newcommand{\bmat}[1]{\begin{bmatrix} #1 \end{bmatrix}}
\newcommand{\Bmat}[1]{\begin{Bmatrix} #1 \end{Bmatrix}}
\newcommand{\vmat}[1]{\begin{vmatrix} #1 \end{vmatrix}}
\newcommand{\Vmat}[1]{\begin{Vmatrix} #1 \end{Vmatrix}}


% more stuff
\newenvironment{enumproblem}{\begin{enumerate}[label=\textbf{(\alph*)}]}{\end{enumerate}}
	% for easily enumerating letters in problems
\newcommand{\grad}[1]{\v{\nabla} #1} % for gradient
\let \divsymb = \div % rename builtin command \div to \divsymb
\renewcommand{\div}[1]{\v{\nabla} \cdot #1} % for divergence
\newcommand{\curl}[1]{\v{\nabla} \times #1} % for curl
\let \baraccent = \= % rename builtin command \= to \baraccent
\renewcommand{\=}[1]{\stackrel{#1}{=}} % for putting numbers above =


% theorem-style environments. note amsthm builtin proof environment: \begin{proof}[title]
% appending [section] resets counter and prepends section number
% use \setcounter{counter}{0} to reset counter
% typical use cases:
% plain: Theorem, Lemma, Corollary, Proposition, Conjecture, Criterion, Algorithm
% definition: Definition, Condition, Problem, Example
% remark: Remark, Note, Notation, Claim, Summary, Acknowledgment, Case, Conclusion
\theoremstyle{plain} % default
\newtheorem{theorem}{Theorem}[section]
\newtheorem{lemma}[theorem]{Lemma}
\newtheorem{corollary}[theorem]{Corollary}
\newtheorem{proposition}[theorem]{Proposition}
\newtheorem{conjecture}[theorem]{Conjecture}
% definition style
\theoremstyle{definition}
\newtheorem{definition}{Definition}
\newtheorem{problem}{Problem}
\newtheorem{exercise}{Exercise}
\newtheorem{example}{Example}
% remark style
\theoremstyle{remark}
\newtheorem{remark}{Remark}
\newtheorem{note}{Note}
\newtheorem{claim}{Claim}
\newtheorem{conclusion}{Conclusion}
% to-do: add problem/subproblem/answer environments for homeworks









%%%%% derivatives


\let \underdot = \d % rename builtin command \d{} to \underdot{}
\let \d = \od % for derivatives

% BUG: derivatives revert to text mode often when in smaller environments in math mode?


% Command for functional derivatives. The first argument denotes the function and the second argument denotes the variable with respect to which the derivative is taken. The optional argument denotes the order of differentiation. The style (text style/display style) is determined automatically
\providecommand{\fd}[3][]{\ensuremath{
\ifinner
\tfrac{\delta{^{#1}}#2}{\delta{#3^{#1}}}
\else
\dfrac{\delta{^{#1}}#2}{\delta{#3^{#1}}}
\fi
}}

% \tfd[2]{f}{k} denotes the second functional derivative of f with respect to k
% The first letter t means "text style"
\providecommand{\tfd}[3][]{\ensuremath{\mathinner{
\tfrac{\delta{^{#1}}#2}{\delta{#3^{#1}}}
}}}
% \dfd[2]{f}{k} denotes the second functional derivative of f with respect to k
% The first letter d means "display style"
\providecommand{\dfd}[3][]{\ensuremath{\mathinner{
\dfrac{\delta{^{#1}}#2}{\delta{#3^{#1}}}
}}}

% mixed functional derivative - analogous to the functional derivative command
% \mfd{F}{5}{x}{2}{y}{3}
\providecommand{\mfd}[6]{\ensuremath{
\ifinner
\tfrac{\delta{^{#2}}#1}{\delta{#3^{#4}}\delta{#5^{#6}}}
\else
\dfrac{\delta{^{#2}}#1}{\delta{#3^{#4}}\delta{#5^{#6}}}
\fi
}}


% Command for thermodynamic (chemistry?) partial derivatives. The first argument denotes the function and the second argument denotes the variable with respect to which the derivative is taken. The optional argument denotes the order of differentiation. The style (text style/display style) is determined automatically
\providecommand{\pdc}[4][]{\ensuremath{
\ifinner
\left( \tfrac{\partial{^{#1}}#2}{\partial{#3^{#1}}} \right)_{#4}
\else
\left( \dfrac{\partial{^{#1}}#2}{\partial{#3^{#1}}} \right)_{#4}
\fi
}}

% \tpd[2]{f}{k} denotes the second thermo partial derivative of f with respect to k
% The first letter t means "text style"
\providecommand{\tpdc}[4][]{\ensuremath{\mathinner{
\left( \tfrac{\partial{^{#1}}#2}{\partial{#3^{#1}}} \right)_{#4}
}}}
% \dpd[2]{f}{k} denotes the second thermo partial derivative of f with respect to k
% The first letter d means "display style"
\providecommand{\dpdc}[4][]{\ensuremath{\mathinner{
\left( \dfrac{\partial{^{#1}}#2}{\partial{#3^{#1}}} \right)_{#4}
}}}


%%%%%%





%%%%%%%%%%%%%%%%%%%
% some templates for various things
\begin{comment}

% template for figures
\begin{figure}
\centering
\includegraphics{myfile.png}
\caption{This is a caption}
\label{fig:myfigure}
\end{figure}

% template for Feynman diagrams using feynmf/feynmp
\begin{fmfgraph*}(40,25)
\fmfleft{em,ep}
\fmf{fermion}{em,Zee,ep}
\fmf{photon,label=$Z$}{Zee,Zff}
\fmf{fermion}{fb,Zff,f}
\fmfright{fb,f}
\fmfdot{Zee,Zff}
\end{fmfgraph*}

% template for drawing plots with pgfplot
\pgfplotsset{compat=1.3,compat/path replacement=1.5.1}
\begin{tikzpicture}
\begin{axis}[
extra x ticks={-2,2},
extra y ticks={-2,2},
extra tick style={grid=major}]
\addplot {x};
\draw (axis cs:0,0) circle[radius=2];
\end{axis}
\end{tikzpicture}

%% find package for easily drawing mapping / algebraic / commutative diagrams..

\end{comment}
%%%%%%%%%%%%%%%%%%%



%%%%% A note on spacing
% 5) \qquad
% 4) \quad
% 3) \thickspace = \;
% 2) \medspace = \:
% 1) \thinspace = \,
% -1) \negthinspace = \!
% -2) \negmedspace
% -3) \negthickspace




\title{Phys 220A -- Classical Mechanics -- Review}
\author{UCLA, Fall 2014}
\date{\formatdate{10}{12}{2014}} % Activate to display a given date or no date (if empty),
         % otherwise the current date is printed 

\begin{document}
\setlength{\unitlength}{1mm}
\maketitle


\section{Topics}

The main topics to know well for the final are the following:
\begin{itemize}
\item Newtonian mechanics
\item Lagrangian (+ small oscillations)
\item Hamiltonian
\item Rigid bodies
\item Relativity
\end{itemize}


\section{Newtonian mechanics}

Key points and subjects:

\begin{itemize}
\item Free particle
\item Particle in 1 dimension in a potential
\item Harmonic oscillator (various dimensions)
\item Central potential / Kepler problem
\item Charged particle in EM field
\end{itemize}


\section{Lagrangian mechanics}

\begin{itemize}
\item Action: $S = \int \dif t \, L(q, \dot q, t)$
\item Hamilton's principle $\delta S = 0$ with boundary conditions $\delta q_1 = \delta q_2 = 0$ give Euler-Lagrange equations
\begin{eqn}
0 = \pd{L}{q_i} - \od{}{t} \pd{L}{\dot q_i}.
\end{eqn}
\item Also for fields, have Lagrangian density $\Lag(\phi, \dot \phi, \partial_x \phi)$, action $S = \int \dif t \, \dif x \, \Lag$ and EL equations
\begin{eqn}
0 = \pd{\Lag}{\phi} - \pd{}{t} \pd{\Lag}{\dot \phi} - \pd{}{x} \pd{\Lag}{(\partial_x \phi)}.
\end{eqn}
\end{itemize}

Advantages of the Lagrangian formalism:
\begin{itemize}
\item The equations take the same form under coordinate changes in the Lagrangian
\item Enables us to solve systems with constraints $\Phi_I(q, \dot q, t) = 0$. 
\begin{itemize}
	\item Holonomic constraints (no $\dot q$ dependence) make it easier: $\phi(q,t) = 0$
	\item Sometimes non-holonomic constraints are integrable, i.e. $\tod{\phi}{t} = 0$ which means we can integrate to find new constraint $\widetilde{\phi}(q,t) = \text{const}$ which is holonomic.
	\item Lagrange multipliers:
	\begin{eqn}
	L \to L + \sum_I \lambda_I \Phi_I (q,t)
	\end{eqn}
	this gives us new EL equations
	\begin{eqn}
	0 = \od{}{t} \pd{L}{\dot q_i} - \pd{L}{q_i} - \sum_I \lambda_I \pd{\Phi_I}{q}
	\end{eqn}
	which enforce the constraint. The new term in the EL equations are basically a force term (gradient of contstraint surface) which enforce the constraint. 
	\item Sometimes can find new generalized coordinates $\widetilde{q}_j$, $j = 1,\dots, N-s$ wihich solve the constraint, e.g.
	\begin{eqn}
	0 = \od{}{t} \pd{L}{\dot{\widetilde{q}}_j} - \pd{L}{\widetilde{q}_j}.
	\end{eqn}
\end{itemize}
\end{itemize}

Noether's theorem and symmetries

\begin{itemize}
\item Under a variation $q_i \to q_i' = q_i + \epsilon \delta q_i$, Lagrangian is invariant up to a total time derivative
\begin{eqn}
blah
\end{eqn}
\item Conserved charge and simple example and stuff, but argh he erased the board after writing it
\end{itemize}


\section{Small oscillations}

Generally have a Lagrangian of the form
\begin{eqn}
L = \frac{1}{2} \sum_{ij} M_{ij} (q) \dot q_i \dot q_j + \sum_i N_i (q) \dot q_i - V(q).
\end{eqn}
Next we find the equilibrium points $\bar q_i$ where $\dot{\bar q}_i = \pd{V}{\bar q_i} = 0$, take the difference from equilibrium $\delta q_i = q_i - \bar q_i$, to find a linearized equation of motion
\begin{eqn}
0 = \sum_j \left[ \widetilde{M}_{ij} (\bar q) \delta \ddot q_j + \widetilde{N}_{ij} (\bar q) \delta \dot q_j + \widetilde{V}_{ij} (\bar q) \delta q_j \right].
\end{eqn}
Next find the normal modes
\begin{eqn}
\delta q_i = v_i e^{i \omega t}
\end{eqn}
satisfying
\begin{eqn}
0 = (-\omega^2 \widetilde{M} + i \omega \widetilde{N} + \widetilde{V}) \cdot v.
\end{eqn}
Frequencies are determined by
\begin{eqn}
0 = \det(-\omega^2 \widetilde{M} + i \omega \widetilde{N} + \widetilde{V}),
\end{eqn}
and once we've found frequencies we can find the normal modes $v$. 


\section{Hamiltonian mechanics}

We take the conjugate momentum $p_i = \pd{L}{\dot q_i}$ and form the Hamiltonian via Legendre transform of the Lagrangian,
\begin{eqn}
H = p_i \dot q_i - L(d, \dot q).
\end{eqn}
Then we have Hamilton's equations in terms of $q$ and $p$,
\begin{eqn}
\dot p_i = - \pd{H}{q_i}, \qquad
\dot q_i = \pd{H}{p_i}.
\end{eqn}
Similarly for fields $\phi(x,t)$, we have the canonically conjugate momentum $\Pi_\phi = \pd{\Lag}{\dot \phi}$ and Hamiltonian density
\begin{eqn}
\Ham = \dot \phi \Pi_\phi - \Lag, \qquad
H = \int \dif x \, \Ham.
\end{eqn}

In classical mechanics, and especially Hamiltonian mechanics, we generally want to use Poisson brackets
\begin{eqn}
\cbr{A,B} = \sum_i \left[ \pd{A}{q_i} \pd{B}{p_i} - \pd{A}{p_i} \pd{B}{q_i} \right].
\end{eqn}
The Poisson brackets package the time derivative in a nice form,
\begin{eqn}
\od{A}{t} = \pd{A}{t} + \cbr{A,H}.
\end{eqn}
Note that the conserved charge $Q(p,q)$ from Noether's theorem is conserved so $\cbr{Q,H} = 0$. If we have conserved quantities $Q_1, Q_2$, then the bracket $\cbr{Q_1,Q_2}$ is also conserved. 

A \emph{canonical transformation} 
\begin{eqn}
Q = Q(q,p), \qquad
P = P(q,p)
\end{eqn}
is a coordinate transformation is one which obeys Hamilton's equations in terms of $Q, P$ as canonical coordinates. Simplest characterization is that the Poisson brackets are conserved,
\begin{eqn}
\cbr{Q_i, Q_j}_{qp} = 0, \quad
\cbr{P_i, P_j}_{qp} = 0, \quad
\cbr{Q_i, P_j}_{qp} = \delta_{ij}.
\end{eqn}

\emph{Generating functions} give us a way of constructing canonical transformations. There are four types,
\begin{eqn}
F_1 (q,Q), \quad
F_2 (q,P), \quad
F_3 (p,Q), \quad
F_4 (p,P),
\end{eqn}
and can also depend explicitly on time. Most important are types one (Hamilton-Jacobi) and two (infinitesimal canonical transformations). For type 1, we have
\begin{eqn}
p_i = \pd{F_1}{q_i}, \qquad
P_i = - \pd{F_1}{Q_i}, \qquad
K(Q,P) = H(q,p) + \pd{F_1}{t}.
\end{eqn}
Similarly for type 2 we have
\begin{eqn}
p_i = \pd{F_2}{q_i}, \qquad
Q_i = \pd{F_2}{P_i}, \qquad
K = H + \pd{F_2}{t}.
\end{eqn}

\emph{Action-angle variables} are nice because the motion is basically a circle in phase space. These are coordinates obtained by canonical transformation
\begin{eqn}
(q_i, p_i) \to (\theta_i, I_i)
\end{eqn}
where the momenta $I_i$ are conserved, $\dot I_i = 0$, and the coordiantes $\theta_i$ are linear in time,
\begin{eqn}
\dot \theta_i = \pd{H}{I_i} = \text{const} 
\qquad \implies \qquad
\theta_i = \omega_i t + \theta_0. 
\end{eqn}

Hamilton-Jacobi formalism gives us PDE for a generating function that makes the transformed Hamitlonian trivial, $K = 0$. From above, the expression for the new Hamiltonian and the old momentum give us the HJE
\begin{eqn}
0 = H(\tpd{F_1}{q_i}, q_i) + \pd{F_1}{t},
\end{eqn}
where the new coordinates and momenta are all constants of motion, $\dot Q_i = \dot P_i = 0$ so $Q_i = \alpha_i$, $P_i = \beta_i$. The HJE is not always readily soluble, but if $H$ is not time-dependent then we can make a separation ansatz, 
\begin{eqn}
F = \widetilde{F}(q,Q) - \alpha t. 
\end{eqn}
Sometimes we can further separate $\widetilde{F}$ out into sums of functions of subsets of the coordinates, which simplifies the problem. 


\section{Rigid bodies}

Generally in a rigid body problem we can decompose points in the lab frame as
\begin{eqn}
\v x_\mathrm{lab} = \v x_\mathrm{CM} + R \cdot \v x_\mathrm{body}
\end{eqn}
where $\v x_\mathrm{body}$ is a constant vector in the body frame. In terms of the angular velocity $\v \omega$ we find the lab velocity
\begin{eqn}
\vd x_\mathrm{lab} = \vd x_\mathrm{CM} + R \cdot (\v \omega\times \v x_\mathrm{body}).
\end{eqn}
Note that all quantities here are dependent on time (except $\v x_\mathrm{body}$). So we find for the whole rigid body, the kinetic energy
\begin{eqn}
T = \frac{1}{2} M_\mathrm{tot} \vd x_\mathrm{CM}^2 + \frac{1}{2} \v \omega^\top \cdot I \cdot \v \omega
\end{eqn}
and angular momentum
\begin{eqn}
\v L_\mathrm{lab} = R (I \cdot \v \omega),
\end{eqn}
where $I$ is the moment of inertia tensor
\begin{eqn}
I_{ij} = \sum_{(k)} m_{(k)} \left[ \delta_{ij} \v x_{(k)}^2 - x_{(k)i} x_{(k)j} \right].
\end{eqn}
Generally we can diagonalize the tensor into a form
\begin{eqn}
I = \pmat{I_1 & & \\ & I_2 & \\ & & I_3},
\end{eqn}
so that we find Euler'es equations from $\od{}{t} \v L_\mathrm{lab} = 0$,
\begin{eqn}
0 = I \cdot \vd \omega + \omega \times (I \cdot \v \omega).
\end{eqn}
Also Euler angles and stuff.


\section{Relativity}

Know about boosts and Lorentz transformations. Useful to go to CM frame in collisions. Charged particle in EM field. Kinematics with $p^\mu$, conservation of momentum and energy. 




\end{document}
