% declare document class and geometry
\documentclass[12pt]{article} % use larger type; default would be 10pt
\usepackage[margin=1in]{geometry} % handle page geometry

\input{../header2.tex}

\title{Phys 220A -- Classical Mechanics -- Review}
\author{UCLA, Fall 2014}
\date{\formatdate{10}{12}{2014}} % Activate to display a given date or no date (if empty),
         % otherwise the current date is printed 

\begin{document}
\setlength{\unitlength}{1mm}
\maketitle


\section{Topics}

The main topics to know well for the final are the following:
\begin{itemize}
\item Newtonian mechanics
\item Lagrangian (+ small oscillations)
\item Hamiltonian
\item Rigid bodies
\item Relativity
\end{itemize}


\section{Newtonian mechanics}

Key points and subjects:

\begin{itemize}
\item Free particle
\item Particle in 1 dimension in a potential
\item Harmonic oscillator (various dimensions)
\item Central potential / Kepler problem
\item Charged particle in EM field
\end{itemize}


\section{Lagrangian mechanics}

\begin{itemize}
\item Action: $S = \int \dif t \, L(q, \dot q, t)$
\item Hamilton's principle $\delta S = 0$ with boundary conditions $\delta q_1 = \delta q_2 = 0$ give Euler-Lagrange equations
\begin{eqn}
0 = \pd{L}{q_i} - \od{}{t} \pd{L}{\dot q_i}.
\end{eqn}
\item Also for fields, have Lagrangian density $\Lag(\phi, \dot \phi, \partial_x \phi)$, action $S = \int \dif t \, \dif x \, \Lag$ and EL equations
\begin{eqn}
0 = \pd{\Lag}{\phi} - \pd{}{t} \pd{\Lag}{\dot \phi} - \pd{}{x} \pd{\Lag}{(\partial_x \phi)}.
\end{eqn}
\end{itemize}

Advantages of the Lagrangian formalism:
\begin{itemize}
\item The equations take the same form under coordinate changes in the Lagrangian
\item Enables us to solve systems with constraints $\Phi_I(q, \dot q, t) = 0$. 
\begin{itemize}
	\item Holonomic constraints (no $\dot q$ dependence) make it easier: $\phi(q,t) = 0$
	\item Sometimes non-holonomic constraints are integrable, i.e. $\tod{\phi}{t} = 0$ which means we can integrate to find new constraint $\widetilde{\phi}(q,t) = \text{const}$ which is holonomic.
	\item Lagrange multipliers:
	\begin{eqn}
	L \to L + \sum_I \lambda_I \Phi_I (q,t)
	\end{eqn}
	this gives us new EL equations
	\begin{eqn}
	0 = \od{}{t} \pd{L}{\dot q_i} - \pd{L}{q_i} - \sum_I \lambda_I \pd{\Phi_I}{q}
	\end{eqn}
	which enforce the constraint. The new term in the EL equations are basically a force term (gradient of contstraint surface) which enforce the constraint. 
	\item Sometimes can find new generalized coordinates $\widetilde{q}_j$, $j = 1,\dots, N-s$ wihich solve the constraint, e.g.
	\begin{eqn}
	0 = \od{}{t} \pd{L}{\dot{\widetilde{q}}_j} - \pd{L}{\widetilde{q}_j}.
	\end{eqn}
\end{itemize}
\end{itemize}

Noether's theorem and symmetries

\begin{itemize}
\item Under a variation $q_i \to q_i' = q_i + \epsilon \delta q_i$, Lagrangian is invariant up to a total time derivative
\begin{eqn}
blah
\end{eqn}
\item Conserved charge and simple example and stuff, but argh he erased the board after writing it
\end{itemize}


\section{Small oscillations}

Generally have a Lagrangian of the form
\begin{eqn}
L = \frac{1}{2} \sum_{ij} M_{ij} (q) \dot q_i \dot q_j + \sum_i N_i (q) \dot q_i - V(q).
\end{eqn}
Next we find the equilibrium points $\bar q_i$ where $\dot{\bar q}_i = \pd{V}{\bar q_i} = 0$, take the difference from equilibrium $\delta q_i = q_i - \bar q_i$, to find a linearized equation of motion
\begin{eqn}
0 = \sum_j \left[ \widetilde{M}_{ij} (\bar q) \delta \ddot q_j + \widetilde{N}_{ij} (\bar q) \delta \dot q_j + \widetilde{V}_{ij} (\bar q) \delta q_j \right].
\end{eqn}
Next find the normal modes
\begin{eqn}
\delta q_i = v_i e^{i \omega t}
\end{eqn}
satisfying
\begin{eqn}
0 = (-\omega^2 \widetilde{M} + i \omega \widetilde{N} + \widetilde{V}) \cdot v.
\end{eqn}
Frequencies are determined by
\begin{eqn}
0 = \det(-\omega^2 \widetilde{M} + i \omega \widetilde{N} + \widetilde{V}),
\end{eqn}
and once we've found frequencies we can find the normal modes $v$. 


\section{Hamiltonian mechanics}

We take the conjugate momentum $p_i = \pd{L}{\dot q_i}$ and form the Hamiltonian via Legendre transform of the Lagrangian,
\begin{eqn}
H = p_i \dot q_i - L(d, \dot q).
\end{eqn}
Then we have Hamilton's equations in terms of $q$ and $p$,
\begin{eqn}
\dot p_i = - \pd{H}{q_i}, \qquad
\dot q_i = \pd{H}{p_i}.
\end{eqn}
Similarly for fields $\phi(x,t)$, we have the canonically conjugate momentum $\Pi_\phi = \pd{\Lag}{\dot \phi}$ and Hamiltonian density
\begin{eqn}
\Ham = \dot \phi \Pi_\phi - \Lag, \qquad
H = \int \dif x \, \Ham.
\end{eqn}

In classical mechanics, and especially Hamiltonian mechanics, we generally want to use Poisson brackets
\begin{eqn}
\cbr{A,B} = \sum_i \left[ \pd{A}{q_i} \pd{B}{p_i} - \pd{A}{p_i} \pd{B}{q_i} \right].
\end{eqn}
The Poisson brackets package the time derivative in a nice form,
\begin{eqn}
\od{A}{t} = \pd{A}{t} + \cbr{A,H}.
\end{eqn}
Note that the conserved charge $Q(p,q)$ from Noether's theorem is conserved so $\cbr{Q,H} = 0$. If we have conserved quantities $Q_1, Q_2$, then the bracket $\cbr{Q_1,Q_2}$ is also conserved. 

A \emph{canonical transformation} 
\begin{eqn}
Q = Q(q,p), \qquad
P = P(q,p)
\end{eqn}
is a coordinate transformation is one which obeys Hamilton's equations in terms of $Q, P$ as canonical coordinates. Simplest characterization is that the Poisson brackets are conserved,
\begin{eqn}
\cbr{Q_i, Q_j}_{qp} = 0, \quad
\cbr{P_i, P_j}_{qp} = 0, \quad
\cbr{Q_i, P_j}_{qp} = \delta_{ij}.
\end{eqn}

\emph{Generating functions} give us a way of constructing canonical transformations. There are four types,
\begin{eqn}
F_1 (q,Q), \quad
F_2 (q,P), \quad
F_3 (p,Q), \quad
F_4 (p,P),
\end{eqn}
and can also depend explicitly on time. Most important are types one (Hamilton-Jacobi) and two (infinitesimal canonical transformations). For type 1, we have
\begin{eqn}
p_i = \pd{F_1}{q_i}, \qquad
P_i = - \pd{F_1}{Q_i}, \qquad
K(Q,P) = H(q,p) + \pd{F_1}{t}.
\end{eqn}
Similarly for type 2 we have
\begin{eqn}
p_i = \pd{F_2}{q_i}, \qquad
Q_i = \pd{F_2}{P_i}, \qquad
K = H + \pd{F_2}{t}.
\end{eqn}

\emph{Action-angle variables} are nice because the motion is basically a circle in phase space. These are coordinates obtained by canonical transformation
\begin{eqn}
(q_i, p_i) \to (\theta_i, I_i)
\end{eqn}
where the momenta $I_i$ are conserved, $\dot I_i = 0$, and the coordiantes $\theta_i$ are linear in time,
\begin{eqn}
\dot \theta_i = \pd{H}{I_i} = \text{const} 
\qquad \implies \qquad
\theta_i = \omega_i t + \theta_0. 
\end{eqn}

Hamilton-Jacobi formalism gives us PDE for a generating function that makes the transformed Hamitlonian trivial, $K = 0$. From above, the expression for the new Hamiltonian and the old momentum give us the HJE
\begin{eqn}
0 = H(\tpd{F_1}{q_i}, q_i) + \pd{F_1}{t},
\end{eqn}
where the new coordinates and momenta are all constants of motion, $\dot Q_i = \dot P_i = 0$ so $Q_i = \alpha_i$, $P_i = \beta_i$. The HJE is not always readily soluble, but if $H$ is not time-dependent then we can make a separation ansatz, 
\begin{eqn}
F = \widetilde{F}(q,Q) - \alpha t. 
\end{eqn}
Sometimes we can further separate $\widetilde{F}$ out into sums of functions of subsets of the coordinates, which simplifies the problem. 


\section{Rigid bodies}

Generally in a rigid body problem we can decompose points in the lab frame as
\begin{eqn}
\v x_\mathrm{lab} = \v x_\mathrm{CM} + R \cdot \v x_\mathrm{body}
\end{eqn}
where $\v x_\mathrm{body}$ is a constant vector in the body frame. In terms of the angular velocity $\v \omega$ we find the lab velocity
\begin{eqn}
\vd x_\mathrm{lab} = \vd x_\mathrm{CM} + R \cdot (\v \omega\times \v x_\mathrm{body}).
\end{eqn}
Note that all quantities here are dependent on time (except $\v x_\mathrm{body}$). So we find for the whole rigid body, the kinetic energy
\begin{eqn}
T = \frac{1}{2} M_\mathrm{tot} \vd x_\mathrm{CM}^2 + \frac{1}{2} \v \omega^\top \cdot I \cdot \v \omega
\end{eqn}
and angular momentum
\begin{eqn}
\v L_\mathrm{lab} = R (I \cdot \v \omega),
\end{eqn}
where $I$ is the moment of inertia tensor
\begin{eqn}
I_{ij} = \sum_{(k)} m_{(k)} \left[ \delta_{ij} \v x_{(k)}^2 - x_{(k)i} x_{(k)j} \right].
\end{eqn}
Generally we can diagonalize the tensor into a form
\begin{eqn}
I = \pmat{I_1 & & \\ & I_2 & \\ & & I_3},
\end{eqn}
so that we find Euler'es equations from $\od{}{t} \v L_\mathrm{lab} = 0$,
\begin{eqn}
0 = I \cdot \vd \omega + \omega \times (I \cdot \v \omega).
\end{eqn}
Also Euler angles and stuff.


\section{Relativity}

Know about boosts and Lorentz transformations. Useful to go to CM frame in collisions. Charged particle in EM field. Kinematics with $p^\mu$, conservation of momentum and energy. 




\end{document}
