% declare document class and geometry
%\documentclass[12pt]{article} % use larger type; default would be 10pt
% ***********************************************************
% ******************* PHYSICS HEADER ************************
% ***********************************************************
% Version 2
\documentclass[12pt]{article} 





\usepackage{datetime} % allows easy formatting of dates, e.g. \formatdate{dd}{mm}{yyyy}

\usepackage{amsmath} % AMS Math Package
\usepackage{amsthm} % Theorem Formatting
\usepackage{amssymb}	% Math symbols such as \mathbb
\usepackage{graphicx} % Allows for eps images
\usepackage{multicol} % Allows for multiple columns
\usepackage[dvips,letterpaper,margin=1in,bottom=1in]{geometry}
 % Sets margins and page size
\pagestyle{empty} % Removes page numbers
\makeatletter % Need for anything that contains an @ command 
%\renewcommand{\maketitle} % Redefine maketitle to conserve space
%{ \begingroup \vskip 10pt \begin{center} \large {\bf \@title}
%	\vskip 10pt \large \@author \hskip 20pt \@date \end{center}
%  \vskip 10pt \endgroup \setcounter{footnote}{0} }
\makeatother % End of region containing @ commands
\renewcommand{\labelenumi}{(\alph{enumi})} % Use letters for enumerate
% \DeclareMathOperator{\Sample}{Sample}
\let\vaccent=\v % rename builtin command \v{} to \vaccent{}
\renewcommand{\v}[1]{\ensuremath{\mathbf{#1}}} % for vectors
\newcommand{\gv}[1]{\ensuremath{\mbox{\boldmath$ #1 $}}} 
% for vectors of Greek letters
\newcommand{\vx}{\ensuremath{\v{x}}} 
% for vectors of Greek letters
\newcommand{\vy}{\ensuremath{\v{y}}} 
% for vectors of Greek letters
\newcommand{\xdot}{\ensuremath{\dot{x}}} 
% for vectors of Greek letters

\newcommand{\ydot}{\ensuremath{\dot{y}}} 
% for vectors of Greek letters
\usepackage{commath} % for some nice standardized syntax stuff. 
	% \dif, \Dif, \od, \pd, \md, \(abs | envert), \(norm | enVert), \(set | cbr), \sbr, \eval, \int(o | c)(o | c), etc
\newcommand{\bbar}[1]{\bar{\bar{#1}}} % for barring things twice -- use \cbar or \zbar instead of original \bbar

\newcommand{\uv}[1]{\ensuremath{\mathbf{\hat{#1}}}} % for unit vector
%\newcommand{\abs}[1]{\left| #1 \right|} % for absolute value
\newcommand{\avg}[1]{\left< #1 \right>} % for average
\let\underdot=\d % rename builtin command \d{} to \underdot{}
\renewcommand{\d}[2]{\frac{d #1}{d #2}} % for derivatives
\newcommand{\dd}[2]{\frac{d^2 #1}{d #2^2}} % for double derivatives
%\newcommand{\pd}[2]{\frac{\partial #1}{\partial #2}} 
% for partial derivatives
\newcommand{\fd}[2]{\frac{\delta #1}{\delta #2}} 
% for functional derivatives

\newcommand{\pdd}[2]{\frac{\partial^2 #1}{\partial #2^2}} 
% for double partial derivatives
\newcommand{\pdc}[3]{\left( \frac{\partial #1}{\partial #2}
 \right)_{#3}} % for thermodynamic partial derivatives
\newcommand{\ket}[1]{\left| #1 \right>} % for Dirac bras
\newcommand{\bra}[1]{\left< #1 \right|} % for Dirac kets
\newcommand{\braket}[2]{\left< #1 \vphantom{#2} \right|
 \left. #2 \vphantom{#1} \right>} % for Dirac brackets
\newcommand{\matrixel}[3]{\left< #1 \vphantom{#2#3} \right|
 #2 \left| #3 \vphantom{#1#2} \right>} % for Dirac matrix elements
\newcommand{\grad}[1]{\gv{\nabla} #1} % for gradient
\let\divsymb=\div % rename builtin command \div to \divsymb
\renewcommand{\div}[1]{\gv{\nabla} \cdot #1} % for divergence
\newcommand{\curl}[1]{\gv{\nabla} \times #1} % for curl
\let\baraccent=\= % rename builtin command \= to \baraccent
\renewcommand{\=}[1]{\stackrel{#1}{=}} % for putting numbers above =
\newtheorem{prop}{Proposition}
\newtheorem{thm}{Theorem}[section]
\newtheorem{lem}[thm]{Lemma}
\theoremstyle{definition}
\newtheorem{dfn}{Definition}
\theoremstyle{remark}
\newtheorem*{rmk}{Remark}
\newcommand{\bigO}{\mathcal{O}} % big O notation
\let \bigo = \bigO % deprecated version. keeping for now because need to update instances in older files










\makeatletter
% À droite
\renewcommand\subsection{\@startsection {subsection}{1}{\z@}%
                                   {-3.5ex \@plus -1ex \@minus -.2ex}%
                                   {2.3ex \@plus.2ex}%
                                   {\raggedright\normalfont\Large\bfseries}}
\makeatother


\makeatletter
\def\section{\@ifstar\unnumberedsection\numberedsection}
\def\numberedsection{\@ifnextchar[%]
  \numberedsectionwithtwoarguments\numberedsectionwithoneargument}
\def\unnumberedsection{\@ifnextchar[%]
  \unnumberedsectionwithtwoarguments\unnumberedsectionwithoneargument}
\def\numberedsectionwithoneargument#1{\numberedsectionwithtwoarguments[#1]{#1}}
\def\unnumberedsectionwithoneargument#1{\unnumberedsectionwithtwoarguments[#1]{#1}}
\def\numberedsectionwithtwoarguments[#1]#2{%
  \ifhmode\par\fi
  \removelastskip
  \vskip 5ex\goodbreak
  \refstepcounter{section}%
  \hbox to \hsize{\vbox{%
      \noindent
      \leavevmode
      \begingroup
      \Large\bfseries\raggedleft
      \thesection.\ 
      #2\par
      \endgroup
      \vskip -2ex
      \noindent\hrulefill
      \vskip -2.2ex\nobreak
      \noindent\hrulefill
      }}\nobreak
  \vskip 2ex\nobreak
  \addcontentsline{toc}{section}{%
    \protect\numberline{\thesection}%
    #1}%
  }
\def\unnumberedsectionwithtwoarguments[#1]#2{%
  \ifhmode\par\fi
  \removelastskip
  \vskip 5ex\goodbreak
%  \refstepcounter{section}%
  \hbox to \hsize{\vbox{%
      \noindent
      \leavevmode
      \begingroup
      \Large\bfseries\raggedleft
%      \thesection.\ 
      #2\par
      \endgroup
      \vskip -2ex
      \noindent\hrulefill
      \vskip -2.2ex\nobreak
      \noindent\hrulefill
      }}\nobreak
  \vskip 2ex\nobreak
  \addcontentsline{toc}{section}{%
%    \protect\numberline{\thesection}%
    #1}%
  }
\makeatother
\pagestyle{empty}




% ***********************************************************
% ********************** END HEADER *************************
% ***********************************************************

\usepackage[margin=1in]{geometry} % handle page geometry





\title{Phys 220A -- Classical Mechanics -- Lec06}
\author{UCLA, Fall 2014}
\date{\formatdate{21}{10}{2014}} % Activate to display a given date or no date (if empty),
         % otherwise the current date is printed 

\begin{document}
\setlength{\unitlength}{1mm}
\maketitle


\section{Small oscillations}

Generally, in classical mechanics, the dynamics may often by too complicated to solve exactly, or even numerically. Often we will have a Lagrangian of the form
\begin{equation}
L = \frac{1}{2} \sum_{ij} M_{ij}(q) \dot{q}_i \dot{q}_j + \sum_i N_i (q) \dot{q}_i - V(q),
\end{equation}
or we will be able to approximate it in this form to order $\dot{q}^2$, assuming no time dependence. This gives us stability around equilibrium, which is of course important physically. An example where $N_i$ comes into play would be the potential for a particle coupled to the EM field. For driven systems we will have different stability properties. In condensed matter, small energy excitations around equilibrium are called phonons. 

Equilibrium points occur at the extrema of the potential, that is at solutions $q_i^{(0)}$ to $\pd{V}{q_i} = 0$. The Euler-Lagrange equations are then solved by $q_i(t) = q_i^{(0)}$ for all $t$, i.e. $\dot{q}_i = 0$. Under small oscillations we will have
\begin{equation}
q_i = q_i^{(0)} + \eta_i(t) + \bigo(\eta^2)
\end{equation}
and we must expand the Lagrangian to quadratic order (i.e. linearize),
\begin{align}
L &= \frac{1}{2} \sum_{ij} M_{ij} (q^{0}) \dot{\eta}_i \dot{\eta}_j + \sum_i N_i (q^{0}) \dot{\eta}_i + \sum_{ij} \eval{\pd{N_i}{q_j}}_{q=q^0} \eta_j \dot{\eta}_i - V(q^0) \\
	& \qquad - \sum_i \eval{\pd{V}{q_i}}_{q = q^0} \eta_i - \frac{1}{2} \eval{\md{V}{2}{q_i}{}{q_j}{}}_{q = q^0} \eta_i \eta_j + \bigo(n^3). \\
	&\sim \frac{1}{2} \sum_{ij} m_{ij} \dot \eta_i \dot \eta_j - \frac{1}{2} \sum_{ij} v_{ij} \eta_i \eta_j + \sum_{ij} p_{ij} \eta_i \dot \eta_j + \bigo(\eta^3),
\end{align}
where in the second expression we've thrown out the $\sum_i N_i (q^0) \dot{\eta}_i$ and $V(q^0)$ terms since they are just a total derivative and an overall constant, and 
\begin{equation}
m_{ij} = M_{ij}(q^0), \qquad p_{ij} = \eval{\pd{N_i}{q_j}}_{q = q^0}, \text{and} \qquad v_{ij} = \eval{\md{V}{2}{q_i}{}{q_j}{}}_{q = q^0}
\end{equation}
are constant matrices. Now, if $p_{ij}$ has a symmetric part $p_{(ij)}$ we can throw it out and just consider it antisymmetric since
\begin{equation}
p_{(ij)} \eta_i \dot{\eta}_j = \frac{1}{2} (p_{ij} + p_{ji}) \eta_i \dot{\eta}_j = p_{ij} (\eta_i \dot{\eta}_j + \eta_j \dot{\eta}_i) = \od{}{t} (\frac{1}{2} p_{ij} \eta_i \eta_j)
\end{equation}
is just a total derivative. 

\subsection{First analysis}

As a first analysis, let's set $p_{ij} = 0$. Then we can write
\begin{equation}
L = \frac{1}{2} \dot{\eta}^\top m \dot{\eta} - \frac{1}{2} \eta^\top v \eta + \bigo(\eta^3),
\end{equation}
for which we can write the equation of motion in terms of the vector $\bar{\eta} = (\eta_1, \dots, \eta_n)$,
\begin{equation}
m \ddot{\bar{\eta}} + v \bar{\eta} = 0.
\end{equation}
This has ansatz $\bar{\eta} = \bar{\gamma} e^{i \omega t}$ which, plugging back in to the equation of motion, we have
\begin{equation}
(-\omega^2 m + v) \bar{\gamma} = 0.
\end{equation}
Thus $\omega$ is determined by the vanishing of the determinant
\begin{equation}
\det (-\omega^2 m + v) = 0.
\end{equation}

However are all directions stable? What does our eigenvector look like? Recall some facts about linear algebra. Both matrices $m_{ij}$ and $v_{ij}$ are symmetric and can hence be diagonalized by orthogonal transformations we have 
\begin{equation}
\bigo^T m \bigo = I \begin{array}{c} m_1 \\ m_2 \\ ... \\ m_n\end{array} = \bar{m}
\end{equation}
We'll assume that all $\bar{m_i}$ are positive (i.e. all masses are in general positive). Then we have a well defined square root of m. We can write
\begin{equation}
\bar{m}^{1/2} \bigo^T m \bigo \bar{m}^{-1/2} = 1
\end{equation}
Defining 
\begin{equation}
n = \bigo \bar{m}^{1/2} \bar{n}
\end{equation}
we have 
\begin{equation}
L = \frac{1}{2} \bar{\dot{n}}^T \bar{n} - \frac{1}{2} \bar{n}^T \bar{V} \bar{n}
\end{equation}
where 
\begin{equation}
\bar{m}^{1/2} \bigo^T v \bigo \bar{m}^{-1/2} = \bar{v}
\end{equation}
is still a symmetric matrix

Our characteristic equation then becomes
\begin{equation}
\ddot{\bar{n}} + \bar{v} \bar{n} = 0
\end{equation}
An ansatz of 
\begin{equation}
\bar{n} = e^{i\omega t} \gamma
\end{equation}
gives 
\begin{equation}
(-\omega^2 1 + \bar{v})\gamma = 0
\end{equation}
If the eigenvalues of $\bar{v}$ are all real and nonzero then the oscillations are stable. Certain directions can be stable depending on the sign of the eigenvalues. Given the eigenvalues the equations of motion are oscillatory with frequencies $\omega =  \sqrt{\lambda_i}$


Thus our equatons of motion are given by
\begin{equation}
\ddot{\bbar{\eta}}_i + \lambda_i \bbar{\eta}_i = 0, \qquad i = 1, \dots, N.
\end{equation}
where in the stable case we have $\lambda_i = \omega_i^2$, the characteristic frequencies. So the normal modes are decoupled 1-dimensional oscillators along $\bbar{\eta}_i$ with coordinate relationship
\begin{equation}
\eta = O \bar{m}^{-1/2} \bar{O} \bar{\bar{\eta}}. 
\end{equation}

In the more general case with $\eta_{ij} \neq 0$, we have
\begin{equation}
L = \frac{1}{2} \dot{\bar{\eta}}^\top \bar{\eta} + \bar{\eta}^\top \bar{p} \dot{\bar{\eta}} - \frac{1}{2} \bar{\eta}^\top \bar{v} \bar{\eta}
\end{equation}
where
\begin{equation}
\bar{p} = \bar{m}^{-1/2} O^\top p O \bar{m}^{-1/2}.
\end{equation}
So the Euler-Lagrange equations are
\begin{equation}
\ddot{\bar{\eta}} + \bar{v} \bar{\eta} - 2 \bar{p} \dot{\bar{\eta}} = 0
\end{equation}
which has a new characteristic equation for the ansatz $\bar{\eta} = \gamma e^{i\omega t}$,
\begin{equation}
\det (-\omega^2 + \bar{v} - 2i \omega \bar{p}) = 0
\end{equation}
which determines $\omega$ and $\gamma$ in general. These cases are important for a particle in an EM field and motion in non-inertial frames. 


\subsection{Double pendulum}

Let's take the simple case that
\begin{equation}
m_1 = m_2 = m
\end{equation}
\begin{equation}
l_1 = l_2 = l
\end{equation}
Our Lagrangian in this case is
\begin{equation}
L = ml^2 \dot{\theta_1}^2 + \frac{1}{2}ml^2 \dot{\theta_2}^2 + ml^2 \dot{\theta_1}\dot{\theta_2} \cos(\theta_1 - \theta_2) + 2mgl \cos \theta_1 + mgl\cos\theta_2
\end{equation}
In equilibrium
\begin{equation}
\theta_1 = \theta_2 = 0
\end{equation}
Let's expand around  this equilibrium and drop the small terms and constants
\begin{equation}
L = ml^2 \dot{\theta_1}^2 + \frac{1}{2} ml^2 \dot{\theta_2}^2  + ml^2 \dot{\theta_1} \dot{\theta_2} - mgl \theta_1^2 - \frac{1}{2} mgl\theta_2^2
\end{equation}

\begin{equation}
m_ij = \left( \begin{array}{cc} 2ml^2 & ml^2 \\ ml^2 & ml^2 \end{array}\right)
\end{equation}
\begin{equation}
v_{ij} = \left(\begin{array}{cc} 2mgl & 0\\ 0 & mgl \end{array}\right)
\end{equation}
Our Euler-Lagrange equations are then
\begin{equation}
\left(\begin{array}{cc} 2ml^2 & ml^2 \\ ml^2 & ml^2 \end{array}\right) \left( \begin{array}{c} \ddot{\theta_1} \\ \ddot{\theta_2} \end{array}\right) + \left( \begin{array}{cc} 2mgl & 0 \\ 0 & mgl \end{array}\right) \left(\begin{array}{c} \theta_1 \\ \theta_2\end{array}\right) = 0
\end{equation}
Dividing by $ml^2$
\begin{equation}
\left(\begin{array}{cc} 2 & 1 \\ 1 & 1 \end{array}\right) \left( \begin{array}{c} \ddot{\theta_1} \\ \ddot{\theta_2} \end{array}\right) + \left( \begin{array}{cc} \frac{2g}{l} & 0 \\ 0 & \frac{g}{l} \end{array}\right) \left(\begin{array}{c} \theta_1 \\ \theta_2\end{array}\right) = 0
\end{equation}
Multiplying on both sides by the inverse of the first term
\begin{equation}
 \left( \begin{array}{c} \ddot{\theta_1} \\ \ddot{\theta_2} \end{array}\right) + \frac{g}{l}\left( \begin{array}{cc} 2 & -1 \\ -2 & 2 \end{array}\right) \left(\begin{array}{c} \theta_1 \\ \theta_2\end{array}\right) = 0
\end{equation}
We then obtain the frequencies from the eigenvalue equation
\begin{equation}
det \left( -\omega^2 + \frac{g}{l} \left( \begin{array}{cc} 2 & -1 \\ -2 & 2\end{array} \right)\right ) = 0
\end{equation}
This results in the equation
\begin{equation}
\omega^4 - \frac{4g}{l} \omega^2 + \frac{2g^2}{l^2} = 0 
\end{equation}
\begin{equation}
\omega^2_\pm = \frac{2g}{l} \left( 1 \pm \frac{1}{\sqrt{2}}\right)
\end{equation}
The eigenvector for the positive frequency is
\begin{equation}
v_+ = \left(\begin{array}{c} -\frac{1}{\sqrt{2}} \\ 1 \end{array} \right)
\end{equation}
This corresponds to the segments oscillating in opposite directions. For the negative frequencies
\begin{equation}
v_- = \left(\begin{array}{c} \frac{1}{\sqrt{2}} \\ 1 \end{array} \right)
\end{equation}
which corresponds to the segments oscillating in the same direction


\section{Lagrange Points}
Consider a two-body planet system. The two masses move in circular
orbits about their common center of mass, with equal velocities. We
can transform into this frame by rotating our frame by the angular
velocity. The Lagrange points are points where you can put a test
mass and it will be stationary \textbf{in the rotating frame}. In the
rest frame, it has the same angular velocity as the other
masses. From the inertial frame the points have positions
\begin{equation}
\v{x_1} = \left(\begin{array} r_1 \cos\omega_t \\ r_1 \sin \omega_t \end{array}\right)
\end{equation}
\begin{equation}
\v{x_2} = \left(\begin{array}{c} -r_2 \cos\omega_t \\ -r_2 \sin \omega_t \end{array}\right)
\end{equation}
where d is the separation between the two, and $r_1$ are and $r_2$ are the distances of each of the particles from the center of mass, such that 
\begin{equation}
d = r_1 + r_2
\end{equation}
\begin{equation}
M = m_1 + m_2
\end{equation}
\begin{equation}
m_1 r_1 = m_2 r_2
\end{equation}
This results in the expressions
\begin{equation}
r_1 = \frac{m_2}{m_1 + m_2} d
\end{equation}
and
\begin{equation}
r_2 = \frac{m_1}{m_1 + m_2} d
\end{equation}
This frame rotates with a frequency determined by Kepler's law and is unaffected by the third mass. 
\begin{equation}
G (m_1 + m_2) = d^3 \omega^2
\end{equation}
In the inertial frame we thus have the Lagrangian
\begin{equation}
\frac{1}{2} \dot{x}^2 + \frac{Gm m_1}{|\v{x} - \v{x_1}|}
+ \frac{Gmm_2}{|\v{x} - \v{x_2}|}
\end{equation}
In general this is dificult since everything depends on t. Let's transform into a rotating frame with angular velocity $\omega$ such that
\begin{equation}
\v{x} = \left( \begin{array}{c} x\cos\omega t - y\sin \omega t \\
x \sin \omega t + y \cos \omega t \\ z\end{array} \right)
\end{equation}

\begin{equation}
\v{x} - \v{x_1} =\left( \begin{array}{c} x\cos\omega t - y\sin \omega
t - r_1 \cos\omega t \\
x \sin \omega t + y \cos \omega t - r_1 \sin\omega t \\z\end{array} \right)
\end{equation}
So we find that the magnitude by multiplying and skipping the gross algebra
\begin{equation}
|\v{x} - \v{x_1}| = \sqrt{(x - r_1)^2 + y^2 + z^2}
\end{equation}
$R_2$ is more or less the same except we flip the sign on r
\begin{equation}
|\v{x} - \v{x_2}| = \sqrt{(x - r_2^2 + y^2 + z^2}
\end{equation}
Now, when we take all of this into our Lagrangian, we can get the
kinetic energy from our normal motion in a rotational frame. We have a
centripetal term (the first) and a centrifugal term (the second)
\begin{equation}
L = \frac{1}{2} m [ \dot{x} + \dot{y} + \dot{z} + \omega^2 (x^2 +
y^2)+ 2\omega (x\dot{y} - y\dot{x})] + Gm [\frac{m_1}{\sqrt{(x -
r_1)^2 + y^2 + z^2}} + \frac{m_2}{\sqrt{(x - r_2^2 + y^2 + z^2)}}]
\end{equation}
Okay, so from common sense, we know that the stability points cannot
be anywhere except for the z plane, there are no forces to stabilize
them in any other plane. The name of the game now is to have a balance
of these forces while minimizing the effective potential. Our first
Lagrange point is simply a balance between the gravitational
forces. This point is stationary in both the inertial frame and the
non-inertial frame so we don't have to consider coriolis forces or
centripetal. 

Now let's consider the other two forces. We have two points that move
in circular orbits in the inertial frame. These points only experience
centripetal forces which balance the gravitational forces. From the
math, these points form equilateral triangles with the two masses. 

The third point is behind each of the masses. These
points only have inward pointing forces from the gravitational forces,
but if we combine the centripetal and centrifugal forces (which point
in the same direction) they cancel the gravitational forces. 


Let's consider the stability of the noncolinear points
The effective potential for is
\begin{align}
\phi_{eff} &= \frac{G\mu}{2R^3}(x^2 + y^2) - \frac{Gm_1}{\sqrt{(r_1 + x)^2 + y^2}} + \frac{Gm_2}{\sqrt{(r_2 - x)^2 + y^2}}
\end{align}
Now we take the first derivative. For our Lagrange points this derivative is equal to zero
\begin{align}
\d{\phi_{eff}}{x} &= \frac{G\mu}{R^3}x - \frac{Gm_1 (r_1-x)}{((r_1 - x)^2 + y^2)^{3/2}} + \frac{Gm_2(r_2 + x)}{((r_2 + x)^2 + y^2)^{3/2}}
\end{align}
Since $r_1 - x = r_2 + x$
\begin{align}
\d{\phi_{eff}}{x} &= \frac{G\mu}{R^3}x - G\frac{m_1  r_1- m_1x - m_2 r_2 - m_2 x}{((r_1 - x)^2 + y^2)^{3/2}} \\ 
&= \frac{G\mu}{R^3}x + G\frac{ x(m_1 + m_2)}{((r_1 - x)^2 + y^2)^{3/2}} &= 0 \\ 
\end{align}
For y we have
\begin{align}
\d{\phi_{eff}}{y} &= \frac{G\mu}{R^3} y - \frac{Gm_1y}{((r_1 -x)^2 + y^2)^{3/2}} + \frac{Gm_2 y }{((r_2 + x)^2 + y^2)^{3/2}}\\
&= \frac{G\mu}{R^3} y + \frac{G(m_1 + m_2)y}{((r_1 -x)^2 + y^2)^{3/2}} 
\end{align}

Okay, now to do the stability analysis, let's perturb the Lagrangian by 
\begin{equation}
x = x_L + \epsilon_x
\end{equation}
and
\begin{equation}
y = y_L + \epsilon_y
\end{equation}
\begin{equation}
\d{}{t} \pd{L}{\dot{\epsilon}} = \pd{L}{\epsilon}
\end{equation}
\begin{equation}
\d{}{t} ( m\dot{\epsilon_x} - 2\omega \epsilon_y) = m\ddot{\epsilon_x} - 2\omega \dot{\epsilon_y} =  \epsilon_x \pdd{\phi_{eff}}{x} + \epsilon_y \frac{\partial^2 \phi_{eff}}{\partial x\partial y}
\end{equation}
Where the partial derivatives are all evaluated at the Lagrange points
\begin{equation}
\ddot{\epsilon_y} + 2\omega\dot{\epsilon_x} = \epsilon_y \pdd{\phi_{eff}}{y} + \epsilon_x \frac{\partial^2 \phi_{eff}}{\partial x\partial y} 
\end{equation}
Let's take these partial derivatives
\begin{equation}
\pdd{\phi_{eff}}{x} = \frac{G\mu}{R^2} + \frac{G(m_1 + m_2)}{((r_1 - x)^2 + y^2)^{3/2}} + \frac{G(m_1 + m_2)x}{((r_1 - x)^2 + y^2)^{3/2}}
\end{equation}
The first and second term are zero due to the first derivative. 
\begin{equation}
\pdd{\phi_{eff}}{x} = \frac{\frac{3}{2}G(m_1 + m_2)x(r_1 - x)}{((r_1 - x)^2 + y^2)^{5/2}}
\end{equation}
We found above that
\begin{equation}
x = \frac{1}{2}(r_1 - r_2)
\end{equation}
\begin{equation}
y = \frac{\sqrt{3}}{2} (r_1 + r_2)
\end{equation}
and 
\begin{equation}
(r_1 - x)^2 + y^2 = d^2
\end{equation}
\begin{equation}
\pdd{\phi_{eff}}{x} = \frac{\frac{3}{2}G(m_1 + m_2)x((\frac{1}{2}r_1 + \frac{1}{2}r_2)}{d^5}
\end{equation}
\begin{equation}
\pdd{\phi_{eff}}{x} = \frac{3}{4} x_l\frac{G(m_1 + m_2)}{R^3}= -\frac{3}{4} \frac{G(m_1 + m_2)}{d^3} = -\frac{3}{4} \omega^2
\end{equation}
You get the idea. The rest are derived in the textbook so I'll skip to the punchline
\begin{equation}
\ddot{\epsilon_x} - 2\omega \dot{\epsilon_y} - \frac{3}{4} \omega^2 \epsilon_x + \frac{3\sqrt{3}}{4} \frac{m_1 - m_2}{m_1 + m_2} \omega^2 \epsilon_y = 0
\end{equation}
\begin{equation}
\ddot{\epsilon_x} +2\omega \dot{\epsilon_y} - \frac{9}{4} \omega^2 \epsilon_x + \frac{3\sqrt{4}}{4}\frac{m_1 - m_2}{m_1 + m_2}\omega^2 \epsilon_y = 0
\end{equation}
We make the ansatz
\begin{equation}
\epsilon_x = c_1 e^{i\omega_s t}
\end{equation}
\begin{equation}
\epsilon_y = c_2 e^{i\omega_s t}
\end{equation}
Which gives us two equations
\begin{equation}
-\omega_s^2 c_1 - 2i\omega \omega_s c_2 - \frac{3}{4} \omega^2 c_1 + \alpha \omega^2 c_2 = 0
\end{equation}
where we have defined
\begin{equation}
\alpha = \frac{3\sqrt{4}}{4}\frac{m_1 - m_2}{m_1 + m_2}
\end{equation}
\begin{equation}
-c_2 \omega_s^2 + 2i\omega \omega_sc_1 - \frac{9}{4} \omega^2 c_2 + \alpha \omega^2 c_1 = 0
\end{equation}
Eliminating the constants we get
\begin{equation}
- 16\omega_s^4 + 16 \omega^2\omega_s^2 + (-27+16\alpha^2)\omega^4  = 0
\end{equation}
Using quadratic formula
\begin{equation}
\omega_s^2 = \frac{1}{32}\left[ 16\omega^2 \pm \sqrt{16^2 \omega^4 - 64 (-27 + 16 \alpha^2) \omega^4}\right] 
\end{equation}
\begin{equation}
\omega_s^2 = \omega^2\left[ \frac{1}{2} \pm \sqrt{\frac{1}{4}  - \frac{1}{16} (-27 + 27 \frac{m_1 - m_2}{m_1 + m_2}) \omega^4}\right] = \omega^2\left[ \frac{1}{2} \pm \sqrt{-\frac{23}{16} + \frac{27}{16} \frac{m_1 - m_2}{m_1 + m_2}) \omega^4}\right] 
\end{equation}
Ourstability condition is
\begin{equation}
\frac{1}{2} > \sqrt{-\frac{23}{16} + \frac{26}{16}( (\frac{m_1}{m_1 + m_2})^2 + \frac{m_1}{m_1 + m_2})}
\end{equation}
Solving we get
\begin{equation}
\mu < \frac{27 - \sqrt{621}}{54} \approx .038
\end{equation}



\end{document}
