% declare document class and geometry
\documentclass[12pt]{article} % use larger type; default would be 10pt
\usepackage[margin=1in]{geometry} % handle page geometry

\input{../header2.tex}

\title{Phys 220A -- Classical Mechanics -- Lec07}
\author{UCLA, Fall 2014}
\date{\formatdate{23}{10}{2014}} % Activate to display a given date or no date (if empty),
         % otherwise the current date is printed 

\begin{document}
\setlength{\unitlength}{1mm}
\maketitle


%\section{Lagrange points}

%Placed in Lec06 material



\section{Parametric resonance}

Question: If a swing has a stable equilibrium how can you make it swing higher and higher?

Let's look at a simple example where we set the mass equal to 1
\begin{eqn}
L = \frac{1}{2} \dot{q}^2 - \frac{1}{2} \omega(t)^2 q^2
\end{eqn}
where $\omega(t + T) = \omega(t)$, i.e. is a periodic function. The simplest example of this is
\begin{eqn}
\omega(t)^2 = \omega_0^2 \left[ 1 + \epsilon \cos\left(\frac{2\pi t}{T}\right) \right]
\end{eqn}
where $\epsilon$ is small. We will simplify by making the choice $T = 2\pi$.

Our equation of motion is thus
\begin{eqn}
\ddot{q} + \omega(t)^2 q = 0
\end{eqn}
We can rewrite this as a first order system
\begin{eqn}
x_1 = q, \qquad 
x_2 = \dot{q}
\end{eqn}
with dynamics given by
\begin{eqn}
\dot{x_1} = x_2, \qquad 
\dot{x_2} = -\omega(t)^2 x_1.
\end{eqn}
Denoting
\begin{eqn}
\v x = \pmat{ x_1 \\ x_2 },
\end{eqn}
our equation becomes
\begin{eqn}
\dot{\v x} = M(t) \v x
\end{eqn}
where $M(t)$ is also a periodic function. The solution is a matrix integral
\begin{eqn}
\v x(t) = P e^{\int_0^t \dif{t'} \, M(t')} \v x(0)
\end{eqn}
But we know that the solution is periodic so that
\begin{eqn}
x(t + 2\pi) = A\v x(t)
\end{eqn}
for some matrix $A$. More generally we have
\begin{eqn}
x(t + 2\pi n) = A^n\v x(t).
\end{eqn}
Now, for a general $2 \times 2$ matrix the characteristic equation is given by
\begin{eqn}
\lambda^2 - \tr(A) \lambda + \det(A) = 0.
\end{eqn}
Thus we have
\begin{eqn}
\lambda_\pm = \frac{\tr A}{2} \pm \sqrt{ \frac{(\tr A)^2}{4}  - 1},
\end{eqn}
where we can see that the determinant should be $1$ from the Hamiltonian. Thus for real eigenvalues our solutions will be stable for $\abs{\tr A} < 2$ and unstable for $\abs{\tr A} > 2$. 

For a general $\omega(t)$ the analysis is difficult, but it's simple if we consider a small perturbation. First consider the unperturbed case $\omega^2(t) = \omega_0^2$ so that the dynamical system is given by
\begin{eqn}
\dot{x_1} = x_2, \qquad
\dot{x_2} = -\omega_0^2 x_1
\end{eqn}
or represented by
\begin{eqn}
M = \left( \begin{array}{cc} 0 & 1 \\ -\omega_0^2 & 0  \end{array} \right).
\end{eqn}
Then we have
\begin{gather}
A(t) = \exp(Mt) = 
\begin{pmatrix}
\cos \omega_0 t & \frac{1}{\omega_0} \sin \omega_0 t \\
-\omega_0 \sin \omega_0 t & \cos \omega_0 t 
\end{pmatrix}, \\
A(t = 2\pi) = 
\begin{pmatrix}
\cos 2\pi \omega_0 & \frac{1}{\omega_0} \sin 2\pi \omega_0 \\
-\omega_0 \sin 2\pi \omega_0 & \cos 2\pi \omega_0
\end{pmatrix}
\end{gather}
and thus we have $\det A = 1$ as expected and $\tr A = 2 \cos 2\pi \omega_0$. Therefore, we have marginal stability when $\abs{\tr A} = 2$ so that $\abs{\cos 2\pi \omega_0} = 1$, thus
\begin{eqn}
\omega_0 = 0, \frac{1}{2}, 1, \frac{3}{2}, \dots.
\end{eqn}
So, away from $\omega_0$ the motion is stable even for small $\epsilon$. But for perturbations at or very near these \emph{parametric resonance frequencies}, we have instability. 


\section{Legendre transform (for Hamiltonian mechanics)}

Given a function $f(x)$, we say that it is convex if $f''(x) > 0$. If it is, then we will have a Legendre transform
\begin{eqn}
\mathcal{L} : f(x) \mapsto (\mathcal{L}f)(p)
\end{eqn}
taking $f(x)$ from position space to a function $(\mathcal{L}f)(p)$ in momentum space. Define
\begin{eqn}
F(p,x) = px - f(x),
\end{eqn}
which we picture as the difference between a line of slope $p$ and the convex function $f(x)$. Then the Legendre transformed function is defined as
\begin{eqn}
(\mathcal{L}f)(p) = \sup_{x \in \reals} F(p,x). 
\end{eqn}
Put another way, given $p$ there exists some $x(p)$ which maximizes $F(p,x)$, i.e.
\begin{eqn}
\eval[3]{\pd{F}{x}(p,x)}_{x = x(p)} = 0,
\end{eqn}
Then we have 
\begin{eqn}
\mathcal{L}f(p) = F(p,x(p)).
\end{eqn}

[missed examples here: $f(x) = ax^2$ and $f(x) = x^a / a$]

An important property of Legendre transforms is that they are \textit{involutions}, that is the Legendre transform squares to the identity. In other words,
\begin{eqn}
\mathcal{L}^2 f(x) = \mathcal{L}(\mathcal{L}f)(x) = f(x).
\end{eqn}
Another useful property is \textit{Young's inequality},
\begin{eqn}
px - f(x) \leq \mathcal{L}f(p).
\end{eqn}



\end{document}
