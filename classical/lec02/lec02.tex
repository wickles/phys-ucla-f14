% declare document class and geometry
\documentclass[12pt]{article} % use larger type; default would be 10pt
\usepackage[margin=1in]{geometry} % handle page geometry

% standard packages
\usepackage{graphicx} % support the \includegraphics command and options
\usepackage{amsmath} % for nice math commands and environments

% font packages
\usepackage{amssymb} % for \mathbb, \mathfrak fonts
\usepackage{mathrsfs} % for \mathscr font
\DeclareMathAlphabet{\mathpzc}{OT1}{pzc}{m}{it} % defines \mathpzc for Zapf Chancery (standard postscript) font

% other packages
\usepackage{datetime} % allows easy formatting of dates, e.g. \formatdate{dd}{mm}{yyyy}
\usepackage{caption} % makes figure captions better, more configurable
\usepackage[squaren]{SIunits} % for nice units formatting e.g. \unit{50}{\kilo\gram}
\usepackage{cancel} % for crossing out terms with \cancel
\usepackage{verbatim} % for verbatim and comment environments
\usepackage{tensor} % for \indices e.g. M\indices{^a_b^{cd}_e}, and \tensor e.g. \tensor[^a_b^c_d]{M}{^a_b^c_d}
\usepackage{feynmp-auto} % for Feynman diagrams. 
\usepackage{pgfplots} % for plotting in tikzpicture environment

% new commands
\newcommand{\fslash}[1]{#1\!\!\!/} % feynman slash
\newcommand{\opname}[1]{\operatorname{#1}} % custom operator names
\newcommand{\pd}{\partial} % partial differential shortcut
\newcommand{\ket}[1]{\left| #1 \right>} % for Dirac kets
\newcommand{\bra}[1]{\left< #1 \right|} % for Dirac bras
\newcommand{\braket}[2]{\left< #1 \vphantom{#2} \right| 
	\left. #2 \vphantom{#1} \right>} % for Dirac brackets

\let\vaccent=\v % rename builtin command \v{} to \vaccent{}
%\renewcommand{\v}[1]{\ensuremath{\mathbf{#1}}} % for vectors
\renewcommand{\v}[1]{\ensuremath{\boldsymbol{\mathbf{#1}}}} % for vectors
%\newcommand{\gv}[1]{\ensurmath{\mbox{\boldmath$ #1 $}}} % for vectors of Greek letters
\newcommand{\uv}[1]{\ensuremath{\boldsymbol{\mathbf{\widehat{#1}}}}} % for unit vectors
\newcommand{\abs}[1]{\left| #1 \right|} % for absolute value ||x||
\newcommand{\norm}[1]{\left\Vert #1 \right\Vert} % for norm ||v||
\newcommand{\avg}[1]{\left< #1 \right>} % for average <x>
\newcommand{\inner}[2]{\left< #1, #2 \right>} % for inner product <x,y>
\newcommand{\set}[1]{ \left\{ #1 \right\} } % for sets {a,b,c,...}

% shortcuts
\newcommand{\reals}{\mathbb{R}} % real numbers
\newcommand{\complexes}{\mathbb{C}} % complex numbers
\newcommand{\nats}{\mathbb{N}} % natural numbers
\newcommand{\irrats}{\mathbb{Q}} % irrationals
\newcommand{\quats}{\mathbb{H}} % quaternions (a la Hamilton)
\newcommand{\euclids}{\mathbb{E}} % Euclidean space
\newcommand{\bigo}{\mathcal{O}} % big O notation





%%%%%%%%%%%%%%%%%%%
% some templates for various things
\begin{comment}

% template for figures
\begin{figure}
\centering
\includegraphics{myfile.png}
\caption{This is a caption}
\label{fig:myfigure}
\end{figure}

% template for Feynman diagrams using feynmf/feynmp
\begin{fmfgraph*}(40,25)
\fmfleft{em,ep}
\fmf{fermion}{em,Zee,ep}
\fmf{photon,label=$Z$}{Zee,Zff}
\fmf{fermion}{fb,Zff,f}
\fmfright{fb,f}
\fmfdot{Zee,Zff}
\end{fmfgraph*}

% template for drawing plots with pgfplot
\pgfplotsset{compat=1.3,compat/path replacement=1.5.1}
\begin{tikzpicture}
\begin{axis}[
extra x ticks={-2,2},
extra y ticks={-2,2},
extra tick style={grid=major}]
\addplot {x};
\draw (axis cs:0,0) circle[radius=2];
\end{axis}
\end{tikzpicture}

\end{comment}
%%%%%%%%%%%%%%%%%%%


\title{Phys 220A -- Classical Mechanics -- Lec02}
\author{UCLA, Fall 2014}
\date{\formatdate{07}{10}{2014}} % Activate to display a given date or no date (if empty),
         % otherwise the current date is printed 

\begin{document}
\setlength{\unitlength}{1mm}
\maketitle


\section{Introduction}

A couple of general comments about Newton's laws. 

\subsection{Something funny about Lorentz force and magnetic fields}

Consider the Lorentz force between two charged particles, given by $\v{F}_{12} = q \v{v}_1 \times \v{B}_2$. Consider particle 1 to lie on the positive $y$ axis with velocity in the positive $y$ direction and particle 2 to lie on the positive $z$ axis with velocity in the positive $x$ axis. Then we find that $\v{F}_{12} \neq 0$ points in the positive $x$ direction yet $\v{F}_{21} = 0$. 

What's going on here? Is this a case of violation of angular momentum conservation? No, in fact the EM field itself carries angular momentum. 


\subsection{Virial Theorem (useful in stat. mech)}

Take some function $f$ and write the time average
\begin{equation}
\avg{f} = \lim_{T \rightarrow \infty} \frac{1}{T} \int_{t_0}^{t_0 + T} f(t) dt.
\end{equation}
Note that there is no dependence on $t_0$ for a bounded quantity,
\begin{equation}
\frac{\pd}{\pd t_0} \avg{f_0} = \lim_{T \rightarrow \infty} \frac{f(t_0 + T) - f(t_0)}{T} = 0
\end{equation}
which vanishes because the numerator in the limit is bounded. 

Next, we hope to obtain an expression for $\avg{T}$ for conservative forces. Notice that 
\begin{equation}
\sum_i m_i \ddot{\v{x}}_i \cdot \v{x} = \sum_i \v{F}_i \cdot \v{x}_i,
\end{equation}
which we can rewrite as 
\begin{equation}
\frac{d}{dt} \left( \sum_i m_i \dot{\v{x}}_i \cdot \v{x} \right) - \underbrace{\sum_i m_i \dot{\v{x}}_i^2}_{2T} = - \sum_i \v{\nabla}_i V \cdot \v{x}_i.
\end{equation}
If the motion is bounded then the derivative goes away when we average, so averaging over time we find
\begin{equation}
2 \avg{T} = \avg{ \sum_i \v{\nabla}_i V \cdot \v{x}_i }.
\end{equation}
This is the Virial theorem. It is especially useful if $V$ is homogeneous, i.e. 
\begin{equation}
V(\lambda \v{x}_1, \dots \lambda \v{x}_n) = \lambda^k V(\v{x}_1, \dots, \v{x}_n)
\end{equation}
for some constant $k$ and any value of $\lambda$. Taking the derivative of this equation w.r.t. $\lambda$ we have
\begin{equation}
\sum_i \v{\nabla}_i V(\lambda \v{x}_1, \dots, \lambda \v{x}_n) \cdot \v{x}_i = k \lambda^{k-1} V(\v{x}_1, \dots, \v{x}_n)
\end{equation}
and setting $\lambda = 1$ we find $\sum_i \v{\nabla} V \cdot \v{x}_i = k V$. Thus the Virial theorem for a homogeneous potential gives us
\begin{equation}
2 \avg{T} = k \avg{V}.
\end{equation}


[incomplete]






\end{document}
