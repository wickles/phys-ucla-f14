% declare document class and geometry
\documentclass[12pt]{article} % use larger type; default would be 10pt
\usepackage[margin=1in]{geometry} % handle page geometry

% standard packages
\usepackage{graphicx} % support the \includegraphics command and options
\usepackage{amsmath} % for nice math commands and environments
\usepackage{mathtools} % extends amsmath with bug fixes and useful commands e.g. \shortintertext
\usepackage{amsthm} % for theorem and proof environments

% font packages
\usepackage{amssymb} % for \mathbb, \mathfrak fonts
\usepackage{mathrsfs} % for \mathscr font
\DeclareMathAlphabet{\mathpzc}{OT1}{pzc}{m}{it} % defines \mathpzc for Zapf Chancery (standard postscript) font

% other packages
\usepackage{datetime} % allows easy formatting of dates, e.g. \formatdate{dd}{mm}{yyyy}
\usepackage{caption} % makes figure captions better, more configurable
\usepackage{enumitem} % allows for custom labels on enumerated lists, e.g. \begin{enumerate}[label=\textbf{(\alph*)}]
\usepackage[squaren]{SIunits} % for nice units formatting e.g. \unit{50}{\kilo\gram}
\usepackage{cancel} % for crossing out terms with \cancel
\usepackage{verbatim} % for verbatim and comment environments
\usepackage{tensor} % for \indices e.g. M\indices{^a_b^{cd}_e}, and \tensor e.g. \tensor[^a_b^c_d]{M}{^a_b^c_d}
\usepackage{feynmp-auto} % for Feynman diagrams. 
\usepackage{pgfplots} % for plotting in tikzpicture environment
\usepackage{commath} % for some nice standardized syntax stuff. \dif, \Dif \od, \pd, \md, \(abs | envert), \(norm | enVert), \(set | cbr), \sbr, \eval, \int(o | c)(o | c), etc
\usepackage{slashed} % provides a command \slashed[1] for Feynman slash notation
%\newcommand{\fslash}[1]{#1\!\!\!/} % feynman slash
%\newcommand{\fsl}[1]{\ensuremath{\mathrlap{\!\not{\phantom{#1}}}#1}}% \fsl{<symbol>}
	% alternative feynman slash

% new commands
\newcommand{\beg}{\begin} % a few letters less for beginning environments
\newenvironment{eqn}{\begin{equation}}{\end{equation}} % a lot fewer letter for equation environment

% rotate stuff
\usepackage{rotating}
	% provides environments for rotating arbitrary objects, e.g. sideways, turn[ang], rotate[ang]
	% also provides macro \turnbox{ang}{stuff}
%\newcommand{\sideways}[1]{\begin{sideways} #1 \end{sideways}} % turn things 90 degrees CCW
%\newcommand{\turn}[2][]{\begin{turn}{#2} #1 \end{turn}} % \turn[ang]{stuff} turns things arbitrary +/- angle

% notational commands
\newcommand{\opname}[1]{\operatorname{#1}} % custom operator names
%\newcommand{\pd}{\partial} % partial differential shortcut
\newcommand{\ket}[1]{\left| #1 \right>} % for Dirac kets
%\newcommand{\ket}[1]{| #1 \rangle}
\newcommand{\bra}[1]{\left< #1 \right|} % for Dirac bras
%\newcommand{\bra}[1]{\langle #1 |}
\newcommand{\braket}[2]{\left< #1 \vphantom{#2} \right| \left. #2 \vphantom{#1} \right>} 
	% for Dirac bra-kets \braket{bra}{ket}
%\newcommand{\braket}[2]{\langle #1 | #2 \rangle} 
\newcommand{\matrixel}[3]{\left< #1 \vphantom{#2#3} \right| #2 \left| #3 \vphantom{#1#2} \right>} 
	% for Dirac matrix elements \matrixel{bra}{op}{ket}
%\newcommand{\matrixel}[3]{\langle #1 | #2 | #3 \rangle} 

%\newcommand{\pd}[2]{\frac{\partial #1}{\partial #2}} % for partial derivatives
%\newcommand{\fd}[2]{\frac{\delta #1}{\delta #2}} % for functional derivatives
\let \vaccent = \v % rename builtin command \v{} to \vaccent{}
%\renewcommand{\v}[1]{\ensuremath{\mathbf{#1}}} % for vectors
\renewcommand{\v}[1]{\ensuremath{\boldsymbol{\mathbf{#1}}}} % for vectors
%\newcommand{\gv}[1]{\ensurmath{\mbox{\boldmath$ #1 $}}} % for vectors of Greek letters
\newcommand{\uv}[1]{\ensuremath{\boldsymbol{\mathbf{\widehat{#1}}}}} % for unit vectors
%\newcommand{\abs}[1]{\left| #1 \right|} % for absolute value ||x||
%\newcommand{\mag}{\abs} % magnitude, just another name for \abs
%\newcommand{\norm}[1]{\left\Vert #1 \right\Vert} % for norm ||v||
\newcommand{\vd}[1]{\v{\dot{#1}}} % for dotted vectors
\newcommand{\vdd}[1]{\v{\ddot{#1}}} % for ddotted vectors
\newcommand{\vddd}[1]{\v{\dddot{#1}}} % for dddotted vectors
\newcommand{\vdddd}[1]{\v{\ddddot{#1}}} % for ddddotted vectors
\newcommand{\avg}[1]{\left< #1 \right>} % for average <x>
\newcommand{\inner}[2]{\left< #1, #2 \right>} % for inner product <x,y>
%\newcommand{\set}[1]{ \left\{ #1 \right\} } % for sets {a,b,c,...}
\newcommand{\tr}{\opname{tr}} % for trace
\newcommand{\Tr}{\opname{Tr}} % for Trace
\newcommand{\rank}{\opname{rank}} % for rank
\let \fancyre = \Re
\let \fancyim = \Im
\newcommand{\Res}{\opname{Res}\limits} % for residue function -- change to put limits on bottom
\renewcommand{\Re}{\opname{Re}}
\renewcommand{\Im}{\opname{Im}}
\renewcommand{\bbar}[1]{\bar{\bar{#1}}} 
	% for barring things twice -- use \cbar or \zbar instead of original \bbar
\newcommand{\bbbar}[1]{\bar{\bbar{#1}}}
\newcommand{\bbbbar}[1]{\bar{\bbbar{#1}}}

\newcommand{\inv}{^{-1}}

% temporary fixes -- commath's versions are bad for powers, like $\dif^3 x$
\renewcommand{\dif}{\mathrm{d}} % \opname{d} better maybe?
\renewcommand{\Dif}{\mathrm{D}}

% notational shortcuts
\newcommand{\bigO}{\mathcal{O}} % big O notation
\let \bigo = \bigO % keep for now, need to update instances in older files
\newcommand{\Lag}{\mathcal{L}} % fancy Lagrangian
\newcommand{\Ham}{\mathcal{H}} % fancy Hamiltonian
\newcommand{\reals}{\mathbb{R}} % real numbers
\newcommand{\complexes}{\mathbb{C}} % complex numbers
\newcommand{\ints}{\mathbb{Z}} % integers
\newcommand{\nats}{\mathbb{N}} % natural numbers
\newcommand{\irrats}{\mathbb{Q}} % irrationals
\newcommand{\quats}{\mathbb{H}} % quaternions (a la Hamilton)
\newcommand{\euclids}{\mathbb{E}} % Euclidean space
\newcommand{\R}{\reals}
\newcommand{\C}{\complexes}
\newcommand{\Z}{\ints}
\newcommand{\Q}{\irrats}
\newcommand{\N}{\nats}
\newcommand{\E}{\euclids}
\newcommand{\RP}{\mathbb{RP}} % real projective space
\newcommand{\CP}{\mathbb{CP}} % complex projective space

% matrix shortcuts!
\newcommand{\pmat}[1]{\begin{pmatrix} #1 \end{pmatrix}}
\newcommand{\bmat}[1]{\begin{bmatrix} #1 \end{bmatrix}}
\newcommand{\Bmat}[1]{\begin{Bmatrix} #1 \end{Bmatrix}}
\newcommand{\vmat}[1]{\begin{vmatrix} #1 \end{vmatrix}}
\newcommand{\Vmat}[1]{\begin{Vmatrix} #1 \end{Vmatrix}}


% more stuff
\newenvironment{enumproblem}{\begin{enumerate}[label=\textbf{(\alph*)}]}{\end{enumerate}}
	% for easily enumerating letters in problems
\newcommand{\grad}[1]{\v{\nabla} #1} % for gradient
\let \divsymb = \div % rename builtin command \div to \divsymb
\renewcommand{\div}[1]{\v{\nabla} \cdot #1} % for divergence
\newcommand{\curl}[1]{\v{\nabla} \times #1} % for curl
\let \baraccent = \= % rename builtin command \= to \baraccent
\renewcommand{\=}[1]{\stackrel{#1}{=}} % for putting numbers above =


% theorem-style environments. note amsthm builtin proof environment: \begin{proof}[title]
% appending [section] resets counter and prepends section number
% use \setcounter{counter}{0} to reset counter
% typical use cases:
% plain: Theorem, Lemma, Corollary, Proposition, Conjecture, Criterion, Algorithm
% definition: Definition, Condition, Problem, Example
% remark: Remark, Note, Notation, Claim, Summary, Acknowledgment, Case, Conclusion
\theoremstyle{plain} % default
\newtheorem{theorem}{Theorem}[section]
\newtheorem{lemma}[theorem]{Lemma}
\newtheorem{corollary}[theorem]{Corollary}
\newtheorem{proposition}[theorem]{Proposition}
\newtheorem{conjecture}[theorem]{Conjecture}
% definition style
\theoremstyle{definition}
\newtheorem{definition}{Definition}
\newtheorem{problem}{Problem}
\newtheorem{exercise}{Exercise}
\newtheorem{example}{Example}
% remark style
\theoremstyle{remark}
\newtheorem{remark}{Remark}
\newtheorem{note}{Note}
\newtheorem{claim}{Claim}
\newtheorem{conclusion}{Conclusion}
% to-do: add problem/subproblem/answer environments for homeworks









%%%%% derivatives


\let \underdot = \d % rename builtin command \d{} to \underdot{}
\let \d = \od % for derivatives

% BUG: derivatives revert to text mode often when in smaller environments in math mode?


% Command for functional derivatives. The first argument denotes the function and the second argument denotes the variable with respect to which the derivative is taken. The optional argument denotes the order of differentiation. The style (text style/display style) is determined automatically
\providecommand{\fd}[3][]{\ensuremath{
\ifinner
\tfrac{\delta{^{#1}}#2}{\delta{#3^{#1}}}
\else
\dfrac{\delta{^{#1}}#2}{\delta{#3^{#1}}}
\fi
}}

% \tfd[2]{f}{k} denotes the second functional derivative of f with respect to k
% The first letter t means "text style"
\providecommand{\tfd}[3][]{\ensuremath{\mathinner{
\tfrac{\delta{^{#1}}#2}{\delta{#3^{#1}}}
}}}
% \dfd[2]{f}{k} denotes the second functional derivative of f with respect to k
% The first letter d means "display style"
\providecommand{\dfd}[3][]{\ensuremath{\mathinner{
\dfrac{\delta{^{#1}}#2}{\delta{#3^{#1}}}
}}}

% mixed functional derivative - analogous to the functional derivative command
% \mfd{F}{5}{x}{2}{y}{3}
\providecommand{\mfd}[6]{\ensuremath{
\ifinner
\tfrac{\delta{^{#2}}#1}{\delta{#3^{#4}}\delta{#5^{#6}}}
\else
\dfrac{\delta{^{#2}}#1}{\delta{#3^{#4}}\delta{#5^{#6}}}
\fi
}}


% Command for thermodynamic (chemistry?) partial derivatives. The first argument denotes the function and the second argument denotes the variable with respect to which the derivative is taken. The optional argument denotes the order of differentiation. The style (text style/display style) is determined automatically
\providecommand{\pdc}[4][]{\ensuremath{
\ifinner
\left( \tfrac{\partial{^{#1}}#2}{\partial{#3^{#1}}} \right)_{#4}
\else
\left( \dfrac{\partial{^{#1}}#2}{\partial{#3^{#1}}} \right)_{#4}
\fi
}}

% \tpd[2]{f}{k} denotes the second thermo partial derivative of f with respect to k
% The first letter t means "text style"
\providecommand{\tpdc}[4][]{\ensuremath{\mathinner{
\left( \tfrac{\partial{^{#1}}#2}{\partial{#3^{#1}}} \right)_{#4}
}}}
% \dpd[2]{f}{k} denotes the second thermo partial derivative of f with respect to k
% The first letter d means "display style"
\providecommand{\dpdc}[4][]{\ensuremath{\mathinner{
\left( \dfrac{\partial{^{#1}}#2}{\partial{#3^{#1}}} \right)_{#4}
}}}


%%%%%%





%%%%%%%%%%%%%%%%%%%
% some templates for various things
\begin{comment}

% template for figures
\begin{figure}
\centering
\includegraphics{myfile.png}
\caption{This is a caption}
\label{fig:myfigure}
\end{figure}

% template for Feynman diagrams using feynmf/feynmp
\begin{fmfgraph*}(40,25)
\fmfleft{em,ep}
\fmf{fermion}{em,Zee,ep}
\fmf{photon,label=$Z$}{Zee,Zff}
\fmf{fermion}{fb,Zff,f}
\fmfright{fb,f}
\fmfdot{Zee,Zff}
\end{fmfgraph*}

% template for drawing plots with pgfplot
\pgfplotsset{compat=1.3,compat/path replacement=1.5.1}
\begin{tikzpicture}
\begin{axis}[
extra x ticks={-2,2},
extra y ticks={-2,2},
extra tick style={grid=major}]
\addplot {x};
\draw (axis cs:0,0) circle[radius=2];
\end{axis}
\end{tikzpicture}

%% find package for easily drawing mapping / algebraic / commutative diagrams..

\end{comment}
%%%%%%%%%%%%%%%%%%%



%%%%% A note on spacing
% 5) \qquad
% 4) \quad
% 3) \thickspace = \;
% 2) \medspace = \:
% 1) \thinspace = \,
% -1) \negthinspace = \!
% -2) \negmedspace
% -3) \negthickspace




\title{Phys 220A -- Classical Mechanics -- Lec11}
\author{UCLA, Fall 2014}
\date{\formatdate{13}{11}{2014}} % Activate to display a given date or no date (if empty),
         % otherwise the current date is printed 

\begin{document}
\setlength{\unitlength}{1mm}
\maketitle


\section{Action-Angle Variables}

Consider the Harmonic oscillator
\begin{eqn}
H = \frac{p^2}{2m} + \frac{1}{2} m \omega^2 q^2,
\end{eqn}
where we found that the trajectories in phase space make ellipses centered about the origin. This suggests that we could make a canonical transformation to nice polar coordinates in phase space,
\begin{eqn}
(p,q) \rightarrow (I,\theta), \qquad 
q = \sqrt{\frac{2I}{m\omega}} \sin\theta, \qquad
p = \sqrt{2Im\omega} \cos\theta,
\end{eqn}
This results in a very nice form for the Hamiltonian, $H' = \omega I$, and thus the equations of motion are just
\begin{eqn}
\dot \theta = \omega, \qquad
\dot I = 0.
\end{eqn}
So we can visualize the trajectories as constant motion about a ring on a cylinder, where the ring is at height $I$ and angles are labelled $\theta$. These are called action-angle variables.

Suppose instead we have a system with $N$ degrees of freedom. For certain systems there exists a canonical transformation to a complete set of action-angle variables. This defines a special class of \textit{integrable} systems, for which, more precisely, we have
\begin{eqn}
(q_i, p_i) \rightarrow (I_i, \theta_i), \qquad
H' = \sum_i \omega_i I_i,
\end{eqn}
so that the equations of motion are just
\begin{eqn}
\dot \theta_i = \omega_i, \qquad
\dot I_i = 0.
\end{eqn}
So we have motion in $\theta_i$ on $(S^1)^n = T^n$, the $n$-torus. 

\begin{theorem}[Liouville's theorem]
Suppose we have some canonical coordinates $I_1, \dots, I_N$ with 
\begin{eqn}
\cbr{I_i, H'} = \cbr{I_i, I_j} = 0, \qquad
i,j = 1, \dots, N.
\end{eqn}
Then the system is integrable. 
\end{theorem}

In other words, the system is integrable if there exists a canonical transformation with a complete set of conserved momenta. For example, in the central potential problem we should also have 3 action-angle variables. This also applies to something with infinitely many degrees of freedom like a field theory. 

Integrability is really the exception and not the rule---integral systems are special. Furthermore, integrability is a global property. Consider motion along the phase space in a non-integrable system. Locally, the path looks like a tiny line segment so we can rotate into action-angle variables. Thus any path in phase space is locally integrable in a small neighborhood, but this is not always the case. In an integrable system, we can define action-angle variables globally, in every coordinate patch. 


\subsection{Some Examples}

\begin{itemize}
\item Elementary examples: Harmonic oscillator; Motion in central potential; Spinning top

\item \textit{Toda-lattice}: mass points coupled by nearest neighbor interactions where the Hamiltonian looks like
\begin{equation}
H = \sum_{i = 1}^{N+1} \frac{1}{2} p_1^2 + \omega^2 e^{q_i - q_{i+1}}
\end{equation}
We can link the last to the first one to make a circle and this is the $SU(N)$ lattice. This is hard to solve with conventional methods because exponentials are scary. 

\item \textit{Calogero-Moser} systems, which have an interparticle potential
\begin{equation}
H = \sum_i \frac{1}{2} p_i^2 + \sum_{i\ne j}^N V(q_i - q_j)
\end{equation}
of the form
\begin{equation}
V_1(q) = \frac{1}{q^2}, \quad
V_2(q) = -\frac{4}{\sinh^2 (q/2)}, \quad
\text{or} \quad
V_3(q) = \frac{4}{\sin^2 (q/2)}
\end{equation}

\item Study of integrable systems is a full subfield of mathematical physics especially when applied to field theories

\item The best approach to finding action angle variables and for checking whether a system is integrable is to use the Hamilton-Jacobi formalism, onto which we now shift our focus. 

\end{itemize}


\section{Hamilton-Jacobi theory}

Motivations:
\begin{itemize}
\item New formulation of mechanics
\item It gives a practical way to construct action/angle variables (separability) (sometimes)
\item Most efficient way of solving dynamics
\item New way of thinking about the action
\item Geometrical optics / semiclassical approximation (QM)
\end{itemize}

In the Hamilton-Jacobi formulation, we are interested in finding a generating function $F(q,Q,t)$ such that the new Hamiltonian is as simple as possible. What's the simplest Hamiltonian?
\begin{eqn}
K(P,Q,t) \equiv 0.
\end{eqn}
Then $\dot Q = \dot P = 0$ or $Q \equiv \alpha, P \equiv \beta$, so we've just moved the dynamics from the Hamiltonian into the generating function. So we can write, using our results from the generating function formalism,
\begin{eqn}
0 = K(Q_i,P_i,t) = H(p_i,q_i,t) + \pd{F}{t}.
\end{eqn}
This is a canonical transformation of type 1, $F_1 = F$, so we have
\begin{eqn}
p_i = \pd{F}{q_i}, \qquad
P_i = -\pd{F}{Q_i}.
\end{eqn}
Then we can rewrite $K = 0$ as 
\begin{eqn}
H(q_i, \tpd{F(q,Q,t)}{q_i}, t) + \pd{F(q,Q,t)}{t} = 0,
\end{eqn}
which is the \textbf{Hamilton-Jacobi equation} (HJE). As a comparison, recall that Hamilton's equations give us a system of $2N$ first order ODEs. Here in the Hamilton-Jacobi formalism, we have a single first order PDE in $N+1$ variables. 

As a first example, let's look at the free particle $H = p^2 / 2m$. Then the HJE can be written
\begin{eqn}
\frac{1}{2m} \left(\pd{F}{q}\right)^2 + \pd{F}{t} = 0.
\end{eqn}
We see that even for a very simple example, we can end up with a difficult problem because we now have a nonlinear PDE. By inspection, we can write down a solution 
\begin{eqn}
F(q,Q,t) = \frac{m(q-Q)^2}{2t},
\end{eqn}
then we find
\begin{eqn}
p = \pd{F}{q} = \frac{m(q-Q)}{t}, \qquad
P = -\pd{F}{Q} = \frac{m(q-Q)}{t}.
\end{eqn}
Recall that $Q$ and $P$ are supposed to be constant---looking at their forms, we notice that $\alpha \equiv Q = q(0)$ and $\beta \equiv P = p(0)$ and we can write
\begin{eqn}
q(t) = \frac{\beta}{m} t + \alpha.
\end{eqn}
Furthermore, $F$ looks familiar:
\begin{eqn}
F = \frac{p(0)^2}{2m} t,
\end{eqn}
it's just the action! 
\begin{eqn}
S = \int_0^t \frac{1}{2} m \dot q^2 \dif{t} = \frac{\beta^2}{2m} t.
\end{eqn}
So we've found that $F$ evaluated on the solution is the action. 

It turns out that this solution is not unique. Another solution is given by
\begin{eqn}
F = q \sqrt{2m\alpha} - \alpha t,
\end{eqn}
so we have
\begin{eqn}
p = \pd{F}{q} = \sqrt{2m \alpha}, \qquad
P = -\pd{F}{\alpha} = \frac{q\sqrt{2m}}{2\sqrt{\alpha}} - t = \beta,
\end{eqn}
or 
\begin{eqn}
q = \sqrt{\frac{2\alpha}{m}} (t+\beta), \qquad
p = \sqrt{2m\alpha}.
\end{eqn}
[Is this an equivalent action? i.e. under an equivalent Lagrangian?]

We can show that indeed $F = S$. Given the action
\begin{eqn}
S[q(t)] = \int_0^t \dif{t'} \, L(q, \dot q, t'),
\end{eqn}
the variation is just
\begin{eqn}
\delta S = \int_0^t \cancelto{0}{\left( \pd{L}{q} - \od{}{t} \pd{L}{\dot q} \right)} \delta q + \pd{L}{\dot q} \delta q \Big|_0^t 
	= p \Big|_t \delta q \Big|_t
\end{eqn}
so that $\tpd{S}{q|_t} = p |_t$. Furthermore
\begin{eqn}
L = \od{S}{t} = \pd{S}{t} + \pd{S}{q_t} \od{q_t}{t} \big|_t
\end{eqn}
gives us
\begin{eqn}
L(q, \dot q, t) |_t = \pd{S}{t} + p |_t \od{q}{t} \big|_t 
\end{eqn}
so that
\begin{eqn}
\pd{S}{t} = - \left[ p \dot q - L(q, \dot q, t) \right]
\end{eqn}
and thus
\begin{eqn}
\pd{S}{t} = -H(p,q,t) = -H(\pd{S}{q}, q, t),
\end{eqn}
which is just the HJE. 








\end{document}
