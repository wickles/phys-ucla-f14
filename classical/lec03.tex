% declare document class and geometry
\documentclass[12pt]{article} % use larger type; default would be 10pt
\usepackage[margin=1in]{geometry} % handle page geometry

% standard packages
\usepackage{graphicx} % support the \includegraphics command and options
\usepackage{amsmath} % for nice math commands and environments

% font packages
\usepackage{amssymb} % for \mathbb, \mathfrak fonts
\usepackage{mathrsfs} % for \mathscr font
\DeclareMathAlphabet{\mathpzc}{OT1}{pzc}{m}{it} % defines \mathpzc for Zapf Chancery (standard postscript) font

% other packages
\usepackage{datetime} % allows easy formatting of dates, e.g. \formatdate{dd}{mm}{yyyy}
\usepackage{caption} % makes figure captions better, more configurable
\usepackage[squaren]{SIunits} % for nice units formatting e.g. \unit{50}{\kilo\gram}
\usepackage{cancel} % for crossing out terms with \cancel
\usepackage{verbatim} % for verbatim and comment environments
\usepackage{tensor} % for \indices e.g. M\indices{^a_b^{cd}_e}, and \tensor e.g. \tensor[^a_b^c_d]{M}{^a_b^c_d}
\usepackage{feynmp-auto} % for Feynman diagrams. 
\usepackage{pgfplots} % for plotting in tikzpicture environment

% new commands
\newcommand{\fslash}[1]{#1\!\!\!/} % feynman slash
\newcommand{\opname}[1]{\operatorname{#1}} % custom operator names
\newcommand{\pd}{\partial} % partial differential shortcut
\newcommand{\ket}[1]{\left| #1 \right>} % for Dirac kets
\newcommand{\bra}[1]{\left< #1 \right|} % for Dirac bras
\newcommand{\braket}[2]{\left< #1 \vphantom{#2} \right| 
	\left. #2 \vphantom{#1} \right>} % for Dirac brackets

\let\vaccent=\v % rename builtin command \v{} to \vaccent{}
%\renewcommand{\v}[1]{\ensuremath{\mathbf{#1}}} % for vectors
\renewcommand{\v}[1]{\ensuremath{\boldsymbol{\mathbf{#1}}}} % for vectors
%\newcommand{\gv}[1]{\ensurmath{\mbox{\boldmath$ #1 $}}} % for vectors of Greek letters
\newcommand{\uv}[1]{\ensuremath{\boldsymbol{\mathbf{\widehat{#1}}}}} % for unit vectors
\newcommand{\abs}[1]{\left| #1 \right|} % for absolute value ||x||
\newcommand{\norm}[1]{\left\Vert #1 \right\Vert} % for norm ||v||
\newcommand{\avg}[1]{\left< #1 \right>} % for average <x>
\newcommand{\inner}[2]{\left< #1, #2 \right>} % for inner product <x,y>
\newcommand{\set}[1]{ \left\{ #1 \right\} } % for sets {a,b,c,...}

% shortcuts
\newcommand{\reals}{\mathbb{R}} % real numbers
\newcommand{\complexes}{\mathbb{C}} % complex numbers
\newcommand{\nats}{\mathbb{N}} % natural numbers
\newcommand{\irrats}{\mathbb{Q}} % irrationals
\newcommand{\quats}{\mathbb{H}} % quaternions (a la Hamilton)
\newcommand{\euclids}{\mathbb{E}} % Euclidean space
\newcommand{\bigo}{\mathcal{O}} % big O notation





%%%%%%%%%%%%%%%%%%%
% some templates for various things
\begin{comment}

% template for figures
\begin{figure}
\centering
\includegraphics{myfile.png}
\caption{This is a caption}
\label{fig:myfigure}
\end{figure}

% template for Feynman diagrams using feynmf/feynmp
\begin{fmfgraph*}(40,25)
\fmfleft{em,ep}
\fmf{fermion}{em,Zee,ep}
\fmf{photon,label=$Z$}{Zee,Zff}
\fmf{fermion}{fb,Zff,f}
\fmfright{fb,f}
\fmfdot{Zee,Zff}
\end{fmfgraph*}

% template for drawing plots with pgfplot
\pgfplotsset{compat=1.3,compat/path replacement=1.5.1}
\begin{tikzpicture}
\begin{axis}[
extra x ticks={-2,2},
extra y ticks={-2,2},
extra tick style={grid=major}]
\addplot {x};
\draw (axis cs:0,0) circle[radius=2];
\end{axis}
\end{tikzpicture}

\end{comment}
%%%%%%%%%%%%%%%%%%%


\title{Phys 220A -- Classical Mechanics -- Lec03}
\author{UCLA, Fall 2014}
\date{\formatdate{09}{10}{2014}} % Activate to display a given date or no date (if empty),
         % otherwise the current date is printed 

\begin{document}
\setlength{\unitlength}{1mm}
\maketitle


\section{Lagrangian formalism}

The Lagrangian formalism has some advantages over the Newtonian formulation:
\begin{itemize}
\item Looks the same in all coordinate systems
\item Conservation laws are manifestly evident
\item Constraints are simpler to deal with (normal forces, string tensions, etc.)
\item Lagrange equations can be derived from an action principle. This is great because all fundamental laws of physics can be formulated this way (general relativity, Maxwell theory, standard model). 
\item Lagrangians and actions are the starting point for the path integral formulation of quantum mechanics.
\item Leads us to Hamiltonian formulation!
\end{itemize}


\subsection{Action principle}

What is the action principle (or Hamilton's principle, the principle of least action)? Start with a system of generalized coordinates $q_i(t)$ and generalized velocities $\dot{q}_i(t)$, $i = 1, \dots N$. The Lagrangian is assumed to be a function $L(q_i, \dot{q}_i, t)$. Then the action is defined as a functional
\begin{equation}
S[q_i(t)] = \int_{t_1}^{t_2} dt L(q_i, \dot{q}_i, t)
\end{equation}
taking for its argument the path $q_i(t)$ with fixed boundary conditions $q_i(t) = q_i^\text{initial}$ and $q_i(t_2)^\text{final}$. (Recall that a functional $F[q(t)]$ maps a function to a real number.) 

The action principle (the ``principle of least action'') says that the physical path is given by an extremum of the action:
\begin{equation}
\text{Euler-Lagrange:}	\qquad	\frac{\delta S[q]}{\delta q_i(t)} = 0, \quad i = 1, \dots, N. 
\end{equation}
What does this mean? Will need to familiarize ourselves with variational calculus and functional derivatives. 

For simplicity let's consider a single coordinate $q(t)$ with initial and final points $q_1, q_2$. Then we have an action
\begin{equation}
S[q(t)] = \int dt L(q, \dot q, t).
\end{equation}
Recall the ordinary derivative is given by
\begin{equation}
\frac{df}{dx} = \lim_{\epsilon \rightarrow 0} \frac{f(x+\epsilon) - f(x)}{\epsilon} = \lim_{\epsilon \rightarrow 0} \frac{\epsilon \delta f + \bigo(\epsilon^2)}{\epsilon} = \delta f
\end{equation}
We consider an infinitesimal variation of $q(t)$,
\begin{equation}
\tilde{q}(t) = q(t) + \delta q(t) = q(t) + \epsilon h(t) + \bigo(\epsilon^2)
\end{equation}
where $h(t)$ is an arbitrary function with $h(t_1) = h(t_2) = 0$ so that initial and final values of $q$ are fixed and the variation vanishes at the end points: $\delta q(t_1) = \delta q(t_2) = 0$. 

In the variation of the action we only consider terms to order $\epsilon$,
\begin{align}
\delta S &= S[q+\delta q] - S[q] \\
	&= \int_{t_1}^{t_2} L(q(t) + \epsilon h(t), \dot{q}(t) + \epsilon \dot{h}(t), t) - \int_{t_1}^{t_2} dt L(q(t), \dot{q}(t), t) \\
	&= \epsilon \int_{t_1}^{t_2} dt \left\{ \frac{\pd L}{\pd q} h(t) + \frac{\pd L}{\pd \dot q} \right\} + \bigo(\epsilon^2),
\end{align}
and integrating the second term by parts we find
\begin{equation}
\delta S = \epsilon \int_{t_1}^{t_2} \left\{ \frac{\pd L}{\pd q} h - (\frac{d}{dt} \frac{\pd L}{\pd \dot{q}}) h + \frac{d}{dt} (\frac{\pd L}{\pd \dot{q}} h) \right\} + \bigo(\epsilon^2).
\end{equation}
The third term is a total derivative with vanishing boundary conditions so vanishes when integrated, so we're left with
\begin{equation}
\delta S = \epsilon \int_{t_1}^{t_2} dt \left\{ \frac{\pd L}{\pd q} - \frac{d}{dt} \frac{\pd L}{\pd \dot{q}} \right\} h(t) + \bigo(\epsilon^2).
\end{equation}
The action is an extremum if $\delta S = 0$, it's still an integral but the function $h(t)$ is arbitrary. Thus we are left with the Euler-Lagrange equations,
\begin{equation}
\delta S = 0	\qquad \iff \qquad \frac{\pd L}{\pd q} - \frac{d}{dt} \frac{\pd L}{\pd \dot{q}} = 0.
\end{equation}
Note that the extremum is not always a minimum of the action (as the name ``least action principle'' implies). For long trajectories we can find paths that are in fact saddle points of the action. However, it will in fact never be a maximum. This is because we can always take a longer path which increases the action. 

Some comments:
\begin{enumerate}
\item We can easily generalize to more than one coordinate. Taking variations $\delta q_i = q_i + \epsilon h_i(t)$ we have
\begin{equation}
\delta S = \epsilon \int dt \sum_i \left( \frac{\pd L}{\pd q_i} - \frac{d}{dt} \frac{\pd L}{\pd \dot{q}_i} \right) h_i (t) + \bigo(\epsilon^2).
\end{equation}
Since we can just set all $h_j = 0$ for $j \neq i$ we just find the Euler-Lagrange equation applies for each variable,
\begin{equation}
\frac{\pd L}{\pd q_i} - \frac{d}{dt} \frac{\pd L}{\dot{q}_i} = 0, \quad i = 1, \dots, N.
\end{equation}

\item Newton's equation are of course reproduced for conservative forces by choosing $L = T - V$, so for a single particle $L = (1/2)m\dot{x}^2 - V(x)$. Then the Euler-Lagrange equation just reproduces newton's equations
\begin{equation}
-m\ddot{x} - \frac{\pd V}{\pd x} = 0,
\end{equation}
which is 3 dimensions would is $m \ddot{\v{x}} = -\v{\nabla}V(\v{x})$. Similarly for $N$ particles we would have $m_i \ddot{\v{x}}_i = -\v{\nabla}_i V(\v{x})$. 

\item One big advantage of the Euler-Lagrange equations is that they take the same form in any coordinate system. To show this, consider a change of coordinates $q_i \rightarrow r_j$ which is a diffeomorphism, in other words a map
\begin{equation}
r_j = r_j(q_1, \dots, q_N, t), \qquad j = 1, \dots, N
\end{equation}
which is differentiable and has a differentiable inverse. (For example, take $q_i$ to be cartesian coordinates $x, y, z$ and $r_j$ to be spherical coordinates $r, \theta, \phi$.) Then we have the Jacobian $J\indices{_j^i} = \frac{\pd r_j}{\pd q_i}$ which has full rank ($\det J \neq 0$). Note that the Lagrangian should of course be independent of coordinates if it is just the difference $T-V$ between kinetic and potential energy. We can write this as
\begin{equation}
L'(r_j, \dot{r}_j, t) = L(q_i(r_j, \dot{r}_j), \dot{q}_i(r_j, \dot{r}_j), t),
\end{equation}
thus we have 
\begin{equation}
S = \int dt L(q_i, \dot{q}_i, t) = \int dt L'(r_j, \dot{r}_j, t).
\end{equation}
Therefore the condition that the variation $\delta S$ vanishes is entirely geometric so the Euler-Lagrange equations must take the same form $r_j$ coordinates,
\begin{equation}
\frac{\pd L'}{\pd r_j} - \frac{d}{dt} \frac{\pd L'}{\pd \dot{r}_j} = 0.
\end{equation}

We can check this on the level of the Euler-Lagrange equations themselves. Notice that we have by change of coordinates
\begin{equation}
\frac{\pd L}{\pd q_i} - \frac{d}{dt} \frac{\pd L}{\pd \dot{q}_i} = \sum_j \frac{\pd r_j}{\pd q_i} ( \frac{\pd L'}{\pd r_j} - \frac{d}{dt} \frac{\pd L'}{\pd \dot{r}_j} ).
\end{equation}
Writing the term on the right in parentheses as $v_j$ this is just a matrix equation $J\indices{_i^j}v_j = 0$ which since $J$ has full rank implies that $v_j = 0$. 

[He works it out in coordinates a bit more but it's unnecessary given the linear algebraic treatment is enough.]
\end{enumerate}

Now, depending on the coordinate the Euler-Lagrange equation might look simpler or more complicated. For example, for a free particle in 3D we have $L = T = (1/2)m \dot{\v{x}}^2$ with EL $\implies \ddot{x}_i = 0$. On the other hand, in spherical coordinates
\begin{equation}
L = \frac{1}{2} m (\dot{r}^2 + r^2 \dot{\theta}^2 + r^2\sin^2\theta \dot{\phi}^2 )
\end{equation}
and the EL equations give us 
\begin{align}
0 &= r \dot{\theta}^2 + r\sin^2\theta \dot{\phi}^2 \\
0 &= \frac{d}{dt} (r^2 \sin^2\theta \dot{\phi}^2) \\
0 &= r^2 \sin\theta \cos\theta \dot{\phi}^2 - \frac{d}{dt} (r^2 \dot\theta)
\end{align}
which look much worse to work with. The challenge is to find coordinates which make $L$ as simple as possible!

Note also that the derivation allowed for explicitly time dependent coordinate transformations. This implies that the EL equations also have the same form in non-inertial coordinate systems. Some examples:
\begin{enumerate}
\item \textbf{Constant acceleration}
\begin{align}
x &= x' \\
y &= y' \\
z &= z' + \frac{1}{2} a t^2
\end{align}
for the free particles gives us
\begin{align}
L &= \frac{1}{2} m (\dot{x}^2 + \dot{y}^2 + \dot{z}^2) \\
L' &= \frac{1}{2} m \left[ \dot{x}'^2 + \dot{y}'^2 + (\dot{z}' + at)^2 \right] \\
	&= \frac{1}{2} m (\dot{x}'^2 + \dot{y}'^2 + \dot{z}'^2) + mat \cdot \dot{z}' + \frac{1}{2} mat^2.
\end{align}
That last term does not depend on coordinates. Then EL equations give
\begin{equation}
\ddot{x} = 0, \quad \ddot{y} = 0, \quad \ddot{z} + ma = 0
\end{equation}
which is the same as the equations for a constant gravitational field (this is the equivalence principle). 

\item \textbf{Rotating frame}
\begin{align}
x &= x' \cos \omega t + y' \sin \omega t \\
y &= y' \cos \omega t - x' \sin \omega t \\
z &= z',
\end{align}
gives us
\begin{align}
\dot{x} &= \dot{x}' \cos\omega t - \omega x' \sin \omega t + \dot{y}' \sin \omega t + \omega y' \cos \omega t \\
\dot{y} &= \dot{y}' \cos \omega t - y' \omega \sin \omega t - \dot{x}' \sin \omega t - x' \omega \cos \omega t \\
\dot{z} &= \dot{z}'.
\end{align}
Then our Lagrangian is given by
\begin{align}
L &= \frac{1}{2} m (\dot{x}^2 + \dot{y}^2 + \dot{z}^2) \\
	&= \frac{1}{2} m \left[ (\dot{x}' + \omega y')^2 + (\dot{y}' - \omega x')^2 + \dot{z}'^2 \right] \\
	&= \frac{1}{2} m (\dot{\v{x}}' - \v{\omega} \times \v{x}')^2
\end{align}
where $\v{\omega} = (0, 0, \omega)^\top$. We find that the EL equations give us
\begin{equation}
\ddot{\v{x}} - 2 \v{\omega} \times \dot{\v{x}} + \v{\omega} \times \v{\omega} \times \v{x} = 0
\end{equation}
where we recognize the second term as the Coriolis force and the third term as the centrifugal force. We got this from:
\begin{align}
\frac{\pd L}{\pd \v{x}} &= \frac{\pd}{\pd \v{x}} \left( -m \v{x} \cdot (\v{\omega} \times \v{x}) + (m/2) (\v{\omega} \times \v{x})^2 \right) \\
	&= -m \dot{\v{x}} \times \v{\omega} - m (\v{\omega} \times \v{\omega} \times \v{x})
\end{align}
where the first term is obtained by:
\begin{align}
\frac{\pd}{\pd x_m} \dot{\v{x}} \cdot (\v{\omega} \times \v{x}) &= \dot{x}_i \varepsilon_{ijk} \omega_j \delta_{km} \\
	&= \dot{x}_i \omega_j \varepsilon_{ijm} \\
	&= \dot{\v{x}} \times \v{\omega}.
\end{align}

\end{enumerate}



\section{A little more on variational calculus and functional derivatives}

Before, we had  the statement $\delta S / \delta q_i(t) = 0$ as the analog of $df / dx_i = 0$ for extrema. How is this equality defined? For the ordinary derivative we can write $df = f' dx$. For the functional derivative, we have
\begin{align}
\frac{\pd S[q(t) + \epsilon h(t)]}{\pd \epsilon} \Big|_{\epsilon = 0} &= \epsilon \int dt \left( \frac{\pd L}{\pd q} - \frac{d}{dt} \frac{\pd L}{\pd \dot{q}} \right) h(t) \\
	&= \epsilon \int dt \frac{\delta S}{\delta q(t)} h(t)
\end{align}
where $\delta q(t) = \epsilon h(t)$. Then we can write
\begin{equation}
\delta S = \int dt \frac{\delta S}{\delta q(t)} \delta q(t).
\end{equation}
Note the integral! This actually looks different than the ordinary derivative. 

Most rules for derivatives (product, chain, etc) apply with this definition of the functional derivative. Another quick way to define the functional derivative in general is:
\begin{equation}
\frac{\delta S[q]}{\delta q(t')} = \lim_{\epsilon \rightarrow 0} \frac{S[q(t) + \epsilon \delta(t-t')] - \delta q(t)}{\epsilon}.
\end{equation}
For $S = \int dt L(q, \dot q, t)$ this gives us
\begin{align}
\frac{\delta S[q]}{\delta q} &= \lim_{\epsilon \rightarrow 0} \frac{1}{\epsilon} \int dt \left[ L(q(t) + \epsilon \delta(t-t'), \dot{q}(t) + \epsilon \frac{d}{dt} \delta(t-t'), t) - L(q(t), \dot{q}(t), t) \right] \\
	&= \lim_{\epsilon \rightarrow 0} \frac{1}{\epsilon} \int dt \left[ \cancel{L(q, \dot q, t)} + \epsilon \frac{\pd L}{\pd q} \delta(t-t') + \epsilon \frac{\pd L}{\pd \dot{q}} \frac{d}{dt} \delta(t-t') - \cancel{L(q, \dot q, t)} + \bigo(\epsilon^2) \right] \\
	&= \int dt \left[ \frac{\pd L}{\pd q} - \frac{d}{dt} \frac{\pd L}{\pd \dot{q}} \right] \delta(t-t') \\
	&= \frac{\pd L}{\pd q} - \frac{d}{dt} \frac{\pd L}{\pd \dot{q}}.
\end{align}






\end{document}
