% declare document class and geometry
\documentclass[12pt]{article} % use larger type; default would be 10pt
\usepackage[margin=1in]{geometry} % handle page geometry

% standard packages
\usepackage{graphicx} % support the \includegraphics command and options
\usepackage{amsmath} % for nice math commands and environments
\usepackage{mathtools} % extends amsmath with bug fixes and useful commands e.g. \shortintertext
\usepackage{amsthm} % for theorem and proof environments

% font packages
\usepackage{amssymb} % for \mathbb, \mathfrak fonts
\usepackage{mathrsfs} % for \mathscr font
\DeclareMathAlphabet{\mathpzc}{OT1}{pzc}{m}{it} % defines \mathpzc for Zapf Chancery (standard postscript) font

% other packages
\usepackage{datetime} % allows easy formatting of dates, e.g. \formatdate{dd}{mm}{yyyy}
\usepackage{caption} % makes figure captions better, more configurable
\usepackage{enumitem} % allows for custom labels on enumerated lists, e.g. \begin{enumerate}[label=\textbf{(\alph*)}]
\usepackage[squaren]{SIunits} % for nice units formatting e.g. \unit{50}{\kilo\gram}
\usepackage{cancel} % for crossing out terms with \cancel
\usepackage{verbatim} % for verbatim and comment environments
\usepackage{tensor} % for \indices e.g. M\indices{^a_b^{cd}_e}, and \tensor e.g. \tensor[^a_b^c_d]{M}{^a_b^c_d}
\usepackage{feynmp-auto} % for Feynman diagrams. 
\usepackage{pgfplots} % for plotting in tikzpicture environment
\usepackage{commath} % for some nice standardized syntax stuff. \dif, \Dif \od, \pd, \md, \(abs | envert), \(norm | enVert), \(set | cbr), \sbr, \eval, \int(o | c)(o | c), etc
\usepackage{slashed} % provides a command \slashed[1] for Feynman slash notation
%\newcommand{\fslash}[1]{#1\!\!\!/} % feynman slash
%\newcommand{\fsl}[1]{\ensuremath{\mathrlap{\!\not{\phantom{#1}}}#1}}% \fsl{<symbol>}
	% alternative feynman slash

% new commands
\newcommand{\beg}{\begin} % a few letters less for beginning environments
\newenvironment{eqn}{\begin{equation}}{\end{equation}} % a lot fewer letter for equation environment

% rotate stuff
\usepackage{rotating}
	% provides environments for rotating arbitrary objects, e.g. sideways, turn[ang], rotate[ang]
	% also provides macro \turnbox{ang}{stuff}
%\newcommand{\sideways}[1]{\begin{sideways} #1 \end{sideways}} % turn things 90 degrees CCW
%\newcommand{\turn}[2][]{\begin{turn}{#2} #1 \end{turn}} % \turn[ang]{stuff} turns things arbitrary +/- angle

% notational commands
\newcommand{\opname}[1]{\operatorname{#1}} % custom operator names
%\newcommand{\pd}{\partial} % partial differential shortcut
\newcommand{\ket}[1]{\left| #1 \right>} % for Dirac kets
%\newcommand{\ket}[1]{| #1 \rangle}
\newcommand{\bra}[1]{\left< #1 \right|} % for Dirac bras
%\newcommand{\bra}[1]{\langle #1 |}
\newcommand{\braket}[2]{\left< #1 \vphantom{#2} \right| \left. #2 \vphantom{#1} \right>} 
	% for Dirac bra-kets \braket{bra}{ket}
%\newcommand{\braket}[2]{\langle #1 | #2 \rangle} 
\newcommand{\matrixel}[3]{\left< #1 \vphantom{#2#3} \right| #2 \left| #3 \vphantom{#1#2} \right>} 
	% for Dirac matrix elements \matrixel{bra}{op}{ket}
%\newcommand{\matrixel}[3]{\langle #1 | #2 | #3 \rangle} 

%\newcommand{\pd}[2]{\frac{\partial #1}{\partial #2}} % for partial derivatives
%\newcommand{\fd}[2]{\frac{\delta #1}{\delta #2}} % for functional derivatives
\let \vaccent = \v % rename builtin command \v{} to \vaccent{}
%\renewcommand{\v}[1]{\ensuremath{\mathbf{#1}}} % for vectors
\renewcommand{\v}[1]{\ensuremath{\boldsymbol{\mathbf{#1}}}} % for vectors
%\newcommand{\gv}[1]{\ensurmath{\mbox{\boldmath$ #1 $}}} % for vectors of Greek letters
\newcommand{\uv}[1]{\ensuremath{\boldsymbol{\mathbf{\widehat{#1}}}}} % for unit vectors
%\newcommand{\abs}[1]{\left| #1 \right|} % for absolute value ||x||
%\newcommand{\mag}{\abs} % magnitude, just another name for \abs
%\newcommand{\norm}[1]{\left\Vert #1 \right\Vert} % for norm ||v||
\newcommand{\vd}[1]{\v{\dot{#1}}} % for dotted vectors
\newcommand{\vdd}[1]{\v{\ddot{#1}}} % for ddotted vectors
\newcommand{\vddd}[1]{\v{\dddot{#1}}} % for dddotted vectors
\newcommand{\vdddd}[1]{\v{\ddddot{#1}}} % for ddddotted vectors
\newcommand{\avg}[1]{\left< #1 \right>} % for average <x>
\newcommand{\inner}[2]{\left< #1, #2 \right>} % for inner product <x,y>
%\newcommand{\set}[1]{ \left\{ #1 \right\} } % for sets {a,b,c,...}
\newcommand{\tr}{\opname{tr}} % for trace
\newcommand{\Tr}{\opname{Tr}} % for Trace
\newcommand{\rank}{\opname{rank}} % for rank
\let \fancyre = \Re
\let \fancyim = \Im
\newcommand{\Res}{\opname{Res}\limits} % for residue function -- change to put limits on bottom
\renewcommand{\Re}{\opname{Re}}
\renewcommand{\Im}{\opname{Im}}
\renewcommand{\bbar}[1]{\bar{\bar{#1}}} 
	% for barring things twice -- use \cbar or \zbar instead of original \bbar
\newcommand{\bbbar}[1]{\bar{\bbar{#1}}}
\newcommand{\bbbbar}[1]{\bar{\bbbar{#1}}}

\newcommand{\inv}{^{-1}}

% temporary fixes -- commath's versions are bad for powers, like $\dif^3 x$
\renewcommand{\dif}{\mathrm{d}} % \opname{d} better maybe?
\renewcommand{\Dif}{\mathrm{D}}

% notational shortcuts
\newcommand{\bigO}{\mathcal{O}} % big O notation
\let \bigo = \bigO % keep for now, need to update instances in older files
\newcommand{\Lag}{\mathcal{L}} % fancy Lagrangian
\newcommand{\Ham}{\mathcal{H}} % fancy Hamiltonian
\newcommand{\reals}{\mathbb{R}} % real numbers
\newcommand{\complexes}{\mathbb{C}} % complex numbers
\newcommand{\ints}{\mathbb{Z}} % integers
\newcommand{\nats}{\mathbb{N}} % natural numbers
\newcommand{\irrats}{\mathbb{Q}} % irrationals
\newcommand{\quats}{\mathbb{H}} % quaternions (a la Hamilton)
\newcommand{\euclids}{\mathbb{E}} % Euclidean space
\newcommand{\R}{\reals}
\newcommand{\C}{\complexes}
\newcommand{\Z}{\ints}
\newcommand{\Q}{\irrats}
\newcommand{\N}{\nats}
\newcommand{\E}{\euclids}
\newcommand{\RP}{\mathbb{RP}} % real projective space
\newcommand{\CP}{\mathbb{CP}} % complex projective space

% matrix shortcuts!
\newcommand{\pmat}[1]{\begin{pmatrix} #1 \end{pmatrix}}
\newcommand{\bmat}[1]{\begin{bmatrix} #1 \end{bmatrix}}
\newcommand{\Bmat}[1]{\begin{Bmatrix} #1 \end{Bmatrix}}
\newcommand{\vmat}[1]{\begin{vmatrix} #1 \end{vmatrix}}
\newcommand{\Vmat}[1]{\begin{Vmatrix} #1 \end{Vmatrix}}


% more stuff
\newenvironment{enumproblem}{\begin{enumerate}[label=\textbf{(\alph*)}]}{\end{enumerate}}
	% for easily enumerating letters in problems
\newcommand{\grad}[1]{\v{\nabla} #1} % for gradient
\let \divsymb = \div % rename builtin command \div to \divsymb
\renewcommand{\div}[1]{\v{\nabla} \cdot #1} % for divergence
\newcommand{\curl}[1]{\v{\nabla} \times #1} % for curl
\let \baraccent = \= % rename builtin command \= to \baraccent
\renewcommand{\=}[1]{\stackrel{#1}{=}} % for putting numbers above =


% theorem-style environments. note amsthm builtin proof environment: \begin{proof}[title]
% appending [section] resets counter and prepends section number
% use \setcounter{counter}{0} to reset counter
% typical use cases:
% plain: Theorem, Lemma, Corollary, Proposition, Conjecture, Criterion, Algorithm
% definition: Definition, Condition, Problem, Example
% remark: Remark, Note, Notation, Claim, Summary, Acknowledgment, Case, Conclusion
\theoremstyle{plain} % default
\newtheorem{theorem}{Theorem}[section]
\newtheorem{lemma}[theorem]{Lemma}
\newtheorem{corollary}[theorem]{Corollary}
\newtheorem{proposition}[theorem]{Proposition}
\newtheorem{conjecture}[theorem]{Conjecture}
% definition style
\theoremstyle{definition}
\newtheorem{definition}{Definition}
\newtheorem{problem}{Problem}
\newtheorem{exercise}{Exercise}
\newtheorem{example}{Example}
% remark style
\theoremstyle{remark}
\newtheorem{remark}{Remark}
\newtheorem{note}{Note}
\newtheorem{claim}{Claim}
\newtheorem{conclusion}{Conclusion}
% to-do: add problem/subproblem/answer environments for homeworks









%%%%% derivatives


\let \underdot = \d % rename builtin command \d{} to \underdot{}
\let \d = \od % for derivatives

% BUG: derivatives revert to text mode often when in smaller environments in math mode?


% Command for functional derivatives. The first argument denotes the function and the second argument denotes the variable with respect to which the derivative is taken. The optional argument denotes the order of differentiation. The style (text style/display style) is determined automatically
\providecommand{\fd}[3][]{\ensuremath{
\ifinner
\tfrac{\delta{^{#1}}#2}{\delta{#3^{#1}}}
\else
\dfrac{\delta{^{#1}}#2}{\delta{#3^{#1}}}
\fi
}}

% \tfd[2]{f}{k} denotes the second functional derivative of f with respect to k
% The first letter t means "text style"
\providecommand{\tfd}[3][]{\ensuremath{\mathinner{
\tfrac{\delta{^{#1}}#2}{\delta{#3^{#1}}}
}}}
% \dfd[2]{f}{k} denotes the second functional derivative of f with respect to k
% The first letter d means "display style"
\providecommand{\dfd}[3][]{\ensuremath{\mathinner{
\dfrac{\delta{^{#1}}#2}{\delta{#3^{#1}}}
}}}

% mixed functional derivative - analogous to the functional derivative command
% \mfd{F}{5}{x}{2}{y}{3}
\providecommand{\mfd}[6]{\ensuremath{
\ifinner
\tfrac{\delta{^{#2}}#1}{\delta{#3^{#4}}\delta{#5^{#6}}}
\else
\dfrac{\delta{^{#2}}#1}{\delta{#3^{#4}}\delta{#5^{#6}}}
\fi
}}


% Command for thermodynamic (chemistry?) partial derivatives. The first argument denotes the function and the second argument denotes the variable with respect to which the derivative is taken. The optional argument denotes the order of differentiation. The style (text style/display style) is determined automatically
\providecommand{\pdc}[4][]{\ensuremath{
\ifinner
\left( \tfrac{\partial{^{#1}}#2}{\partial{#3^{#1}}} \right)_{#4}
\else
\left( \dfrac{\partial{^{#1}}#2}{\partial{#3^{#1}}} \right)_{#4}
\fi
}}

% \tpd[2]{f}{k} denotes the second thermo partial derivative of f with respect to k
% The first letter t means "text style"
\providecommand{\tpdc}[4][]{\ensuremath{\mathinner{
\left( \tfrac{\partial{^{#1}}#2}{\partial{#3^{#1}}} \right)_{#4}
}}}
% \dpd[2]{f}{k} denotes the second thermo partial derivative of f with respect to k
% The first letter d means "display style"
\providecommand{\dpdc}[4][]{\ensuremath{\mathinner{
\left( \dfrac{\partial{^{#1}}#2}{\partial{#3^{#1}}} \right)_{#4}
}}}


%%%%%%





%%%%%%%%%%%%%%%%%%%
% some templates for various things
\begin{comment}

% template for figures
\begin{figure}
\centering
\includegraphics{myfile.png}
\caption{This is a caption}
\label{fig:myfigure}
\end{figure}

% template for Feynman diagrams using feynmf/feynmp
\begin{fmfgraph*}(40,25)
\fmfleft{em,ep}
\fmf{fermion}{em,Zee,ep}
\fmf{photon,label=$Z$}{Zee,Zff}
\fmf{fermion}{fb,Zff,f}
\fmfright{fb,f}
\fmfdot{Zee,Zff}
\end{fmfgraph*}

% template for drawing plots with pgfplot
\pgfplotsset{compat=1.3,compat/path replacement=1.5.1}
\begin{tikzpicture}
\begin{axis}[
extra x ticks={-2,2},
extra y ticks={-2,2},
extra tick style={grid=major}]
\addplot {x};
\draw (axis cs:0,0) circle[radius=2];
\end{axis}
\end{tikzpicture}

%% find package for easily drawing mapping / algebraic / commutative diagrams..

\end{comment}
%%%%%%%%%%%%%%%%%%%



%%%%% A note on spacing
% 5) \qquad
% 4) \quad
% 3) \thickspace = \;
% 2) \medspace = \:
% 1) \thinspace = \,
% -1) \negthinspace = \!
% -2) \negmedspace
% -3) \negthickspace




\title{Phys 220A -- Classical Mechanics -- Lec02}
\author{UCLA, Fall 2014}
\date{\formatdate{07}{10}{2014}} % Activate to display a given date or no date (if empty),
         % otherwise the current date is printed 

\begin{document}
\setlength{\unitlength}{1mm}
\maketitle


\section{Introduction}

A couple of general comments about Newton's laws. 

\subsection{Something funny about Lorentz force and magnetic fields}

Consider the Lorentz force between two charged particles, given by $\v{F}_{12} = q \v{v}_1 \times \v{B}_2$. Consider particle 1 to lie on the positive $y$ axis with velocity in the positive $y$ direction and particle 2 to lie on the positive $z$ axis with velocity in the positive $x$ axis. Then we find that $\v{F}_{12} \neq 0$ points in the positive $x$ direction yet $\v{F}_{21} = 0$. 

What's going on here? Is this a case of violation of angular momentum conservation? No, in fact the EM field itself carries angular momentum. 


\subsection{Virial Theorem (useful in stat. mech)}

Take some function $f$ and write the time average
\begin{equation}
\avg{f} = \lim_{T \rightarrow \infty} \frac{1}{T} \int_{t_0}^{t_0 + T} f(t) dt.
\end{equation}
Note that there is no dependence on $t_0$ for a bounded quantity,
\begin{equation}
\pd{}{t_0} \avg{f_0} = \lim_{T \rightarrow \infty} \frac{f(t_0 + T) - f(t_0)}{T} = 0
\end{equation}
which vanishes because the numerator in the limit is bounded. 

Next, we hope to obtain an expression for $\avg{T}$ for conservative forces. Notice that 
\begin{equation}
\sum_i m_i \ddot{\v{x}}_i \cdot \v{x} = \sum_i \v{F}_i \cdot \v{x}_i,
\end{equation}
which we can rewrite as 
\begin{equation}
\frac{d}{dt} \left( \sum_i m_i \dot{\v{x}}_i \cdot \v{x} \right) - \underbrace{\sum_i m_i \dot{\v{x}}_i^2}_{2T} = - \sum_i \v{\nabla}_i V \cdot \v{x}_i.
\end{equation}
If the motion is bounded then the derivative goes away when we average, so averaging over time we find
\begin{equation}
2 \avg{T} = \avg{ \sum_i \v{\nabla}_i V \cdot \v{x}_i }.
\end{equation}
This is the Virial theorem. It is especially useful if $V$ is homogeneous, i.e. 
\begin{equation}
V(\lambda \v{x}_1, \dots \lambda \v{x}_n) = \lambda^k V(\v{x}_1, \dots, \v{x}_n)
\end{equation}
for some constant $k$ and any value of $\lambda$. Taking the derivative of this equation w.r.t. $\lambda$ we have
\begin{equation}
\sum_i \v{\nabla}_i V(\lambda \v{x}_1, \dots, \lambda \v{x}_n) \cdot \v{x}_i = k \lambda^{k-1} V(\v{x}_1, \dots, \v{x}_n)
\end{equation}
and setting $\lambda = 1$ we find $\sum_i \v{\nabla} V \cdot \v{x}_i = k V$. Thus the Virial theorem for a homogeneous potential gives us
\begin{equation}
2 \avg{T} = k \avg{V}.
\end{equation}

Let's look at some simple systems. For the harmonic oscillator
\begin{equation}
V = \frac{1}{2} kx^2
\end{equation}
So our parameter from above $k$ is 2, so virial theorem for this system is:
\begin{equation}
\avg{T} = \avg{V}
\end{equation}
For a Keplerian potential it's:
\begin{equation}
V = \frac{\alpha}{r}
\end{equation}
So $k = 1$ and Virial theorem is
\begin{equation}
\avg{T} = -\frac{1}{2} \avg{V}
\end{equation}
\textbf{Note that Virial theorem in general only holds for bound motion with no scattering}


\section{Simple systems}
\subsection{1-D Harmonic Oscillator}
Let's look at the solutions to some systems

Hookes law:
\begin{equation}
F = -kx 
\end{equation}
\begin{equation}
m\ddot{x} = -kx
\end{equation}
\begin{equation}
\ddot{x} + \frac{k}{m} x = 0
\end{equation}
This system has a very general ansatz where $\omega = k / m$
\begin{equation}
x(t) = c_1 \sin\omega t + c_2 \cos\omega t
\end{equation}
\begin{equation}
x(t) = c\sin(\omega t + \phi_0)
\end{equation}
\begin{equation}
c_1 e^{i\omega t} + c_2^* e^{-i \omega t}
\end{equation}
There are 2 integration constants that determine $x$ and $\dot{x}$ at t = 0. We can also write the momentum of this system
\begin{equation}
p = m\dot{x} = mc_1 \omega \cos(\omega t + \phi_0)
\end{equation}
\begin{equation}
p^2 = m^2 \omega^2 (c_1^2 - x^2)
\end{equation}
\begin{equation}
p^2 + \omega x^2 = m^2 \omega^2
\end{equation}
Looking at this in phase space, we get closed circles because this is a conservative field. The origin is the equilibrium point and the circles do not intersect. This changes if there is friction, and for driven harmonic oscillator

\subsection{Particle in Constant Magnetic Field}
Let's set up a constant magnetic field in the $\uv{z}$ direction. Then our forces are 
\begin{equation}
\v{F} = e\v{v} \times \v{B}, \quad \text{or} \quad
F_i = e \epsilon_{ijk} v_j B_k,
\end{equation}
which gives us equations of motion
\begin{eqn}
m\ddot{z} = 0 \quad \implies \quad z = z_0 + v_z t,
\end{eqn}
and
\begin{eqn}
m\ddot{x} = B e \dot{y}, \qquad
m\ddot{y} = -B e\dot{x}.
\end{eqn}
This is a system of two first order differential equations. Defining
\begin{equation}
\omega_B = \frac{eB}{m}, \quad 
S_1 = \dot x, \quad
S_2 = \dot y,
\end{equation}
our equations become
\begin{align}
\dot{S_1} &= \omega_B S_2, \\
\dot{S_2} &= -\omega_B S_1.
\end{align}
Written in vector form, this becomes
\begin{equation}
\d{}{t} \pmat{ S_1 \\ S_2 } = M \pmat{ S_1 \\ S_2 }
\end{equation}
where $M$ is a matrix
\begin{equation}
M = \pmat{ 0 & \omega_B \\ -\omega_B & 0 }
\end{equation}
The solution of course is given by exponentiation
\begin{eqn}
\pmat{ S_1(t) \\ S_2(t) } = R(t) \pmat{ S_1(0) \\ S_2(0) }
\end{eqn}
where
\begin{eqn}
R(t) = \exp (Mt) = 
\begin{pmatrix}
\cos \omega_B t & \sin \omega_B t \\
-\sin \omega_B t & \cos \omega_B t
\end{pmatrix}.
\end{eqn}
Then $x,y$ can be obtained by integration
\begin{align}
x(t) &= \int S_1(t) \dif{t} + x_0, \\
y(t) &= \int S_2(t) \dif{t} + y_0,
\end{align}
which gives us
\begin{align}
x(t) &= \frac{S_1(0)}{\omega_B} \sin \omega_B t - \frac{S_2(0)}{\omega_B} \cos \omega_B t + x_0, \\
y(t) &= \frac{S_1(0)}{\omega_B} \cos \omega_B t + \frac{S_2(0)}{\omega_B} \sin \omega_B t + y_0.
\end{align}
This motion can be described as a drift in the $\uv{z}$ direction with constant velocity and circular motion in the $xy$-plane with radius $R$ and angular frequency $\omega_B = eB/m$, the Larmor frequency. 


\section{Motion in a central potential}

Next we will study motion in a central potential. This is where the potential is a function only of the radial coordinate $r$, most important being the Kepler/Coulomb potential. In the Kepler case it describes the motion of two masses $m_1, m_2$ in their gravitational field. 

The first step is to transform to the center of mass and relative coordinate
\begin{eqn}
\v{X} = \frac{m_1 \v x_1 + m_2 \v x_2}{m_1 + m_2}, \qquad
\v R = \v x_1 - \v x_2.
\end{eqn}
This results in decoupled coordinates: free motion of the center of mass and a central force equation for $R$,
\begin{eqn}
\vdd{X} = 0, \qquad
\mu \vdd{R} = -\v{\nabla} V(R).
\end{eqn}
Some special cases are as follows.


\subsection{Special cases}

\begin{example}[Spherical harmonic oscillator]
$V(r) = \alpha r^2$
\end{example}

\begin{example}[Gravitational/Kepler potential]
$V(r) = - \beta / r$
\end{example}

In general, $V(r)$ can be a complicated function.

\begin{example}[Yukawa potential]
$V(r) = -\frac{\gamma}{r} e^{-r / r_0}$
\end{example}

\begin{example}[Lennard-Jones potential]
$V(r) = \epsilon \left[ (\frac{r_m}{r})^{12} - 2 (\frac{r_m}{r})^6 \right]$, where the first term is a repulsive (Pauli) force, and the second term is an attractive Van der Waals force. 
\end{example}

As we shall see, apart from being simpler than other potentials, the harmonic oscillator and Kepler potentials are special since they allow for closed bounded orbits. 


\subsection{Angular momentum conservation}

In general, angular momentum $\v L = \v x \times \v p$ is conserved for central potentials:
\begin{align}
\vd L &= \vd x \times \v p + \v x \times \vd p \\
	&= \frac{1}{m} \cancel{\v p \times \v p} - \v x \times \v \nabla V \\
	&= 0
\end{align}
since $\v \nabla V \sim \v x$ for central potentials. By choice of coordinates we can choose $\v L = L \uv z$, which gives us motion solely in the $xy$-plane. Note that angular momentum conservation is Kepler's area law---so it's not limited to $1/r$ potentials. 


\subsection{Cylindrical coordinates}

Here we will go to cylindrical coordinates $(x,y,z) \rightarrow (\rho, \phi, z)$, where we have
\begin{eqn}
x = r \cos\phi, \qquad
y = r \sin\phi, \qquad
z = z
\end{eqn}
and velocity
\begin{eqn}
\vd x = \dot r \uv r + r \dot \phi \uv \phi + \cancel{\dot z \uv z}
\end{eqn}
since
\begin{eqn}
\uv r = \frac{1}{r} \pmat{x \\ y}, \qquad
\uv \phi = \frac{1}{r} \pmat{-y \\ x}.
\end{eqn}
So for angular momentum we have
\begin{align}
\v L &= r \uv r \times m (\dot r \uv r + r \dot \phi \uv \phi) \\
	&= m r^2 \dot \phi \uv z
\end{align}
so that $L = m r^2 \dot \phi$ is constant. An application of Kepler's law is that comets have very eccentric orbits and hence spend most of the time in the outer reaches of the solar system. 


\subsection{Integration of Newton's equation}

Note that
\begin{eqn}
\uv{\dot r} = \dot \phi \uv \phi, \qquad
\uv{\dot \phi} = - \dot \phi \uv r,
\end{eqn}
Furthermore, we have the acceleration
\begin{eqn}
\vdd x = (\ddot r - r \dot \phi^2) \uv r + (2 \dot r \dot \phi + r \ddot \phi) \uv \phi + \cancel{\ddot z \uv z},
\end{eqn}
so that our potential equation
\begin{eqn}
m \vdd x = - \v \nabla V(r) = - \pd{V}{r} \uv r
\end{eqn}
can be written as two equations,
\begin{eqn}
m (\ddot r - r \dot \phi^2) = - \pd{V}{r}
\end{eqn}
and
\begin{eqn}
0 = 2 \dot r \dot \phi + r \ddot \phi = \frac{1}{r} \od{}{t} m r^2 \dot \phi.
\end{eqn}
Since $\dot \phi = L^2 / m r^2$ the second equation is just angular momentum conservaion, while for the first equation we obtain a new equation of motion just in $r$,
\begin{eqn}
m \ddot r = -\pd{V}{r} + \frac{L^2}{mr^3}
\end{eqn}
which we can rewrite as
\begin{eqn}
m \ddot r = - \pd{V_\text{eff}}{r}
\end{eqn}
where
\begin{eqn}
V_\text{eff}(r) = V(r) + \frac{L^2}{2mr^2}.
\end{eqn}

Our new equation of motion is 2nd order and nonlinear---would be very difficult to solve explicitly directly from here. But $V(r)$ is a potential for a conservative force, so we have energy conservation
\begin{align}
E &= \frac{1}{2} m \dot r^2 + \frac{1}{2} m r^2 \dot \phi^2 + V(r) \\
	&= \frac{1}{2} m \dot r^2 + V_\text{eff}(r).
\end{align}
One can check that $\od{E}{t} = 0$ using the equation of motion. So instead of solving the 2nd order equation explicitly, we can use energy conservation to reduce the problem to a first order equation which can be solved by integration,
\begin{eqn}
\dot r = \pm \sqrt{ \frac{2}{m} (E - V_\text{eff}(r))},
\end{eqn}
which gives us
\begin{eqn}
\pm (t - t_0) = \int \frac{\dif r}{\sqrt{\frac{2}{m} (E - V_\text{eff}(r))}}.
\end{eqn}
So we have ``reduced to quadrature'', i.e. a single integral. Depending on the form of $V(r)$ this integral may be elementary (readily soluble) or not. Some comments are in order:
\begin{itemize}
\item $\pm (t-t_0) = f(r)$ implies that we need to find $r(t) = f\inv (\pm (t-t_0))$, the functional inverse. 
\item Can also find $\phi(t)$ via integration of $\dot \phi = L / mr^2$,
\begin{eqn}
\phi - \phi_0 = \int \frac{L}{m r(t)^2} \dif{t}.
\end{eqn}
\item The effective potential $V_\text{eff}(r)$ generally comes down from infinity as $1/r^2$ for small $r$ and then goes as $V(r)$ as $r \rightarrow \infty$. For the Kepler potential this gives us a dip beneath the zero which then comes back up and approaches zero as $r \rightarrow \infty$, so that we have unbounded orbits for $E>0$ but bounded orbits for $E < 0$ between $r_\text{min}$ and $r_\text{man}$ where $V_\text{eff}(r_\text{min}) = V_\text{eff}(r_\text{max}) = E$.
\item Instead of finding $r(t)$ we can determine $r(\phi)$ or $\phi(r)$ to obtain the orbit. Using
\begin{eqn}
\od{\phi}{t} = \frac{L}{mr^2}
\end{eqn}
we find
\begin{eqn}
\frac{\dif r}{\sqrt{\frac{2}{m} (E-V_\text{eff})}} = \pm \dif t = \pm \frac{mr^2}{L} \dif \phi,
\end{eqn}
which gives us a solution
\begin{eqn}
\pm (\phi - \phi_0) = \int \dif{r} \frac{L / mr^2}{\sqrt{\frac{2}{m} (E - V_\text{eff})}}.
\end{eqn}
\end{itemize}

\subsection{Bertrand's theorem}

We will state this theorem without proof. It states that the only potentials for which all bounded orbits are closed are the Kepler potential and the radial harmonic oscillator
\begin{eqn}
V_1(r) = - k / r, \qquad
V_2(r) = \frac{1}{2} m \omega^2 r^2.
\end{eqn}







\end{document}
