% declare document class and geometry
\documentclass[12pt]{article} % use larger type; default would be 10pt
\usepackage[margin=1in]{geometry} % handle page geometry

% import packages and commands
\input{../header2.tex}


\title{Phys 220A -- Classical Mechanics -- HW08}
\author{UCLA, Fall 2014}
\date{\formatdate{2}{12}{2014}} % Activate to display a given date or no date (if empty),
	% otherwise the current date is printed 
	% format: formatdate{dd}{mm}{yyyy}

\begin{document}
\maketitle


\section*{Problem 1 (15 pts)}
\begin{em}
Find the small oscillation frequencies of a real pendulum suspended at a fixed point, in the presence of a constant uniform gravitational field with acceleration $g$. The mass of the pendulum is denoted by $m$, its center of mass is a distance $\ell$ away from the point of suspension, and its three moments of inertia $I_x$, $I_y$, and $I_z$. 
\end{em}


\section*{Problem 2 (15 pts)}
\begin{em}
The Lagrangian for a heavy symmetric top is given by
\begin{eqn}
L = \frac{1}{2} I_1 (\dot \theta^2 + \dot \phi^2 \sin^2 \theta) + \frac{1}{2} I_3 (\dot \psi + \dot \phi \cos \theta)^2 - mgl \cos \theta.
\end{eqn}
\end{em}

\begin{enumproblem}

% part A
\item \begin{em}
Obtain the momenta $p_\theta, p_\phi, p_\psi$.
\end{em}


% part B
\item \begin{em}
Calculate the Hamiltonian
\end{em}


% part C
\item \begin{em}
Calculate Hamilton's equations.
\end{em}


\end{enumproblem}



\section*{Problem 3 (15 pts)}
\begin{em}
A rigid body has inertia tensor $\Theta_{ij}$ for a body coordinate system which goes through the center of mass of the body. Calculate the inertia tensor for a coordinate system which is translated by a vector $\v a$. (This result is sometimes called the Huygens-Steiner parallel axis theorem).
\end{em}



\section*{Problem 4 (15 pts)}
\begin{em}
A torsion pendulum consists of a vertical wire attached to a mass which may rotate about the vertical direction. Consider three torsion pendulums which consist of identical wires from which identical homogeneous solid cubes are suspended. One cube is suspended from a vertex, one from the center of an edge, and one from the center of a face (see figure 1). What are the ratios of the periods of the three pendulums ?
\end{em}



\section*{Problem 5 (20 pts)}
\begin{em}
Consider a heavy symmetric top with mass $m$ which is anchored at a point which has a distance $l$ from the center of mass. The moments of inertia (around the anchor point) are $I_1$, $I_2$ and $I_3$ (See figure 2 for coordinates)
\end{em}

\begin{enumproblem}

% part A
\item \begin{em}
For the motion with initial conditions $\theta = \theta_0$ and $\dot \phi = 0$. Derive the equation of motion for the coordinate $\theta$ alone.
\end{em}


% part B
\item \begin{em}
For a very fast spinning top which satisfies $I_3 \omega_3 \gg \sqrt{mgl I_1}$ show that the motion of the top is nutation with frequency 
\begin{eqn}
\Omega \sim \frac{\omega_3 I_3}{I_1}.
\end{eqn}
\end{em}


\end{enumproblem}




\end{document}
