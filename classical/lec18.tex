% declare document class and geometry
\documentclass[12pt]{article} % use larger type; default would be 10pt
\usepackage[margin=1in]{geometry} % handle page geometry

% standard packages
\usepackage{graphicx} % support the \includegraphics command and options
\usepackage{amsmath} % for nice math commands and environments
\usepackage{mathtools} % extends amsmath with bug fixes and useful commands e.g. \shortintertext
\usepackage{amsthm} % for theorem and proof environments

% font packages
\usepackage{amssymb} % for \mathbb, \mathfrak fonts
\usepackage{mathrsfs} % for \mathscr font
\DeclareMathAlphabet{\mathpzc}{OT1}{pzc}{m}{it} % defines \mathpzc for Zapf Chancery (standard postscript) font

% other packages
\usepackage{datetime} % allows easy formatting of dates, e.g. \formatdate{dd}{mm}{yyyy}
\usepackage{caption} % makes figure captions better, more configurable
\usepackage{enumitem} % allows for custom labels on enumerated lists, e.g. \begin{enumerate}[label=\textbf{(\alph*)}]
\usepackage[squaren]{SIunits} % for nice units formatting e.g. \unit{50}{\kilo\gram}
\usepackage{cancel} % for crossing out terms with \cancel
\usepackage{verbatim} % for verbatim and comment environments
\usepackage{tensor} % for \indices e.g. M\indices{^a_b^{cd}_e}, and \tensor e.g. \tensor[^a_b^c_d]{M}{^a_b^c_d}
\usepackage{feynmp-auto} % for Feynman diagrams. 
\usepackage{pgfplots} % for plotting in tikzpicture environment
\usepackage{commath} % for some nice standardized syntax stuff. \dif, \Dif \od, \pd, \md, \(abs | envert), \(norm | enVert), \(set | cbr), \sbr, \eval, \int(o | c)(o | c), etc
\usepackage{slashed} % provides a command \slashed[1] for Feynman slash notation
%\newcommand{\fslash}[1]{#1\!\!\!/} % feynman slash
%\newcommand{\fsl}[1]{\ensuremath{\mathrlap{\!\not{\phantom{#1}}}#1}}% \fsl{<symbol>}
	% alternative feynman slash

% new commands
\newcommand{\beg}{\begin} % a few letters less for beginning environments
\newenvironment{eqn}{\begin{equation}}{\end{equation}} % a lot fewer letter for equation environment

% rotate stuff
\usepackage{rotating}
	% provides environments for rotating arbitrary objects, e.g. sideways, turn[ang], rotate[ang]
	% also provides macro \turnbox{ang}{stuff}
%\newcommand{\sideways}[1]{\begin{sideways} #1 \end{sideways}} % turn things 90 degrees CCW
%\newcommand{\turn}[2][]{\begin{turn}{#2} #1 \end{turn}} % \turn[ang]{stuff} turns things arbitrary +/- angle

% notational commands
\newcommand{\opname}[1]{\operatorname{#1}} % custom operator names
%\newcommand{\pd}{\partial} % partial differential shortcut
\newcommand{\ket}[1]{\left| #1 \right>} % for Dirac kets
%\newcommand{\ket}[1]{| #1 \rangle}
\newcommand{\bra}[1]{\left< #1 \right|} % for Dirac bras
%\newcommand{\bra}[1]{\langle #1 |}
\newcommand{\braket}[2]{\left< #1 \vphantom{#2} \right| \left. #2 \vphantom{#1} \right>} 
	% for Dirac bra-kets \braket{bra}{ket}
%\newcommand{\braket}[2]{\langle #1 | #2 \rangle} 
\newcommand{\matrixel}[3]{\left< #1 \vphantom{#2#3} \right| #2 \left| #3 \vphantom{#1#2} \right>} 
	% for Dirac matrix elements \matrixel{bra}{op}{ket}
%\newcommand{\matrixel}[3]{\langle #1 | #2 | #3 \rangle} 

%\newcommand{\pd}[2]{\frac{\partial #1}{\partial #2}} % for partial derivatives
%\newcommand{\fd}[2]{\frac{\delta #1}{\delta #2}} % for functional derivatives
\let \vaccent = \v % rename builtin command \v{} to \vaccent{}
%\renewcommand{\v}[1]{\ensuremath{\mathbf{#1}}} % for vectors
\renewcommand{\v}[1]{\ensuremath{\boldsymbol{\mathbf{#1}}}} % for vectors
%\newcommand{\gv}[1]{\ensurmath{\mbox{\boldmath$ #1 $}}} % for vectors of Greek letters
\newcommand{\uv}[1]{\ensuremath{\boldsymbol{\mathbf{\widehat{#1}}}}} % for unit vectors
%\newcommand{\abs}[1]{\left| #1 \right|} % for absolute value ||x||
%\newcommand{\mag}{\abs} % magnitude, just another name for \abs
%\newcommand{\norm}[1]{\left\Vert #1 \right\Vert} % for norm ||v||
\newcommand{\vd}[1]{\v{\dot{#1}}} % for dotted vectors
\newcommand{\vdd}[1]{\v{\ddot{#1}}} % for ddotted vectors
\newcommand{\vddd}[1]{\v{\dddot{#1}}} % for dddotted vectors
\newcommand{\vdddd}[1]{\v{\ddddot{#1}}} % for ddddotted vectors
\newcommand{\avg}[1]{\left< #1 \right>} % for average <x>
\newcommand{\inner}[2]{\left< #1, #2 \right>} % for inner product <x,y>
%\newcommand{\set}[1]{ \left\{ #1 \right\} } % for sets {a,b,c,...}
\newcommand{\tr}{\opname{tr}} % for trace
\newcommand{\Tr}{\opname{Tr}} % for Trace
\newcommand{\rank}{\opname{rank}} % for rank
\let \fancyre = \Re
\let \fancyim = \Im
\newcommand{\Res}{\opname{Res}\limits} % for residue function -- change to put limits on bottom
\renewcommand{\Re}{\opname{Re}}
\renewcommand{\Im}{\opname{Im}}
\renewcommand{\bbar}[1]{\bar{\bar{#1}}} 
	% for barring things twice -- use \cbar or \zbar instead of original \bbar
\newcommand{\bbbar}[1]{\bar{\bbar{#1}}}
\newcommand{\bbbbar}[1]{\bar{\bbbar{#1}}}

\newcommand{\inv}{^{-1}}

% temporary fixes -- commath's versions are bad for powers, like $\dif^3 x$
\renewcommand{\dif}{\mathrm{d}} % \opname{d} better maybe?
\renewcommand{\Dif}{\mathrm{D}}

% notational shortcuts
\newcommand{\bigO}{\mathcal{O}} % big O notation
\let \bigo = \bigO % keep for now, need to update instances in older files
\newcommand{\Lag}{\mathcal{L}} % fancy Lagrangian
\newcommand{\Ham}{\mathcal{H}} % fancy Hamiltonian
\newcommand{\reals}{\mathbb{R}} % real numbers
\newcommand{\complexes}{\mathbb{C}} % complex numbers
\newcommand{\ints}{\mathbb{Z}} % integers
\newcommand{\nats}{\mathbb{N}} % natural numbers
\newcommand{\irrats}{\mathbb{Q}} % irrationals
\newcommand{\quats}{\mathbb{H}} % quaternions (a la Hamilton)
\newcommand{\euclids}{\mathbb{E}} % Euclidean space
\newcommand{\R}{\reals}
\newcommand{\C}{\complexes}
\newcommand{\Z}{\ints}
\newcommand{\Q}{\irrats}
\newcommand{\N}{\nats}
\newcommand{\E}{\euclids}
\newcommand{\RP}{\mathbb{RP}} % real projective space
\newcommand{\CP}{\mathbb{CP}} % complex projective space

% matrix shortcuts!
\newcommand{\pmat}[1]{\begin{pmatrix} #1 \end{pmatrix}}
\newcommand{\bmat}[1]{\begin{bmatrix} #1 \end{bmatrix}}
\newcommand{\Bmat}[1]{\begin{Bmatrix} #1 \end{Bmatrix}}
\newcommand{\vmat}[1]{\begin{vmatrix} #1 \end{vmatrix}}
\newcommand{\Vmat}[1]{\begin{Vmatrix} #1 \end{Vmatrix}}


% more stuff
\newenvironment{enumproblem}{\begin{enumerate}[label=\textbf{(\alph*)}]}{\end{enumerate}}
	% for easily enumerating letters in problems
\newcommand{\grad}[1]{\v{\nabla} #1} % for gradient
\let \divsymb = \div % rename builtin command \div to \divsymb
\renewcommand{\div}[1]{\v{\nabla} \cdot #1} % for divergence
\newcommand{\curl}[1]{\v{\nabla} \times #1} % for curl
\let \baraccent = \= % rename builtin command \= to \baraccent
\renewcommand{\=}[1]{\stackrel{#1}{=}} % for putting numbers above =


% theorem-style environments. note amsthm builtin proof environment: \begin{proof}[title]
% appending [section] resets counter and prepends section number
% use \setcounter{counter}{0} to reset counter
% typical use cases:
% plain: Theorem, Lemma, Corollary, Proposition, Conjecture, Criterion, Algorithm
% definition: Definition, Condition, Problem, Example
% remark: Remark, Note, Notation, Claim, Summary, Acknowledgment, Case, Conclusion
\theoremstyle{plain} % default
\newtheorem{theorem}{Theorem}[section]
\newtheorem{lemma}[theorem]{Lemma}
\newtheorem{corollary}[theorem]{Corollary}
\newtheorem{proposition}[theorem]{Proposition}
\newtheorem{conjecture}[theorem]{Conjecture}
% definition style
\theoremstyle{definition}
\newtheorem{definition}{Definition}
\newtheorem{problem}{Problem}
\newtheorem{exercise}{Exercise}
\newtheorem{example}{Example}
% remark style
\theoremstyle{remark}
\newtheorem{remark}{Remark}
\newtheorem{note}{Note}
\newtheorem{claim}{Claim}
\newtheorem{conclusion}{Conclusion}
% to-do: add problem/subproblem/answer environments for homeworks









%%%%% derivatives


\let \underdot = \d % rename builtin command \d{} to \underdot{}
\let \d = \od % for derivatives

% BUG: derivatives revert to text mode often when in smaller environments in math mode?


% Command for functional derivatives. The first argument denotes the function and the second argument denotes the variable with respect to which the derivative is taken. The optional argument denotes the order of differentiation. The style (text style/display style) is determined automatically
\providecommand{\fd}[3][]{\ensuremath{
\ifinner
\tfrac{\delta{^{#1}}#2}{\delta{#3^{#1}}}
\else
\dfrac{\delta{^{#1}}#2}{\delta{#3^{#1}}}
\fi
}}

% \tfd[2]{f}{k} denotes the second functional derivative of f with respect to k
% The first letter t means "text style"
\providecommand{\tfd}[3][]{\ensuremath{\mathinner{
\tfrac{\delta{^{#1}}#2}{\delta{#3^{#1}}}
}}}
% \dfd[2]{f}{k} denotes the second functional derivative of f with respect to k
% The first letter d means "display style"
\providecommand{\dfd}[3][]{\ensuremath{\mathinner{
\dfrac{\delta{^{#1}}#2}{\delta{#3^{#1}}}
}}}

% mixed functional derivative - analogous to the functional derivative command
% \mfd{F}{5}{x}{2}{y}{3}
\providecommand{\mfd}[6]{\ensuremath{
\ifinner
\tfrac{\delta{^{#2}}#1}{\delta{#3^{#4}}\delta{#5^{#6}}}
\else
\dfrac{\delta{^{#2}}#1}{\delta{#3^{#4}}\delta{#5^{#6}}}
\fi
}}


% Command for thermodynamic (chemistry?) partial derivatives. The first argument denotes the function and the second argument denotes the variable with respect to which the derivative is taken. The optional argument denotes the order of differentiation. The style (text style/display style) is determined automatically
\providecommand{\pdc}[4][]{\ensuremath{
\ifinner
\left( \tfrac{\partial{^{#1}}#2}{\partial{#3^{#1}}} \right)_{#4}
\else
\left( \dfrac{\partial{^{#1}}#2}{\partial{#3^{#1}}} \right)_{#4}
\fi
}}

% \tpd[2]{f}{k} denotes the second thermo partial derivative of f with respect to k
% The first letter t means "text style"
\providecommand{\tpdc}[4][]{\ensuremath{\mathinner{
\left( \tfrac{\partial{^{#1}}#2}{\partial{#3^{#1}}} \right)_{#4}
}}}
% \dpd[2]{f}{k} denotes the second thermo partial derivative of f with respect to k
% The first letter d means "display style"
\providecommand{\dpdc}[4][]{\ensuremath{\mathinner{
\left( \dfrac{\partial{^{#1}}#2}{\partial{#3^{#1}}} \right)_{#4}
}}}


%%%%%%





%%%%%%%%%%%%%%%%%%%
% some templates for various things
\begin{comment}

% template for figures
\begin{figure}
\centering
\includegraphics{myfile.png}
\caption{This is a caption}
\label{fig:myfigure}
\end{figure}

% template for Feynman diagrams using feynmf/feynmp
\begin{fmfgraph*}(40,25)
\fmfleft{em,ep}
\fmf{fermion}{em,Zee,ep}
\fmf{photon,label=$Z$}{Zee,Zff}
\fmf{fermion}{fb,Zff,f}
\fmfright{fb,f}
\fmfdot{Zee,Zff}
\end{fmfgraph*}

% template for drawing plots with pgfplot
\pgfplotsset{compat=1.3,compat/path replacement=1.5.1}
\begin{tikzpicture}
\begin{axis}[
extra x ticks={-2,2},
extra y ticks={-2,2},
extra tick style={grid=major}]
\addplot {x};
\draw (axis cs:0,0) circle[radius=2];
\end{axis}
\end{tikzpicture}

%% find package for easily drawing mapping / algebraic / commutative diagrams..

\end{comment}
%%%%%%%%%%%%%%%%%%%



%%%%% A note on spacing
% 5) \qquad
% 4) \quad
% 3) \thickspace = \;
% 2) \medspace = \:
% 1) \thinspace = \,
% -1) \negthinspace = \!
% -2) \negmedspace
% -3) \negthickspace




\title{Phys 220A -- Classical Mechanics -- Lec18}
\author{UCLA, Fall 2014}
\date{\formatdate{11}{12}{2014}} % Activate to display a given date or no date (if empty),
         % otherwise the current date is printed 

\begin{document}
\setlength{\unitlength}{1mm}
\maketitle


% Note on missing lectures: No lecture Nov. 11 (Veteran's day), anomalous missing day on Nov. 4 (election day?)
% Plus extra day from Thurs-Fri first week, minus one day from Thanksgiving 
% --> matches up with number of quantum lectures (just not sure about anomalous missing day)

\section{More on fluids}

\textbf{Note:} Office hours will be held during finals week on Monday/Tuesday from 1--2 PM.

\subsection{Ideal fluids}

In ideal fluids, our equations of motion are generally encompassed by the continuity equation
\begin{eqn}
\pd{\rho}{t} + \v \nabla \cdot (\rho \v v) = 0
\end{eqn}
and Euler's equation 
\begin{eqn}
\rho \od{\v v}{t} + \rho (\v v \cdot \v \nabla) \v v = - \v \nabla p.
\end{eqn}
One simplifying assumption (though generally not a very good approximation) is that the fluid is incompressible, $\rho = \text{const}$, so that $\v \nabla \cdot \v v = 0$ from the continuity equation. Another simplifying assumption is that the fluid is isentropic, 
\begin{eqn}
\frac{\v \nabla p}{\rho} = \v \nabla h.
\end{eqn}
In both cases taking the curl of Euler's equation gives an equation for $\v v$,
\begin{eqn}
\pd{}{t} (\v \nabla \times \v v) = \v \nabla \times \left( \v v \times (\v \nabla \times \v v) \right)
\end{eqn}
It is useful to define the \emph{vorticity} $\v \Omega = \v \nabla \times \v v$, then another simplifying assumption is that the fluid is \emph{non-rotational}, i.e. $\v \Omega = 0$. If the fluid is incompressible and non-rotational, we have
\begin{eqn}
\v \nabla \times \v v = 0, \qquad
\v \nabla \cdot \v v = 0
\end{eqn}
so that we can find a potential for $\v v$ so that $\v v = \v \nabla \phi$ which satisfies $\nabla^2 \phi = 0$. This is not a very realistic approximation but it does allow very simple solutions, e.g. electrostatic vacuum solutions. 

\begin{theorem}[Kelvin's theorem: Conservation of circulation]
Define the \emph{circulation} $\Gamma_C$ of a closed loop $C$ as
\begin{eqn}
\Gamma_C = \oint_C \v v \cdot \dif{\v x}.
\end{eqn}
For an isentropic fluid, the circulation is conserved, $\tod{\Gamma}{t} = 0$. 
\end{theorem}

\begin{proof}
The rate of change in circulation is 
\begin{align}
\od{\Gamma_C}{t} &= \oint_C \od{\v v}{t} \cdot \dif{\v x} + \oint_C \v v \cdot \dif{\vd x} \\
	&= -\oint_C \Big[ (\v v \cdot \v \nabla) \v v + \underbrace{\v \nabla h}_{\int \to 0} \Big] + \frac{1}{2} \underbrace{\oint_C \dif(\v v^2)}_{=0} \\
	&= - \int_\Sigma \v \nabla \times \left( (\v v \cdot \v \nabla) \v v \right) \cdot \dif{\v \Sigma}
\end{align}
We should be able to show that this is then
\begin{eqn}
\od{\Gamma_C}{t} = \int_\Sigma \v \nabla \times \left( \v v \times (\v \nabla \times \v v) \right) \cdot \dif{\v \Sigma} = 0,
\end{eqn}
but we can't seem to show this directly in class. 
\end{proof}


\subsection{Viscosity, non-ideal fluids}

How do we work with viscous fluids? Suppose we have a fluid in a container with a fixed bottom a distance $d$ from top of the fluid, and suppose we move a plate slowly across the top. In a viscous fluid, the top layer will move along with the plate with speed $u$, while the bottom layer will be stationary, inducing a force per area
\begin{eqn}
\frac{F}{A} = \eta \, \frac{u}{d}.
\end{eqn}
The coefficient $\eta$ is the \emph{dynamic viscosity} or \emph{viscosity coefficient}. How do we incorporate the viscosity into the equations of motion? It's not entirely clear. It turns out that it is most useful rewrite the equations in components
\begin{eqn}
\pd{}{t} (\rho v_i) = - \pd{}{x_j} \Pi_{ij},
\end{eqn}
where 
\begin{eqn}
\Pi_{ij} = \delta_{ij} p + \rho v_i v_j - \sigma'_{ij} = -\sigma_{ij} + \rho v_i v_j
\end{eqn}
and $\sigma'_{ij}$ is called the \emph{viscous stress tensor} (and $\sigma_{ij} = \sigma'_{ij} - p \delta_{ij}$).

We can write an ansatz for the stress tensor
\begin{eqn}
\sigma'_{ij} = \alpha_1 \pd{v_i}{x_j} + \alpha_2 \pd{v_j}{x_i} + \alpha_3 \delta_{ij} \sum_k \partial_k v_k. 
\end{eqn}
This can be seen as sort of a spin-0 tensor plus a spin-2 tensor for some reason. Furthermore we can rewrite as
\begin{eqn}
\sigma'_{ij} = \eta \left( \pd{v_i}{x_j} + \pd{v_j}{x_i} - \frac{2}{3} \delta_{ij} \sum_k \partial_k v_k \right) + \rho \delta_{ij} \sum_k \partial_k v_k
\end{eqn}
where $n$ turns out to be the ordinary viscosity coefficient. For an incompressible fluid $\v \nabla \cdot \v v = 0$ so that the $\rho$ term goes away. Finally we can plug into the tensor equation for the general equation, where we simplify assuming $n, \rho$ are constant and uniform so that
\begin{eqn}
\rho \left( \pd{\v v}{t} + (\v v \cdot \v \nabla) \v v \right) = - \v \nabla p + n \nabla^2 \v v + (\rho + \frac{\eta}{3}) \v \nabla (\v \nabla \cdot \v v),
\end{eqn}
which is the famous \emph{Navier-Stokes equation}. Notice that the last term drops out for incompressible fluids. Now, taking the curl we obtain an equation in terms of the vorticity $\v \Omega$,
\begin{eqn}
\pd{\v \Omega}{t} = \v \nabla \times (\v v \times \v \Omega) + \frac{\eta}{\rho} \, \nabla^2 \v \Omega.
\end{eqn}

Now, suppose we rescale $\v v \to \alpha \v v$ and $\v x \to \beta \v x$, so that $\v \Omega \to (\alpha / \beta) \v \Omega$ and $t \to (\beta / \alpha) t$. Then we obtain a new equation
\begin{eqn}
(\alpha / \beta)^2 \pd{\v \Omega}{t} = (\alpha / \beta)^2 \v \nabla \times (\v v \times \v \omega) - \frac{\eta}{\rho \beta^2} \nabla^2 \v \Omega (\alpha / \beta),
\end{eqn}
which we can rewrite 
\begin{eqn}
\pd{\v \Omega}{t} = \v \nabla \times (\v v \times \v \Omega) - \frac{\eta}{\rho \alpha \beta} \nabla^2 \v \Omega.
\end{eqn}
Then we define the Reynolds number by
\begin{eqn}
R = \frac{\eta D v}{\rho}.
\end{eqn}




\end{document}
