% declare document class and geometry
\documentclass[12pt]{article} % use larger type; default would be 10pt
\usepackage[margin=1in]{geometry} % handle page geometry

% import packages and commands
\input{../header2.tex}

\title{Phys 221A -- Quantum Mechanics -- HW07}
\author{UCLA, Fall 2014}
\date{\formatdate{07}{12}{2014}} % Activate to display a given date or no date (if empty),
         % otherwise the current date is printed 

\begin{document}
\maketitle



\begin{em}
Problems 3.3 (p. 256), 3.9 (p. 256), 3.15 (p. 257), and 3.18 (p. 258) from the textbook (Sakurai and Napolitano, Modern Quantum Mechanics, 2nd edition).
\end{em}



\section*{Problem 3.3}
\begin{em}
Consider the $2 \times 2$ matrix defined by
\begin{eqn}
U = \frac{a_0 + i \v \sigma \cdot \v a}{a_0 - i \v \sigma \cdot \v a},
\end{eqn}
where $a_0$ is a real number and $\v a$ is a three-dimensional vector with real components.
\end{em}


\begin{enumproblem}

% part A
\item \begin{em}
Prove that $U$ is unitary and unimodular.
\end{em}


% part B
\item \begin{em}
In general, a $2 \times 2$ unitary unimodular matrix represents a rotation in three dimensions. Find the axis and angle of rotation appropriate for $U$ in terms of $a_0$, $a_1$, $a_2$, and $a_3$. 
\end{em}


\end{enumproblem}



\section*{Problem 3.9}
\begin{em}
Consider a sequence of Euler rotations represented by
\begin{align}
\mathcal{D}^{(1/2)} (\alpha, \beta, \gamma) 
	&= \exp \left( \frac{-i \sigma_3 \alpha}{2} \right) \exp \left( \frac{-i \sigma_2 \beta}{2} \right) \exp \left( \frac{-i \sigma_3 \gamma}{2} \right) \\
	&= 
	\begin{pmatrix}
	e^{-i(\alpha + \gamma) / 2} \cos \frac{\beta}{2} & -e^{-i(\alpha - \gamma) / 2} \sin \frac{\beta}{2} \\
	e^{i(\alpha + \gamma) / 2} \sin \frac{\beta}{2} & e^{i(\alpha - \gamma) / 2} \cos \frac{\beta}{2}
	\end{pmatrix}.
\end{align}
Because of the group properties of rotations, we expect that this sequence of operations is equivalent to a \emph{single} rotation about some axis by an angle $\theta$. Find $\theta$.
\end{em}



\section*{Problem 3.15}

\begin{enumproblem}

% part A
\item \begin{em}
Let $J$ be angular momentum. (It may stand for orbital $\v L$, spin $\v S$, or $\v J_\mathrm{total}$.) Using the fact that $J_x, J_y, J_z (J_\pm \equiv J_x \pm i J_y)$ satisfy the usual angular-momentum commutation relations, prove
\begin{eqn}
\v J^2 = J_z^2 + J_+ J_- - \hbar J_z.
\end{eqn}
\end{em}


% part B
\item \begin{em}
Using (a) (or otherwise), derive the ``famous'' expression for the coefficient $c_-$ that appears in 
\begin{eqn}
J_- \psi_{jm} = c_- \psi_{j,m-1}.
\end{eqn}
\end{em}


\end{enumproblem}



\section*{Problem 3.18}
\begin{em}
A particle in a spherically symmetrical potential is known to be in an eigenstate of $\v L^2$ and $L_z$ with eigenvalues $\hbar^2 l (l+1)$ and $m \hbar$, respectively. Prove that the expectation values between $\ket{lm}$ states satisfy
\begin{eqn}
\avg{L_x} = \avg{L_y} = 0, \quad
\avg{L_x^2} = \avg{L_y^2} = \frac{1}{2} \left[ l (l+1) \hbar^2 - m^2 \hbar^2 \right].
\end{eqn}
Interpret this result semiclassicaly.
\end{em}





\end{document}
