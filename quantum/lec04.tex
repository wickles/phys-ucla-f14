% declare document class and geometry
\documentclass[12pt]{article} % use larger type; default would be 10pt
\usepackage[margin=1in]{geometry} % handle page geometry

% import packages and commands
% standard packages
\usepackage{graphicx} % support the \includegraphics command and options
\usepackage{amsmath} % for nice math commands and environments

% font packages
\usepackage{amssymb} % for \mathbb, \mathfrak fonts
\usepackage{mathrsfs} % for \mathscr font
\DeclareMathAlphabet{\mathpzc}{OT1}{pzc}{m}{it} % defines \mathpzc for Zapf Chancery (standard postscript) font

% other packages
\usepackage{datetime} % allows easy formatting of dates, e.g. \formatdate{dd}{mm}{yyyy}
\usepackage{caption} % makes figure captions better, more configurable
\usepackage[squaren]{SIunits} % for nice units formatting e.g. \unit{50}{\kilo\gram}
\usepackage{cancel} % for crossing out terms with \cancel
\usepackage{verbatim} % for verbatim and comment environments
\usepackage{tensor} % for \indices e.g. M\indices{^a_b^{cd}_e}, and \tensor e.g. \tensor[^a_b^c_d]{M}{^a_b^c_d}
\usepackage{feynmp-auto} % for Feynman diagrams. 
\usepackage{pgfplots} % for plotting in tikzpicture environment

% new commands
\newcommand{\fslash}[1]{#1\!\!\!/} % feynman slash
\newcommand{\opname}[1]{\operatorname{#1}} % custom operator names
\newcommand{\pd}{\partial} % partial differential shortcut
\newcommand{\ket}[1]{\left| #1 \right>} % for Dirac kets
\newcommand{\bra}[1]{\left< #1 \right|} % for Dirac bras
\newcommand{\braket}[2]{\left< #1 \vphantom{#2} \right| 
	\left. #2 \vphantom{#1} \right>} % for Dirac brackets

\let\vaccent=\v % rename builtin command \v{} to \vaccent{}
%\renewcommand{\v}[1]{\ensuremath{\mathbf{#1}}} % for vectors
\renewcommand{\v}[1]{\ensuremath{\boldsymbol{\mathbf{#1}}}} % for vectors
%\newcommand{\gv}[1]{\ensurmath{\mbox{\boldmath$ #1 $}}} % for vectors of Greek letters
\newcommand{\uv}[1]{\ensuremath{\boldsymbol{\mathbf{\widehat{#1}}}}} % for unit vectors
\newcommand{\abs}[1]{\left| #1 \right|} % for absolute value ||x||
\newcommand{\norm}[1]{\left\Vert #1 \right\Vert} % for norm ||v||
\newcommand{\avg}[1]{\left< #1 \right>} % for average <x>
\newcommand{\inner}[2]{\left< #1, #2 \right>} % for inner product <x,y>
\newcommand{\set}[1]{ \left\{ #1 \right\} } % for sets {a,b,c,...}

% shortcuts
\newcommand{\reals}{\mathbb{R}} % real numbers
\newcommand{\complexes}{\mathbb{C}} % complex numbers
\newcommand{\nats}{\mathbb{N}} % natural numbers
\newcommand{\irrats}{\mathbb{Q}} % irrationals
\newcommand{\quats}{\mathbb{H}} % quaternions (a la Hamilton)
\newcommand{\euclids}{\mathbb{E}} % Euclidean space
\newcommand{\bigo}{\mathcal{O}} % big O notation





%%%%%%%%%%%%%%%%%%%
% some templates for various things
\begin{comment}

% template for figures
\begin{figure}
\centering
\includegraphics{myfile.png}
\caption{This is a caption}
\label{fig:myfigure}
\end{figure}

% template for Feynman diagrams using feynmf/feynmp
\begin{fmfgraph*}(40,25)
\fmfleft{em,ep}
\fmf{fermion}{em,Zee,ep}
\fmf{photon,label=$Z$}{Zee,Zff}
\fmf{fermion}{fb,Zff,f}
\fmfright{fb,f}
\fmfdot{Zee,Zff}
\end{fmfgraph*}

% template for drawing plots with pgfplot
\pgfplotsset{compat=1.3,compat/path replacement=1.5.1}
\begin{tikzpicture}
\begin{axis}[
extra x ticks={-2,2},
extra y ticks={-2,2},
extra tick style={grid=major}]
\addplot {x};
\draw (axis cs:0,0) circle[radius=2];
\end{axis}
\end{tikzpicture}

\end{comment}
%%%%%%%%%%%%%%%%%%%


% title information
\title{Phys 221A -- Quantum Mechanics -- Lec04}
\author{UCLA, Fall 2014}
\date{\formatdate{15}{10}{2014}} % Activate to display a given date or no date (if empty),
         % otherwise the current date is printed 

\begin{document}
\maketitle


\section{Continuous variables}

It's possible to choose an orthonormal basis $\set{\ket{\xi}}$ such that 
\begin{equation}
\braket{\xi}{\xi'} = \delta(\xi - \xi').
\end{equation}
Then for any $\ket{\psi} \in G$ we have
\begin{equation}
\ket{\psi} = \int d\xi ~ c_\xi \ket{\xi}.
\end{equation}
This immediately implies the closure relation
\begin{equation}
\int d\xi \ket{\xi} \bra{\xi} = 1,
\end{equation}
and furthermore we have
\begin{equation}
\braket{\xi'}{\psi} = \int d\xi c_\xi \braket{\xi'}{\xi} = c_{\xi'}.
\end{equation}
We can check the closure relation:
\begin{equation}
\int d\xi \ket{\xi} \braket{\xi}{\psi} = \int d\xi ~ c_\xi \ket{\xi} = \ket{\psi},
\end{equation}
which is what we wanted.

Now, given an operator $O$, we can write $O(\xi, \xi') = \bra{\xi} O \ket{\xi'}$. If we have $O\ket{\psi} = \ket{\psi'}$ and 
\begin{equation}
\int d\xi ~ c_\xi \ket{\xi}, \qquad \int d\xi ~ b_\xi \ket{\xi},
\end{equation}
then we have
\begin{eqn}
b_\xi = \int d\xi' ~ O(\xi, \xi') c_{\xi'}.
\end{eqn}

In the position representation we have $\xi \rightarrow \v{r}$ in $n$-dimensional space. So
\begin{eqn}
\ket{\psi} \rightarrow c_{\v{r}} = \braket{\v{r}}{\psi} \equiv \psi(\v{r})
\end{eqn}
or we can write
\begin{eqn}
\ket{\psi} = \int d^3r ~ \psi(\v{r}) \ket{\v{r}}.
\end{eqn}
In the momentum representation we would have
\begin{eqn}
\ket{\psi} = \int d^3p ~ \psi(\v{p}) \ket{\v{p}}.
\end{eqn}
Then if we measure the position, the probability density is given by $\abs{\psi(\v{r})}^2$ and the probability to find the particle in a given volume $dV$ is given by $\abs{\psi(\v{r})}^2 dV$. This of course assumes normalization,
\begin{eqn}
\int d^3r \abs{\psi(\v{r})}^2 = 1, \qquad \text{or} \qquad \braket{\psi}{\psi} = 1.
\end{eqn}


\section{Spinless particle in 1D space}

Classically we would have the Lagrangian $\mathcal{L}(x, \dot{x}, t)$. Quantum mechanically, we will have a Hilbert space $L_2$ over $c$-valued function $f(x)$. That is, functions in the space have the property
\begin{eqn}
\int dx \abs{f(x)}^2 < \infty,
\end{eqn}
with inner product given by
\begin{eqn}
\braket{f}{g} = \int dx f^*(x) g(x).
\end{eqn}
Physical states will be normalized in this space,
\begin{eqn}
\braket{\psi}{\psi} = 1, \qquad \text{or} \qquad \int dx \abs{\psi(x)}^2 = 1.
\end{eqn}

Let's look at some operators in this space. We find that $x$ is a Hermitian operator:
\begin{eqn}
\int dx (x f(x))^* g(x) \int dx f^*(x) (x g(x)).
\end{eqn}
We find that $d/dx$ is anti-Hermitian:
\begin{eqn}
\int dx (\frac{d}{dx} f(x))^* g(x) = \int dx f^*(x) (-\frac{d}{dx}) g(x)
\end{eqn}
so that $(d/dx)^\dagger = -d/dx$ and $i d/dx$ is Hermitian. 

We will, for now, \textit{define} the linear momentum: $p = -i \hbar d/dx$. Later we will see that this reduces to the classical momentum. Furthermore, the momentum part of the Hamiltonian will typically be given by
\begin{eqn}
H = \frac{p^2}{2m} = -\frac{\hbar}{2m} \partial_x^2.
\end{eqn}
In general $H(\v{r},\v{p})$ will be the same in classical and quantum mechanics. 

We also have the commutation relation
\begin{eqn}
[x,p] = i\hbar. 
\end{eqn}
This gives us the Heisenberg uncertainy principle,
\begin{eqn}
\avg{(\Delta x)^2} \avg{(\Delta p)^2} \geq \hbar^2 / 4.
\end{eqn}
A Gaussian wave packet will take the form
\begin{eqn}
\psi(x) = \frac{1}{\pi^{1/4} \sqrt{d}} ~ e^{ikx - x^2 / 2d}
\end{eqn}
for all $k, d \in \reals$ such that $d > 0$. Gaussian wave packets minimize the uncertainty,
\begin{eqn}
\avg{(\Delta x)^2} \avg{(\Delta p)^2} = \hbar^2 / 4.
\end{eqn}

In the position representation we might have an equation
\begin{eqn}
x g_y(x) = y g_y(x) 
\end{eqn}
for any fixed $y \in \reals$. Then $y$ is the eigenvalue of $x$ and $g_y(x)$ is the eigenfunction of $x$, and we have
\begin{eqn}
g_y(x) = \delta(x-y) \equiv \ket{y}
\end{eqn}
and 
\begin{eqn}
\braket{y}{y'} = \int dx \delta(x-y) \delta(x-y') = \delta(y-y').
\end{eqn}

Next we have the eigenvalue equation
\begin{eqn}
(-i \hbar \frac{d}{dx}) \ket{f_p} = p \ket{f_p}.
\end{eqn}
The eigenstate will be given by 
\begin{eqn}
f_p (x) = e^{ipx / \hbar} / \sqrt{2\pi \hbar}
\end{eqn}
where we can write a wavelength $\lambda = 2 \pi \hbar / p$ which is just the de Broglie wavelength. Furthermore we have
\begin{eqn}
\braket{f_p}{f_{p'}} = \int dx \frac{1}{2\pi \hbar} e^{i(p-p') x / \hbar} = \delta(p - p').
\end{eqn}
Recall that we can write
\begin{eqn}
\delta(k) = \int \frac{dx}{2\pi} e^{ikx}.
\end{eqn}

We can write down a translation operator in $L_2$,
\begin{eqn}
U(a) = e^{-a \partial_x} = 1 - a \partial_x + (a/2) \partial_x^2 + \dots
\end{eqn}
which acts on functions by translating the position by a constant,
\begin{eqn}
U(a) f(x) = f(x-a).
\end{eqn}
We can write 
\begin{eqn}
\partial_a U(a) \Big|_{a \rightarrow 0} = -\partial_x
\end{eqn}
so we say that $\partial_x$ is the \textit{generator} of translations. Up to a constant, we can say that the momentum $p$ is the generator of translation,
\begin{eqn}
p = i \hbar \partial_a U(a) \Big|_{a \rightarrow 0}.
\end{eqn}


\section{Quantum Dynamics}

Temporal evolution will take us from $\ket{\psi(0)} \mapsto \ket{\psi(t)}$. We will consider $t=0$ the present, $t<0$ the past, and $t>0$ the future, and denote the present state $\ket{\psi(0)} = \ket{\psi}$. We will be able to write time evolution as an operator
\begin{eqn}
\ket{\psi} \mapsto \ket{\psi(t)} \equiv U(t) \ket{\psi}.
\end{eqn}
Because the state should be physical and thus normalized at all times, we have
\begin{eqn}
1 = \braket{U(t) \psi}{U(t) \psi} = \bra{\psi} U^\dagger U \ket{\psi}
\end{eqn}
so $U(t)$ is a unitary operator. 

Finally, we can write the Schroedinger equation,
\begin{eqn}
i \hbar \frac{d}{dt} \ket{\psi(t)} = H(t) \ket{\psi(t)}
\end{eqn}
for any physical state $\ket{\psi}$, where $H(t)$ is a Hermitian operator called the Hamiltonian (aka the energy operator). We can rewrite this as
\begin{eqn}
\left[ i \hbar \frac{d}{dt} U(t) \right] \ket{\psi} = \left[ H(t) U(t) \right] \ket{psi},
\end{eqn}
which is true for any $\ket{\psi}$, so we can write the Schroedinger equation instead as an equation for $U(t)$,
\begin{eqn}
i \hbar \frac{d}{dt} U(t) = H(t) U(t), \quad \text{where} \quad U(0) = 1.
\end{eqn}
We can solve this pretty easily -- consider the differential equation
\begin{eqn}
\frac{d}{dt} f(t) = h(t) f(t).
\end{eqn}
We can solve this by rewriting
\begin{eqn}
df / f = h ~ dt
\end{eqn}
which has solution 
\begin{eqn}
\log(f(t) / f(0)) = \int_0^t h(t') dt'.
\end{eqn}
Thus applying this to the equation for $U(t)$ we find the time evolution operator,
\begin{eqn}
U(t) = \exp \left( -\frac{i}{\hbar} \int_0^t H(t') dt' \right). 
\end{eqn}









\end{document}
