% declare document class and geometry
\documentclass[12pt]{article} % use larger type; default would be 10pt
\usepackage[margin=1in]{geometry} % handle page geometry

% import packages and commands
\input{../header2.tex}

\title{Phys 221A -- Quantum Mechanics -- HW08}
\author{UCLA, Fall 2014}
\date{\formatdate{10}{12}{2014}} % Activate to display a given date or no date (if empty),
         % otherwise the current date is printed 

\begin{document}
\maketitle



\begin{em}
Problems 3.5 (p. 256), 3.6 (p. 256), 3.13 (p. 257), 3.26 [p. 260; for part (c), see definition in Eq. (3.5.51)], and 3.33 (p. 261) from the textbook (Sakurai and Napolitano, Modern Quantum Mechanics, 2nd edition).
\end{em}



\section{Problem 3.5}
\begin{em}
Consider a spin 1 particle. Evaluate the matrix elements of 
\begin{eqn}
S_z (S_z + \hbar) (S_z - \hbar)
\quad \text{and} \quad
S_x (S_x + \hbar) (S_x - \hbar).
\end{eqn}
\end{em}



\section{Problem 3.6}
\begin{em}
Let the Hamiltonian of a rigid body by 
\begin{eqn}
H = \frac{1}{2} \left( \frac{K_1^2}{I_1} + \frac{K_2^2}{I_2} + \frac{K_3^2}{I_3} \right),
\end{eqn}
where $\v K$ is the angular momentum in the body frame. From this expression obtain the Heisenberg equation of motion for $\v K$, and then find Euler's equation of motion in the correspondence limit. 
\end{em}



\section{Problem 3.13}
\begin{em}
An angular-momentum eigenstate $\ket{j,m = m_\mathrm{max} = j}$ is rotated by an infinitesimal angle $\varepsilon$ about the $y$-axis. Without using the explicit form of the $d^{(j)}_{m'm}$ function, obtain an expression for the probability for the new rotated state to be found in the original state up to terms of order $\varepsilon^2$.
\end{em}



\section{Problem 3.26}

\begin{enumproblem}

% part A
\item \begin{em}
Consider a system with $j=1$. Explicitly write
\begin{eqn}
\matrixel{j=1,m'}{J_y}{j=1,m}
\end{eqn}
in $3 \times 3$ matrix form. 
\end{em}


% part B
\item \begin{em}
Show that for $j=1$ only, it is legitimate to replace $e^{-iJ_y \beta / \hbar}$ by 
\begin{eqn}
1 - i \left( \frac{J_y}{\hbar} \right) \sin \beta - \left( \frac{J_y}{\hbar} \right)^2 (1 - \cos\beta).
\end{eqn}
\end{em}


% part C
\item \begin{em}
Using (b), prove
\begin{eqn}
d^{(j=1)} (\beta) = 
\begin{pmatrix}
\frac{1}{2} (1+\cos\beta) & -\frac{1}{\sqrt 2} \sin\beta & \frac{1}{2} (1-\cos\beta) \\
\frac{1}{\sqrt 2} \sin\beta & \cos \beta & -\frac{1}{\sqrt 2} \sin\beta \\
\frac{1}{2} (1-\cos\beta) & \frac{1}{\sqrt 2} \sin\beta & \frac{1}{2} (1+\cos\beta)
\end{pmatrix}.
\end{eqn}
\end{em}


\end{enumproblem}



\section{Problem 3.33}
\begin{em}
A spin $\frac{3}{2}$ nucleus situated at the origin is subjected to an external inhomogeneous electric field. The basic electric quadrupole interaction may be taken to be
\begin{eqn}
H_\mathrm{int} = \frac{eQ}{2s(s-1)\hbar^2} \left[ \left( \pd[2]{\phi}{x} \right)_0 S_x^2 + \left( \pd[2]{\phi}{y} \right)_0 S_y^2 + \left( \pd[2]{\phi}{z} \right)_0 S_z^2 \right],
\end{eqn}
where $\phi$ is the electrostatic potential satisfying Laplace's equation, and the coordinate axes are chosen such that
\begin{eqn}
\left( \md{\phi}{2}{x}{}{y}{} \right)_0 = \left( \md{\phi}{2}{y}{}{z}{} \right)_0 = \left( \md{\phi}{2}{x}{}{z}{} \right)_0 = 0.
\end{eqn}
Show that the interaction energy can be written as
\begin{eqn}
A (3 S_z^2 - \v S^2) + B (S_+^2 + S_-^2),
\end{eqn}
and express $A$ and $B$ in terms of $(\partial^2 \phi / \partial x^2)_0$ and so on. Determine the energy eigenkets (in terms of $\ket{m}$, where $m = \pm \frac{3}{2}, \pm \frac{1}{2}$) and the corresponding energy eigenvalues. Is there any degeneracy?
\end{em}





\end{document}
