% declare document class and geometry
\documentclass[12pt]{article} % use larger type; default would be 10pt
\usepackage[margin=1in]{geometry} % handle page geometry

% standard packages
\usepackage{graphicx} % support the \includegraphics command and options
\usepackage{amsmath} % for nice math commands and environments

% font packages
\usepackage{amssymb} % for \mathbb, \mathfrak fonts
\usepackage{mathrsfs} % for \mathscr font
\DeclareMathAlphabet{\mathpzc}{OT1}{pzc}{m}{it} % defines \mathpzc for Zapf Chancery (standard postscript) font

% other packages
\usepackage{datetime} % allows easy formatting of dates, e.g. \formatdate{dd}{mm}{yyyy}
\usepackage{caption} % makes figure captions better, more configurable
\usepackage[squaren]{SIunits} % for nice units formatting e.g. \unit{50}{\kilo\gram}
\usepackage{cancel} % for crossing out terms with \cancel
\usepackage{verbatim} % for verbatim and comment environments
\usepackage{tensor} % for \indices e.g. M\indices{^a_b^{cd}_e}, and \tensor e.g. \tensor[^a_b^c_d]{M}{^a_b^c_d}
\usepackage{feynmp-auto} % for Feynman diagrams. 
\usepackage{pgfplots} % for plotting in tikzpicture environment

% new commands
\newcommand{\fslash}[1]{#1\!\!\!/} % feynman slash
\newcommand{\opname}[1]{\operatorname{#1}} % custom operator names
\newcommand{\pd}{\partial} % partial differential shortcut
\newcommand{\ket}[1]{\left| #1 \right>} % for Dirac kets
\newcommand{\bra}[1]{\left< #1 \right|} % for Dirac bras
\newcommand{\braket}[2]{\left< #1 \vphantom{#2} \right| 
	\left. #2 \vphantom{#1} \right>} % for Dirac brackets

\let\vaccent=\v % rename builtin command \v{} to \vaccent{}
%\renewcommand{\v}[1]{\ensuremath{\mathbf{#1}}} % for vectors
\renewcommand{\v}[1]{\ensuremath{\boldsymbol{\mathbf{#1}}}} % for vectors
%\newcommand{\gv}[1]{\ensurmath{\mbox{\boldmath$ #1 $}}} % for vectors of Greek letters
\newcommand{\uv}[1]{\ensuremath{\boldsymbol{\mathbf{\widehat{#1}}}}} % for unit vectors
\newcommand{\abs}[1]{\left| #1 \right|} % for absolute value ||x||
\newcommand{\norm}[1]{\left\Vert #1 \right\Vert} % for norm ||v||
\newcommand{\avg}[1]{\left< #1 \right>} % for average <x>
\newcommand{\inner}[2]{\left< #1, #2 \right>} % for inner product <x,y>
\newcommand{\set}[1]{ \left\{ #1 \right\} } % for sets {a,b,c,...}

% shortcuts
\newcommand{\reals}{\mathbb{R}} % real numbers
\newcommand{\complexes}{\mathbb{C}} % complex numbers
\newcommand{\nats}{\mathbb{N}} % natural numbers
\newcommand{\irrats}{\mathbb{Q}} % irrationals
\newcommand{\quats}{\mathbb{H}} % quaternions (a la Hamilton)
\newcommand{\euclids}{\mathbb{E}} % Euclidean space
\newcommand{\bigo}{\mathcal{O}} % big O notation





%%%%%%%%%%%%%%%%%%%
% some templates for various things
\begin{comment}

% template for figures
\begin{figure}
\centering
\includegraphics{myfile.png}
\caption{This is a caption}
\label{fig:myfigure}
\end{figure}

% template for Feynman diagrams using feynmf/feynmp
\begin{fmfgraph*}(40,25)
\fmfleft{em,ep}
\fmf{fermion}{em,Zee,ep}
\fmf{photon,label=$Z$}{Zee,Zff}
\fmf{fermion}{fb,Zff,f}
\fmfright{fb,f}
\fmfdot{Zee,Zff}
\end{fmfgraph*}

% template for drawing plots with pgfplot
\pgfplotsset{compat=1.3,compat/path replacement=1.5.1}
\begin{tikzpicture}
\begin{axis}[
extra x ticks={-2,2},
extra y ticks={-2,2},
extra tick style={grid=major}]
\addplot {x};
\draw (axis cs:0,0) circle[radius=2];
\end{axis}
\end{tikzpicture}

\end{comment}
%%%%%%%%%%%%%%%%%%%


\title{Phys 230A -- QFT -- Lec01}
\author{UCLA, Fall 2014}
\date{\formatdate{06}{10}{2014}} % Activate to display a given date or no date (if empty),
         % otherwise the current date is printed 

\begin{document}
\maketitle


\section{Introduction}

Professor: Per Kraus. TA: Ashwin

Main books for course
\begin{itemize}
\item Peskin \& Schroeder
\item Srednicki
\item There's also Schwartz, which is less formal and more conceptual
\item There's also Zee, which is more fun and casual and conceptual
\item And of course Weinberg volumes I, II, III
\item Other books to check out: Ryder, Ramond
\end{itemize}


\subsection{Why QFT?}

In non-relativistic QM, you have a fixed number of particles. Given $n$ particles, we have the wavefunction $\psi(x_1, x_2, \dots, x_n, t)$ with the Schroedinger equation $H\psi = i\hbar (\pd \psi / \pd t)$ and the probability density $|\psi|^2$. This treatment is not necessarily non-relativistic, in fact it could be relativistic if given the proper Hamiltonian. 

However, non-relativistic QM cannot treat particle creation/annihilation e.g. pair production. This is because we need an infinite number of degrees of freedom. We must introduce a field $\phi(x^\mu)$, which provides a degree of freedom per spacetime point. But we can have fields and particles classically, so how is this different? Classically, they are distinct; in QFT, particles are the quanta of the fields. What exactly does this mean? We will see that particles are localized eigenstates of the Hamiltonian with a definite mass and momentum. 


\subsection{Quantization of free scalar field}

We will use the metric convention $\eta_{\mu\nu} = (+,-,-,-)$ and units $\hbar = c = 1$. Start with a classical variable: $\phi(x^\mu)$. We want a Lorentz invariant equation of motion for $\phi$. Denote a Lorentz transform $x^\mu \rightarrow x'^\mu = \Lambda\indices{^\mu_\nu} x^\nu$. It preserves the dot product: $\eta_{\mu\nu} x'^\mu x'^\nu = \eta_{\mu\nu} x^\mu x^\nu$, thus $\eta_{\mu\nu} \Lambda\indices{^\mu_\alpha} \Lambda\indices{^\nu_\beta} = \eta_{\alpha\beta}$.

We can show that $\eta_{\mu\nu} \Lambda\indices{^\mu_\alpha} \Lambda\indices{^\nu_\beta} = \eta_{\alpha\beta}$, or in other words $\Lambda^\top \eta \Lambda = \eta$. Furthermore, $\Lambda^{-1} = \eta \Lambda^\top \eta$, so 
\begin{equation}
(\Lambda^{-1})\indices{^\mu_\nu} = \eta^{\mu\alpha} \Lambda\indices{^\beta_\alpha} \eta_{\beta\nu} = \Lambda\indices{_\nu^\mu}.
\end{equation}
Then we find
\begin{equation}
\frac{\pd}{\pd x'^\mu} = \frac{\pd x^\nu}{\pd x'^\mu} \frac{\pd}{\pd x^\nu} = (\Lambda^{-1})\indices{^\nu_\mu} \frac{\pd}{\pd x^\nu} = \Lambda\indices{_\mu^\nu} \frac{\pd}{\pd x^\nu}
\end{equation}
and thus
\begin{equation}
\eta^{\mu\nu} \frac{\pd}{\pd x'^\mu} \frac{\pd}{\pd x'^\nu} = \eta^{\mu\nu} \frac{\pd}{\pd x^\mu} \frac{\pd}{\pd x^\nu}
\end{equation}

Now we can define the Lorentz invariant D'Alembertian operator: $\Box = \pd^\mu \pd_\mu$. We can also write down the Klein-Gordon equation
\begin{equation}
\Box \phi(x) + m^2 \phi(x) = 0.
\end{equation}
It IS Lorentz invariant --- if $\phi(x^\mu)$ is a solution, then so is $\phi'(x^\mu) = \phi(\Lambda\indices{^\mu_\nu} x^\nu)$. Is this the most general equation? It is the most general Lorentz invariant, \underline{scalar}, \underline{linear}, and with at most \underline{two derivatives}. Why isn't there another term like $\alpha \Box^2 \phi$? Theoretically it could be there but in real life the higher coefficients are very small. 

We can do some dimensional analysis in mass units: $[m] = 1$, $[m^2] = 2$, $[\pd / \pd x] = 1$ (recall Compton wavelength has units $[\hbar/mc] = -1$), and $[\Box] = 2$. Notice that in our higher order derivative version of the KG equation, we would have $[\alpha] = -2$. Since the deviations are not seen in experiments, we must have $\alpha \sim \Lambda^{-2}$ where $\Lambda$ is the UV cutoff, which is some very large energy in modern theories and so the terms are highly suppressed. 


\subsection{Variational principles, functional derivatives}

Consider a Newtonian particle: $m\ddot{x} = -dV/dx$. Then the action is given by $S = \int_{t_1}^{t_2} dt L(x, \dot{x})$ where $L = (1/2)m\dot{x}^2 - V(x)$. The action principle of course says that Newton's 2nd law holds iff $\delta S / \delta x(t) = 0$ subject to $x(t_1) = x_1$, $x(t_2) = x_2$. 

Let's talk about the functional derivative. A \textit{function} $f(x_1, \dots, x_n)$ maps $n$ numbers to a number. A \textit{functional} maps a function with some boundary conditions to a number. Examples:
\begin{equation}
F[x(t)] = \int_{t_1}^{t_2} dt x^2(t), \qquad \text{or } \int_{t_1}^{t_2} dt (\frac{dx}{dt})^2, \qquad \text{or } x(t_3).
\end{equation}
A \textit{functional derivative} measures the change $\delta F$ in $F$ under change of the function $x(t) \rightarrow x(t) + \delta x(t)$, where $\delta x(t) \ll 1$. With an ordinary function we have
\begin{equation}
f(x+\delta x) = f(x) + \frac{df}{dx} \delta x + \frac{1}{2} \frac{d^2 f}{dx^2} \delta x^2 + \dots
\end{equation}
whereas with a functional we have 
\begin{equation}
F[x+\delta x] = F[x] + \int dt \frac{\delta F}{\delta x(t)} \delta x(t) + \frac{1}{2} \int dt_1 dt_2 \frac{\delta^2 F}{\delta x(t_1) \delta x(t_2)} \delta x(t_1) \delta x(t_2) + \dots.
\end{equation}
Take for example 
\begin{equation}
F[x] = \int_{t_1}^{t_2} dt x(t)^2.
\end{equation}
We have
\begin{align}
F[x+\delta x] &= \int_{t_1}^{t_2} dt [x+\delta x]^2 \\
	&= \int_{t_1}^{t_2} dt [x^2 + 2x \delta x + \delta x^2]
\end{align}
so that, picking out the appropriate expressions from the expansion, we have
\begin{equation}
\frac{\delta F}{\delta x(t)} = 2x(t), \qquad \frac{\delta^2 F}{\delta x(t) \delta x(t')} = 2\delta(t-t')
\end{equation}
since
\begin{equation}
\int_{t_1}^{t_2} dt \delta x(t)^2 = \int_{t_1}^{t_2} dt dt' \delta(t-t') \delta x(t) \delta x(t').
\end{equation}

Question: 
\begin{equation}
\text{Does} \quad \frac{\delta x(t)}{\delta x(t')} = \delta(t-t') \quad ?
\end{equation}
Well, using $x(t) = \int_{t_1}^{t_2} dt' \delta(t-t') x(t')$ we have $\delta x(t) = \int_{t_1}^{t_2} dt' \delta(t-t') \delta x(t')$ so the answer is yes. 

Next example: $F[x(t)] = \int_{t_1}^{t_2} dt (dx/dt)^2$ with boundary conditions $x(t_i) = x_i$ and $\delta x(t_i) = 0$ for $i=1,2$. We have
\begin{equation}
F[x + \delta x] = F[x] - \int_{t_1}^{t_2} dt 2 \frac{d^2 x}{dt^2} \delta x + 2 \frac{dx}{dt} \delta x \Big|_{t_1}^{t_2}
\end{equation}
so $\delta F / \delta x(t) = -2 d^2 x / dt^2$. 

We can also write equations like
\begin{equation}
\delta F[x] = \int dt \frac{\delta F}{\delta x} \delta x.
\end{equation}
Recall we can write the ordinary derivative like
\begin{equation}
\frac{df}{dx} = \lim_{\epsilon \rightarrow 0} \frac{f(x + \epsilon) - f(x)}{\epsilon},
\end{equation}
we can also write something like this for functional derivatives
\begin{equation}
\int dt \frac{\delta F}{\delta x(t)} y(t) = \lim_{\epsilon \rightarrow 0} \frac{F[x+\epsilon y] - F[x]}{\epsilon} = \frac{d}{d\epsilon} F[x+\epsilon y] \Big|_{\epsilon=0}.
\end{equation}

Going back to Newton's equations, writing $F[x] = \int dt V(x(t))$ we have $\delta F / \delta x(t) = V'(x(t))$. Then 
\begin{equation}
S[x] = \int_{t_1}^{t_2} dt [\frac{1}{2} m\dot{x}^2 - V(x)]
\end{equation}
has  the functional derivative
\begin{equation}
\frac{\delta S}{\delta x(t)} = -m\ddot{x}(t) - V'(x(t))
\end{equation}
which vanishes when $m\ddot{x} = -V'(x)$ as desired. 

Next let's take a spacetime field $\phi(x^\mu)$ and vary the functional
\begin{equation}
S[\phi] = \int_R d^4 x [\frac{1}{2} \pd^\mu \phi \pd_\mu \phi - V(\phi)], \qquad \phi \Big|_{\pd R} = \text{fixed}
\end{equation}
where $R$ is some spacetime region. Then
\begin{align}
\delta S &= \int_R d^4x [\pd^\mu \phi \pd_\mu \delta\phi - V'(\phi) \delta\phi] \\
	&= \int_R d^4x [-\pd_\mu \pd^\mu \phi - V'(\phi)] \delta\phi
\end{align}
and we see that $\delta S / \delta\phi$, the part of the integrand multiplying $\delta\phi$, which must be zero, gives us the KG equation. In other words, we have KG iff $\delta S / \delta\phi = 0$. 

The action is written $S = \int d^4x \mathcal{L}(\phi, \pd_\mu \phi)$, so the Lagrangian density $\mathcal{L}$ for KG is $\mathcal{L} = (1/2)(\pd^\mu \phi \pd_\mu \phi - m^2 \phi^2)$.






\begin{comment}
\begin{figure}
\centering
\includegraphics{3a.pdf}
\caption{Half the diagrams for photon-photon scattering.}
\label{fig:3a}
\end{figure}
\end{comment}



\end{document}
